% Include the config
\documentclass[
BCOR=5mm,           % Binderkorrektur von 5mm vorsehen
DIV=10,             % Seite in X Kästchen einteilen (Siehe Koma-Script Guide)
%DIVcalc,           % Besten DIV Wert berechnen (Siehe Koma-Script Guide)
fontsize=11pt,      % Schriftgröße 11 Punkte
oneside,            % Einseitig
parskip,            % Paragraphen nicht einrücken
headsepline,        % Kopfzeile nach unten durch Linie abgrenzen (scrheadings)
%footbotline,       % Fußzeile nach unten durch Linie abgrenzen (scrheadings)
plainheadsepline,   % Kopfzeile nach unten durch Linie abgrenzen (scrplain)
plainfootbotline,   % Fußzeile nach unten durch Linie abgrenzen (scrplain)
%headtopline,       % Kopfzeile nach oben durch Linie abgrenzen (scrheadings)
footsepline,        % Fußzeile nach oben durch Linie abgrenzen (scrheadings)
plainheadtopline,   % Kopfzeile nach oben durch Linie abgrenzen (scrplain)
plainfootsepline,   % Fußzeile nach oben durch Linie abgrenzen (scrplain)
headinclude=false,  % Kopfzeile nicht als Teil des Inhalts setzen
footinclude=false,  % Fußzeile nicht als Teil des Inhalts setzen
%bibtotocnumbered,  % Literaturverzeichnis nummeriert ins Inhaltsverzeichnis aufnehmen
bibliography=totoc, % Literaturverzeichnis ins Inhaltsverzeichnis aufnehmen
%liststotocnumbered,% Sonstige Verzeichnise nummeriert ins Inhaltsverzeichnis aufnehmen
listof=totoc,       % Sonstige Verzeichnise ins Inhaltsverzeichnis aufnehmen
%idxtotocnumbered    % Index nummeriert ins Inhaltsverzeichnis aufnehmen
%idxtotoc           % Index ins Inhaltsverzeichnis aufnehmen
]{scrbook}          % Koma-Script Klasse zum setzen eines Buchs

% Die "Standard-Header" für deutsche Dokumente
\usepackage[utf8]{inputenc}
\usepackage[T1]{fontenc}         % T1 Schriften verwenden (sieht besser aus)
\usepackage[ngerman]{babel}      % Neue deutsche Rechtschreibung und Übersetzungenq

% Für URLs und Path
\usepackage[hyphens,spaces,obeyspaces]{url}

% "Schönere" Schriften einbinden
\usepackage{mathpazo}            % Serifen-Font mit passendem Math-Font
\usepackage[scaled=.95]{helvet}  % Serifenloser Font passend zu mathpazo
\usepackage{courier}             % "Schönerer" Festbreiten-Font

\usepackage[babel,german=quotes]{csquotes}

% Koma-Script Paket zum setzen vom Kopf- und Fußzeilen einbinden
\usepackage{scrlayer-scrpage}

% Seitenstil für normale Seiten auf scrheadings setzen
% Für Kapitelanfang und ähnliches wird scrplain verwendet
\pagestyle{scrheadings}

% Kopf- und Fußzeile löschen
\clearscrheadfoot

% Automarkierungen verwenden \automark[rechts]{links}
% Statt \leftmark und \rightmark kann dann bei
% Koma-Script einfach \headmark verwendet werden
\automark[section]{chapter}

% Kopfzeile für scrplain und scrheadings setzen
% \*head[scrplain]{scrheadings}
%\ihead[Innen]{Innen}
%\chead[Mitte]{Mitte}
\ohead[\sffamily\scshape\bfseries\large\headmark]
{\sffamily\scshape\bfseries\large\headmark}

% Fußzeile für scrplain und scrheadings setzen
% \*foot[scrplain]{scrheadings}
%\ifoot[Innen]{Innen}
%\cfoot[Mitte]{Mitte}
\ofoot[\sffamily\thepage]{\sffamily\thepage}

% Trennlinien für Kopf- und Fußzeile formatieren
% Siehe Optionen der Dokumentklasse
%$\KOMAoptions{headtopline=0pt, footbotline=0pt, headsepline=.4pt,footsepline=.4pt}
\KOMAoptions{headsepline=.4pt,footsepline=.4pt}

% Paket zum Einbinden von Quellcode als Listings
% Hinweis: UTF-8 Encoding führt zu Problemen mit Umlauten
\usepackage{listings}
\usepackage{xcolor}
\usepackage{soul}

% Paket für definierte Übersetzungen einbinden
\usepackage[ngerman]{translator}

% Paket für Stichwort- Abkürzungs- und sonstige Verzeichnisse einbinden
\usepackage[
nonumberlist, % Keine Seitenzahlen anzeigen
acronym,      % Abkürzungsverzeichnis erstellen
%toc,         % In Inhaltsverzeichnis aufnehmen
%section      % Verzeichniseintrag als Section
]{glossaries}

% Paket zum generieren von Blindtext
\usepackage{blindtext}

% Paket zum Einbinden von Bildern
\usepackage{graphicx}

% Paket für Wortindex einbinden
\usepackage{makeidx}

% Andere packages
\usepackage{hyperref}
\usepackage{eurosym}

% Ein eigenes Verzeichnis definieren (Smbolverzeichnis)
% Das Stichwort- und Abkürzungsverzeichnis wird analog vordefiniert
% Siehe makeindex Aufrufe - Hier werden die Dateiendungen festgelegt
\newglossary[slg]{symbolslist}{syi}{syg}{Symbolverzeichnis}

% Den Punkt am Ende der Beschreibung deaktivieren
% \renewcommand*{\glspostdescription}{}

% Stichwort-, Abkürzungs- und Symbolverzeichnis erzeugen
\makeglossaries

% Wortindex erzeugen
\makeindex

%
% WORKAROUND, damit lstlistoflistings funktioniert:
% Quelle: http://www.komascript.de/node/477
%
\makeatletter
\@ifundefined{float@listhead}{}{%
    \renewcommand*{\lstlistoflistings}{%
        \begingroup
    	    \if@twocolumn
                \@restonecoltrue\onecolumn
            \else
                \@restonecolfalse
            \fi
            \float@listhead{\lstlistlistingname}%
            \setlength{\parskip}{\z@}%
            \setlength{\parindent}{\z@}%
            \setlength{\parfillskip}{\z@ \@plus 1fil}%
            \@starttoc{lol}%
            \if@restonecol\twocolumn\fi
        \endgroup
    }%
}
\makeatother

% PDF-Metadaten
\AfterPreamble{
    \hypersetup {
        pdftitle = {Künstliche Intelligenz zur Parameteroptimierung in der Fertigung},
    	pdfsubject = {Künstliche Intelligenz zur Parameteroptimierung in der Fertigung},
    	pdfauthor = {Lars Helmuth Probst},
    	pdfkeywords = {Master, Lars, Helmuth, Probst},
    	pdfcreator = {Lars Helmuth Probst},
    	pdfproducer = {Lars Helmuth Probst},
    	%pdfstartpage = 1,
    	%pdffitwindow = true,
    	%pdfpagelayout = SinglePage
    }
}

% Numbering
\setcounter{secnumdepth}{5}
\setcounter{tocdepth}{5}

%\lstset{aboveskip=20pt,belowskip=0pt}
\lstset{aboveskip=0pt,belowskip=0pt}

% Formatierung der Listings
\lstset{
    captionpos=b,                % Beschriftung unterhalb (bottom)
    frame=trbl,                  % Rahmen zeichnen (top, right, bottom, left)
    basicstyle=\small\ttfamily,  % Festbreitenschrift verwenden (small)
    breaklines=true,
    showstringspaces=false,
    stringstyle={\color{red}},
    commentstyle={\color{purple}},
    keywordstyle={\color{blue}\bfseries},
    keywordstyle=[3]{\color{orange}\bfseries},
    keywordstyle=[4]{\color{teal}\bfseries},
    ndkeywordstyle={\color{darkgray}},
    keywords={
        bash, curl, ibmcloud, sudo, npm, ng, new, serve, add, generate, api, login, cf, apt-get
    }
}

% Define language javascript
\lstdefinelanguage{JavaScript}{
    morekeywords={typeof, new, true, false, catch, function, return, null, catch, switch, var, if, in, while, do, else, case, break, apim, let, string, const, console, require, public, private},
    morekeywords=[3]{use, post, urlencoded, send, listen, log, classify, path, load, setvariable, getvariable, networkService, subscribe},
    morekeywords=[4]{express, app, bodyParser, tf, req, res, mn, predictions, token, request, isConnected},
    ndkeywords={class, export, boolean, throw, implements, import, this},
    numbers=left,
    comment=[l]{//},
    morecomment=[s]{/*}{*/},
    morestring=[b]',
    morestring=[b]",
}

% Define language html
\lstdefinelanguage{HTML}{
    morekeywords={
        % JavaScript
        typeof, new, true, false, catch, function, return, null, catch, switch, var, if, in, while, do, else, case, break,
        % HTML
        html, title, meta, style, head, body, script, canvas, domain, allowedOrigins, allowedMethods, allowedHeaders, allowCredentials, maxAge,
        % CSS
        border:, transform:, -moz-transform:, transition-duration:, transition-property:, transition-timing-function:
    },
    otherkeywords={<, >, \/},
    ndkeywords={class, export, boolean, throw, implements, import, this, cors},
    comment=[l]{//},
    morecomment=[s]{/*}{*/},
    morecomment=[s]{<!}{>},
    morestring=[b]',
    morestring=[b]",
    alsoletter={-},
    alsodigit={:}
}

% Define language json
\lstdefinelanguage{JSON}{
    string=[s]{"}{"},
    stringstyle={\color{blue}\bfseries},
    comment=[l]{:},
}


% Include the glossary
% Define Acronyme
\newacronym{acr:FaaS}{FaaS}{Function as a Service}
\newacronym{acr:CLI}{CLI}{Command Line Interface}
\newacronym{acr:SSO}{SSO}{Single Sign-on}
\newacronym{acr:CORS}{CORS}{Cross-Origin Resource Sharing}
\newacronym{acr:NPM}{NPM}{Node Package Manager}
\newacronym{acr:LTS}{LTS}{Long Term Support}
\newacronym{acr:REST}{REST}{Representational State Transfer}
\newacronym{acr:API}{API}{Application Programming Interface}
\newacronym{acr:HTTP}{HTTP}{Hypertext Transfer Protocol}
\newacronym{acr:CSV}{CSV}{Comma-separated values}
\newacronym{acr:JS}{JS}{JavaScript}
\newacronym{acr:macOS}{macOS}{Macintosh Operating System}
\newacronym{acr:PaaS}{PaaS}{Platform as a Service}
\newacronym{acr:MVC}{MVC}{Model-View-Controller}
\newacronym{acr:CI}{CI}{Continuous Integration}
\newacronym{acr:IDE}{IDE}{Integrated development environment}
\newacronym{acr:IP}{IP}{Internet Protocol}
\newacronym{acr:JSON}{JSON}{JavaScript Object Notation}
\newacronym{acr:SaaS}{SaaS}{Software as a Service}
\newacronym{acr:SPA}{SPA}{Single-Page Application}
\newacronym{acr:PWA}{PWA}{Progressive Web App}
\newacronym{acr:IaaS}{IaaS}{Infrastructure as a Service}
\newacronym{acr:TS}{TS}{TypeScript}
\newacronym{acr:KWE}{KWE}{Kontrollwaagengeneration}
\newacronym{acr:VFFS}{VFFS}{Vertikale Form"~, Füll"~ und Schließmaschinen}
\newacronym{acr:KI}{KI}{Künstliche Intelligenz}
\newacronym{acr:AI}{AI}{Artificial Intelligence}
\newacronym{acr:ML}{ML}{Machine Learning}
\newacronym{acr:ID}{ID}{Identifikator}
\newacronym{acr:DNS}{DNS}{Domain Name System}
\newacronym{acr:URL}{URL}{Uniform Resource Locator}
\newacronym{acr:PA}{PA}{Bosch Packaging Technology}
\newacronym{acr:HTML}{HTML}{Hypertext Markup Language}
\newacronym{acr:CSS}{CSS}{Cascading Style Sheets}

% Add Acronyme
\glsadd{acr:FaaS}
\glsadd{acr:CLI}
\glsadd{acr:SSO}
\glsadd{acr:CORS}
\glsadd{acr:NPM}
\glsadd{acr:LTS}
\glsadd{acr:REST}
\glsadd{acr:API}
\glsadd{acr:HTTP}
\glsadd{acr:CSV}
\glsadd{acr:JS}
\glsadd{acr:macOS}
\glsadd{acr:PaaS}
\glsadd{acr:MVC}
\glsadd{acr:CI}
\glsadd{acr:IDE}
\glsadd{acr:IP}
\glsadd{acr:JSON}
\glsadd{acr:SaaS}
\glsadd{acr:SPA}
\glsadd{acr:PWA}
\glsadd{acr:IaaS}
\glsadd{acr:TS}
\glsadd{acr:KWE}
\glsadd{acr:VFFS}
\glsadd{acr:KI}
\glsadd{acr:AI}
\glsadd{acr:ML}
\glsadd{acr:ID}
\glsadd{acr:DNS}
\glsadd{acr:URL}
\glsadd{acr:PA}
\glsadd{acr:HTML}
\glsadd{acr:CSS}

% Begin the document
\begin{document}

% Titlepage
\begin{titlepage}
    \begin{center}
        \includegraphics[scale=2.5]{images/he_logo.pdf}\\
        \vspace{1cm} Fakultät Informationstechnik\\
        \vspace{1.5cm} \Large Abschlussarbeit\\
        \vspace{1.5cm} \Huge Künstliche Intelligenz zur \\ Parameteroptimierung \\ in der Fertigung\\
        \vspace{1.5cm} \Large Lars Helmuth Probst\\\normalsize
        \vspace{0.5cm} Wintersemester 2018\\\normalsize
        \vfill{}
        \begin{tabular}{rl}
            Firma: & Robert Bosch GmbH\\[0.5cm]
            Betreuer: & Sebastian Stöcklmeier\\[0.5cm]
            Erstprüfer: & Prof. Dr. Harald Melcher\\[0.5cm]
            Zweitprüfer: & Prof. Dr. Reiner Marchthaler
        \end{tabular}
    \end{center}
\end{titlepage}

% More content
\thispagestyle{empty}
\vspace*{2cm}
\begin{center}
    % Dedication
    \begin{minipage}{12cm}
        \begin{center}
            % content
        \end{center}
    \end{minipage}

    \vfill{}

    % Quote
    \begin{minipage}{10cm}
        \begin{quote}
            \textit{\enquote{Scientists investigate that which already is;\\ engineers create that which has never been.}}
        \end{quote}
        \hfill Albert Einstein
    \end{minipage}
\end{center}

% Statement
\chapter*{Ehrenwörtliche Erklärung}
\thispagestyle{empty}
Hiermit versichere ich, die vorliegende Arbeit selbstständig und unter ausschließlicher Verwendung der angegebenen
Literatur und Hilfsmittel erstellt zu haben.

Die Arbeit wurde bisher in gleicher oder ähnlicher Form keiner anderen Prüfungsbehörde vorgelegt und auch nicht
veröffentlicht.

\begin{tabbing}
          Esslingen, den \today ~~	\= \rule{6cm}{0.3mm}\\ \> Unterschrift
\end{tabbing}

% Blocking
\chapter*{Sperrvermerk}
\thispagestyle{empty}
Die vorliegende Masterarbeit ist nur den jeweiligen Betreuern und Korrektoren zugänglich zu machen. Das alleinige
Nutzungsrecht der Masterarbeit liegt bei dem Unternehmen Robert Bosch GmbH, Gesellschaftsbereich Bosch Packaging
Technology.

Die Weitergabe der Arbeit im Gesamten oder in Teilen ist grundsätzlich untersagt. Ausnahmen bedürfen der schriftlichen
Genehmigung des Unternehmens.

Eine Aufnahme der Abschlussarbeit in die Bibliothek der Hochschule Esslingen ist somit nicht möglich. Die Sperre ist
nach zwei Jahren aufgehoben.

% Thanks
\chapter*{Danksagung}
\thispagestyle{empty}

An dieser Stelle möchte ich einzelne Personen erwähnen, die maßgeblich zum Erfolg und zur Ermöglichung dieser
Masterarbeit beigetragen haben.

Zu Beginn danke ich Herrn Prof. Dr. Harald Melcher von der Hochschule Esslingen für die zuvorkommende Betreuung im
Rahmen des Projektes.

Des Weiteren gilt mein besonderer Dank den Herren Sebastian Stöcklmeier und Dr. Patrick Risse von der Robert Bosch
Packaging Technology GmbH, welche dieses Projekt ermöglicht und durch ihre Anregungen und guten Ideen stets zum Ergebnis
beigetragen haben.

Mein herzlicher Dank gilt auch André Philipps sowie Björn Krause von der Robert Bosch Packaging Technology GmbH, die mir
eine kompetente und hilfsbereite Anlaufstelle beim Lösen von Problemen waren.

Außerdem möchte ich meiner Familie, Freunden und Kommilitonen für den Rückhalt während des gesamten Studiums danken.
Vielen Dank für das Gegenlesen der Arbeit und dem offenen Ohr bei Fragen und Problemen.

Abschließend gilt mein Dank der Firma Robert Bosch GmbH sowie der Abteilung Bosch Packaging Technology (PA) und der
Hochschule Esslingen für die Möglichkeit der Abschlussarbeit.

% "Frontmatter" beginnen (Formatierung umschalten)
% Platz für Inhaltsverzeichnis und anderes
\frontmatter

% Inhaltsverzeichnis ausgeben
\tableofcontents

% Die Überschrift für den Eintrag mit title= Überschreiben
% Ohne Angabe des type wird type "glossary" verwendet
% Ausser "altlist" existieren einige andere Stile wie z.B. "long"
%\printglossary[style=altlist, title=Stichwortverzeichnis]
%\newpage

% Das Abkürzungsverzeichnis hat den Typ \acronymtype
\printglossary[type=\acronymtype, style=long, title=Abkürzungsverzeichnis]
\newpage

% Das selbstdefinierte Verzeichnis ausgeben
%\printglossary[type=symbolslist, style=long, title=Symbolverzeichnis]
%\newpage

% Hauptteil beginnen (Formatierung umschalten)
\mainmatter

%------------------------------------------------------------------
%                      Kapitelstruktur einfügen
%------------------------------------------------------------------
\chapter{Einleitung}
\label{ch:einleitung}

\section{Motivation}
\label{sec:motivation}
- Warum habe ich das gemacht?
- Was hat Bosch davon?
- Was ändert sich am Markt?
\\ \\
- Machine Learning ein immer Wichtigerer Aspekt im Leben
- Andere Firmen haben viel damit gemacht
- Maschinen schneller und besser einstellen können
- Maschinen Reibungsloser einstellen
- Wiederkehrende Probleme sehen
- Maschinen automatisch einstellen
\\ \\
- Hypothese, dass ich das Problem mit Machine Learning lösen will
\\ \\
2 Seiten

\section{Aufgabenstellung}
\label{sec:aufgabenstellung}
- Was soll in der Arbeit umgesetzt werden?
- Grob beschreiben was alles durchgeführt wird?
- Grob beschreiben, was rauskommen soll.
- Was ist das Ziel des ganzen
- Was kommt nachher raus
\\ \\
- Ich habe in der Cloud ein Machine Learning Netzwerk aufgebaut
- Machine Learning Netzwerk mit Tensorflow
- Webseite (Client) zum eingeben und auslesen der Daten
- REST-Schnittstelle zur Kommunikation
- Smartphone-Apps gebaut
- Bild mit kompletter Architektur
\\ \\
1 Seite

\newpage

\section{Aufbau der Arbeit}
\label{sec:aufbauDerArbeit}
Dieses Kapitel soll zur schnelleren Orientierung innerhalb der Arbeit dienen und zeigt, welche Themen in den jeweiligen
Kapiteln angesprochen werden.

\begin{description}

    \item[Kapitel 2 (Grundlagen)]\hfill \\
    Hier kommt der Inhalt

    \item[Kapitel 3 (Neuronales Netz)]\hfill \\
    Hier kommt der Inhalt

    \item[Kapitel 4 (Client)]\hfill \\
    Hier kommt der Inhalt

    \item[Kapitel 5 (Adaptierbarkeit)]\hfill \\
    Hier kommt der Inhalt

    \item[Kapitel 6 (Ausblick)]\hfill \\
    Hier kommt der Inhalt

    \item[Kapitel 7 (Zusammenfassung)]\hfill \\
    Hier kommt der Inhalt

\end{description}
\chapter{Grundlagen}
\label{ch:grundlagen}
Das folgende Kapitel beschreibt elementare Grundlagen, die zum Verständnis der nachfolgenden Kapitel notwendig sind. Der
erste Abschnitt befasst sich mit den verwendeten Maschinen und Komponenten der Robert Bosch Packaging GmbH.

Der zweite Teil widmet sich der Künstlichen Intelligenz und den verschiedenen Untergruppen und Ausprägungen, welche in
diesem Bereich existieren. Es wird dabei unter anderem auf die Konfiguration der Produkte und Laufzeitumgebungen
eingegangen. Außerdem werden hier die Programme und Konzepzte behandelt, welche in dieser Arbeit benötigt werden.

Im letzten Teil des Kapitels folgt die Beschreibung einer Designidee bei Smartphone-Apps, allgemeine Konzepte der
Softwareentwicklung und ein Framework zum Entwickeln von Webseiten.

\section{Bosch KWE Waage}
Was ist die Bosch KWE Waage. Parameter der Waage definieren.
\colorbox{yellow}{Hier fehlt was}

\section{Künstliche Intelligenz}
Was ist künstliche Intelligenz?
\colorbox{yellow}{Hier fehlt was}

\section{Machine Learning}
Was ist Machine Learning?
\colorbox{yellow}{Hier fehlt was}

\subsection{Algorithmen}
Welche Algorithmen gibt es dafür
\colorbox{yellow}{Hier fehlt was}

\subsection{Bewertungskriterien}
Welche Bewertungskreterien gibt es
\colorbox{yellow}{Hier fehlt was}

\subsection{Klassifikation}
Welche Klassifikationen gibt es
\colorbox{yellow}{Hier fehlt was}

\subsection{Deep Learning}
Was ist Deep Learning
\colorbox{yellow}{Hier fehlt was}

\subsection{Neuronale Netze}
Was ist ein Neuronales Netz
\colorbox{yellow}{Hier fehlt was}

\section{Cloud}
Der Begriff Cloud hat sich als Kurzform des Cloud Computing etabliert und versteht das Zusammenspiel von mehreren Servern.
Die Server übernehmen Aufgaben, wie etwa die Datenspeicherung oder komplizierte Programmabläufe. Dabei erkennt der
Cloud-Nutzer nicht, wie viele Server hinter der Cloud stecken oder wo diese sich befinden.

Selbst wenn ein Server ausfällt, hat dies keine Auswirkungen auf das gesamte System, da die Anfragen und Aufgaben auf
die anderen Systeme umgeleitet werden.

Die Cloud zeichnet sich nach NIST~\cite{online_grundlagen_cloud_nist} und~\cite{online_grundlagen_cloud_computing} durch
fünf wesentliche Eigenschaften aus:

\begin{itemize}
    \item \textbf{On-Demand Self Service} \\
    Registrierte Nutzer können Resourcen selbstständig instantiieren und konfigurieren.
    \item \textbf{Broad Network Access} \\
    Der Zugriff kann von verschiedenen Endgeräten erfolgen.
    \item \textbf{Resource Pooling} \\
    Alle Resourcen des Anbieters werden gebündelt und nach Bedarf den Nutzern zugewiesen.
    \item \textbf{Rapid Elasticity} \\
    Kapazitäten können nach Bedarf skaliert werden und stehen schnell und dynamisch zur Verfügung.
    \item \textbf{Measured Service} \\
    Es existiert eine automatische Kontrolle der Ressourcen durch einen Zähler, welcher die Transparenz für den
    Anbieter und den Benutzer ermöglicht.
\end{itemize}

Auf dem Markt existieren etliche Cloud-Anbieter. Im folgenden wird auf drei der Top 10 größten Anbieter und deren Lösungen
im Bereich künstliche Intelligenz eingegangen (Siehe dazu~\cite{online_grundlagen_cloud}).

\subsection{Microsoft Azure}
Was ist Azure
\colorbox{yellow}{Hier fehlt was}

\subsubsection{Azure Machine Learning Studio}
Beschreiben des Tools von Azure für Machine Learning
\colorbox{yellow}{Hier fehlt was}

\subsection{Amazon Web Services}
Was ist AWS
\colorbox{yellow}{Hier fehlt was}

\subsubsection{Amazon Machine Learning}
Beschreiben des Tools von AWS für Machine Learning
\colorbox{yellow}{Hier fehlt was}

\subsection{IBM Cloud (ehemals Bluemix)}
Bluemix ist die von IBM entwickelte Cloud-Platform. Über Bluemix greifen Entwickler auf mehr als 160 Cloud-Services zu,
um mobile Apps und Webanwendungen zu entwickeln. In Bluemix gibt es zahlreiche Analysewerkzeuge sowie Services von
Drittanbietern. Mit Watson Analytics lassen sich beispielsweise intelligente Systeme realisieren, die Daten kognitiv
(also selbstlernend, ohne für die Problemlösungen programmiert zu sein) auswerten und für die Entscheidungsfindung
aufbereiten.

Bluemix unterstützt diverse integrierte DevOps-Dienste, um Cloud-Anwendungen zu erstellen, auszuführen, bereitzustellen
und zu verwalten. Die Entwicklerplattform basiert auf der Technologie von Cloud Foundry und läuft auf IBMs
Softlayer Cloud Infrastruktur. Sie unterstützt mehrere Programmiersprachen, einschließlich Java, Node.js, Go, PHP,
Python, Ruby Sinatra, Ruby on Rails und kann auch andere Sprachen wie Scala durch den Einsatz von Buildpacks
unterstützen~\cite{online_grundlagen_bluemix}.

Weitere Informationen üer Bluemix können auf der Webseite\footnote{https://bluemix.net} gefunden werden.

\subsubsection{Watson Studio}
Was ist das und was kann man damit machen
\colorbox{yellow}{Hier fehlt was}

\subsubsection{Cloud Object Storage}
Was ist das
\colorbox{yellow}{Hier fehlt was}

\subsubsection{Apache Spark}
Was ist das
\colorbox{yellow}{Hier fehlt was}

\subsubsection{API Connect}
Was ist das
\colorbox{yellow}{Hier fehlt was}

\subsubsection{Toolchain}
Bei einer Toolchain handelt es sich um ein Tool zur Verwaltung von Entwicklung, Bereitstellung sowie Üerwachung einer
Anwendung. Nach der Einrichtung einer Toolchain stehen verschiedene Services zur Verfügung, welche zum Beispiel eine
Integration zu einem GitHub-Projekt ermöglichen.

Mit einer eingerichteten Toolchain und einem verbundenen Git-Repository kann der Quelltext gebaut und auf einem System
installiert werden. Bei der Toolchain handelt es sich unter anderem um ein \textit{Continuous Integration}-Tool (kurz CI)
(Mehr unter~\cite{online_grundlagen_toolchain}).

\section{TensorFlow.js}
Was ist TensorFlow und wie nutzt man das. Es geht hier dann um TensorFlow.js
\colorbox{yellow}{Hier fehlt was}

\section{Angular}
Was ist Angular und was macht man damit
\colorbox{yellow}{Hier fehlt was}

\subsection{Model-View-Controller}
Model-View-Controller (kurz MVC) wurde um 1978 von Xerox entwickelt und es handelt sich dabei um eine Architektur von
Programmen.

In MVC wird eine Anwendungskomponente in drei Teile zerlegt. In das oberflächenunabhängige \textit{Model}, das für die
Ausgabe zuständige \textit{View} und den für die Interpretation von Eingabeereignissen zuständigen \textit{Controller}.

Der Controller ist die Steuerungseinheit, das Model der Anwendungskern und die View der Darsteller bzw. die Ansicht der
Anwendung. Ein View zusammen mit seinem Controller wird hier auch als eine Oberfläche bezeichnet (Mehr hierzu
unter~\cite{book_grundlagen_mvc}).

Es gibt zahlreiche Frameworks, welche das MVC-Patern umsetzen. AngularJS ist einer der bekanntesten Vertreter.

\subsection{Angular-Material}
Was kann man damit machen
\colorbox{yellow}{Hier fehlt was}

\subsection{ngx-restangular}
Was bringt die Library
\colorbox{yellow}{Hier fehlt was}

\subsection{Observable}
Was ist das für ein pattern?
\colorbox{yellow}{Hier fehlt was}

\subsection{Promise}
Was ist das für ein pattern
\colorbox{yellow}{Hier fehlt was}

\subsection{flex-layout}
Was kann man damit erreichen
\colorbox{yellow}{Hier fehlt was}

\section{REST}
REST-API steht für Representational State Transfer - Application Programming Interface. Diese Interfaces machen den
Austausch von Informationen uüber unterschiedliche Systeme möglich. Im Zeitalter verteilter Systeme und Cloud-Computing
trifft man oft auf unterschiedliche Systeme, welche den Einsatz von REST-API notwendig machen.

Man spricht bei REST-API auch von der Maschine-Maschine-Kommunikation, da die verschiedenen Systeme und Geräte
zusammengebracht werden und gewissermaßen die gleiche Sprache sprechen.

Mit REST-API ist es möglich geworden, Informationen und Aufgaben auf verschiedene Systeme zu verteilen und diese mit der
Hilfe von HTTP-Requests anzufordern. Jeder HTTP-Request setzt sich aus dem Endpoint und den entsprechenden Parametern
zusammen~\cite{online_grundlagen_rest}.

\section{Cloud Foundry}
Cloud Foundry ist eine Open Source Platform as a Service (PaaS) Lösung. Mit Hilfe von Cloud Foundry können Entwickler
ihre Anwendungen bauen, hochladen und ausführen. Ein großer Vorteil von Cloud Foundry ist die nahezu grenzenlose
Skalierbarkeit~\cite{online_grundlagen_cf}.

Die Skalierbarkeit wird durch die Container-Architektur erreicht. Jeder Container beinhaltet eine eigene Anwendungsinstanz.
Dabei wird sowohl eine horizontale als auch eine vertikale Skalierung unterstützt.

Bei der horizontalen Skalierung werden zusätzliche Container gestartet oder gestoppt und ein Load Balancer vor die
Container geschaltet. Bei der vertikalen Skalierung werden jeder Instanz individuell z.B. mehr oder weniger Arbeitsspeicher
zugeteilt.

Jede Cloud Foundry Instanz besitzt eine IPV4-Adresse wodurch es mittels A-Record (Zuordnung eines DNS-Namens zu IP-Adresse)
möglich ist, eine eigene Domain mit der Anwendung zu verknüpfen.

\section{Version Control (Git)}
Was ist das und für was ist das gut
\colorbox{yellow}{Hier fehlt was}

\section{Node Package Manager (npm)}
Was ist das und für was ist das gut
\colorbox{yellow}{Hier fehlt was}

\section{Mockups}
Was bringt sowas
\colorbox{yellow}{Hier fehlt was}

\section{TypeScript}
Was hat man da für Vorteile und was ist das
\colorbox{yellow}{Hier fehlt was}

\section{WebView}
Bei WebView handelt es sich um ein Android- und iOS-Layout, das Webseiten darstellen kann. Über Schnitstellen werden
Informationen wie Webseite, URL und Größe üergeben. Das Laden, das Anzeigen und die Interaktion mit der Webseite üernimmt
das Layout selbstständig.

Das Layout steht sowohl unter Android als auch iOS in den frühesten Versionen zur Verfügung (Weitere Informationen
unter~\cite{online_grundlagen_webview}).
\chapter{Neuronales Netz}
\label{ch:neuronalesNetz}

Hier beschreiben, was im folgenden Kapitel gemacht werden soll? Was ist das Ziel? Was muss vorbereitet werden?
\\ \\
Ich habe Testdaten gesammelt und strukturiert. Ich habe ein neuronales Netz in der Cloud (Bluemix) aufgebaut und es
mit den gesammelten Daten trainiert und getestet. - Dann habe ich das Model deployed, damit man es per REST aufrufen kann.
- Dann API Connect genommen um die REST-Schnittstelle anzusprechen und eigene zu bauen. Zur kommunikation.
- Dann damit rumprobiert. - Dann das trainierte Model genommen und mit Tensorflow.JS genutzt und eine NodeJS-Applikation gebaut
- Da dann auch ausprobiert und geschaut, ob das gleiche Erbegnis rauskommt.
- Auch eine Schnittstelle gebaut, damit man die Funktion von außen aufrufen kann. - Am Schluss die Daten verglichen. Was
kommt bei der Cloud raus und was bei der Applikation - Bild mit schematischer Architektur (Client, API Connect, Deployment, Model)
\\ \\
1--2 Seiten
\section{Analyse}
Wie könnte man das ganze umsetzen? Es gibt eigentlich zwei Arten.

\subsection{Cloud}
Wir könnten es in der Cloud machen. Besprechen von Bluemix, Azure, AWS und Python. Vor und auch Nachteile der Variante.

\subsection{Tensorflow}
Wie könnten es aber auch eigenständig mit Tensorflow machen. Vor und auch Nachteile der Variante.

Nun, wir machen beides
\section{Vorbereitung}
In diesem Kapitel werden Vorkehrungen für die Entwicklung eines neuronalen Netzes und einer TensorFlow-Applikation
getroffen. Dafür muss zunächst ein kostenloses IBM Cloud Konto erstellt und drei Services eingerichtet werden.
Für die Entwicklung werden zwei Programme auf dem Entwicklungsrechner installiert.

Im Weiteren werden die dafür notwendigen Schritte einzelnd erläutert und getestet.

\subsection{IBM Cloud Konto}
Damit mit der IBM Cloud gearbeitet werden kann, wird ein kostenlose Konto benötigt. Dieses kann auf der zugehörigen
Registrierungsseite\footnote{https://console.bluemix.net/registration} erstellt werden.

Nachdem das Konto erfolgreich mit einem an die hinterlegte E-Mail-Adresse verschickten Bestätigungslink aktiviert ist,
kann das Konto 30 Tage lang ohne anfallende Gebühren für Services oder Runtimes genutzt werden.

Nach dem erstmaligen Aufruf des IBM Cloud Dashboards, wird nach einem Namen für die automatisch erstellte Organisation
gefragt. Dieser spielt für die Umsetzung keine Rolle und die Organisation kann zu jedem Zeitpunkt umbenannt oder auch
gelöscht werden. Ein Beispiel für den Organisationsnamen ist \textit{Machine-Learning}.

Im Anschluss wird nach einem Namen für den ersten Space in der erstellten Organisation gefragt. Auch dieser
spielt für die Umsetzung keine Rollte und kann zu jedem Zeitpunkt umbenannt oder auch gelöscht werden. Ein Beispiel für
die Benamung des ersten Spaces ist \textit{dev}, was eine Abkürzung für \textit{developper} ist.

Ein \textit{Space} gruppiert mehrere Runtimes und Services in einem Rechenzentrum. Von diesen Rechenzentren gibt es in
der IBM Cloud aktuell sechs, verteilt in allen Kontinenten. Eine \textit{Organisation} kann mehrere Spaces beinhalten.

\subsection{IBM Cloud CLI}
Für die einfache Verwaltung der IBM Cloud empfiehlt sich, dass zugehörige Command Line Interface (kurz \textit{CLI}) zu
installieren. Die Installation unter Linux und macOS erfolgt durch die Eingabe des folgenden Kommandos in eine Shell:

\begin{lstlisting}[language=bash, caption=Installation des IBM Cloud CLI, label=Installation des IBM Cloud CLI]
    $ curl -sL http://ibm.biz/idt-installer | bash
\end{lstlisting}

Anschließend wird die erfolgreiche Installation über das folgende Kommando überprüft:

\begin{lstlisting}[language=bash, caption=Installation des CLI überprüfen, label=Installation des CLI überprüfen]
    $ ibmcloud dev help
\end{lstlisting}

Die Ausgabe zeigt eine Übersicht über alle möglichen Befehle des \textit{ibmcloud}-Tools. Eine angezeigte Fehlermeldung
gibt Auskunft über eine fehlgeschlagene Installation. Weitere Informationen sind auf der betreffenden
Installationsseite\footnote{https://console.bluemix.net/docs/cli/reference/bluemix\_cli/get\_started.html} zu finden.

Mit der Installation des IBM Cloud-CLI wird ein symbolischer Link über das Kommando \texttt{bx} angelegt. Somit kann der
Befehl \textit{ibmcloud} auch immer durch \textit{bx} ersetzt werden.

\subsection{Watson Studio}
Für den Aufbau und die Konfiguration des neuronalen Netzes wird der Service \textit{Watson Studio} benötigt. Aktuell gibt
es zwei Möglichkeiten einen Service oder eine Runtime in einen erstellten Space einzubinden.

\subsubsection*{Über das IBM Cloud Dashboard}
Auf dem IBM Cloud Dashbaord können mit einem Klick auf \textit{Katalog} alle Services und Runtimes, welche aktuell genutzt
werden können, aufgelistet werden. Der Service \textit{Watson Studio} ist in der Kategorie \textit{Künstliche Intelligenz}
zu finden. Ein klick auf diesen öffnet die Konfigurationsseite.

Auf der Konfigurationsseite muss der \textit{Servicename}, welcher frei gewählt werden kann, eingetragen werden.
Anschließend kann noch die Region, in welcher der Service zur Verfügung stehen soll und die Ressourengruppe definiert
werden.

Mit einem Klick auf \texttt{Erstellen}, wird der Service Instanziiert und es erfolgt eine Weiterleitung zurück auf das
Dashbaord. Dort erscheint der Service mit dem eingetragenen Namen in der Liste der \textit{Services}.

Über das Dashboard wird die Übersichtsseite des Watson Studio Service durch einen Klick auf diesen geöffnet. Auf der
folgenden Seite finden sich Hinweise auf die Funktionen des Service. Über die Schaltfläche \texttt{Get Started} wird in
das Watson Studio Dashboard gewechselt.

\subsubsection*{Über die CLI}
Alternativ erfolgt die Installation des Services mittels dem installierten CLI. Dazu wird im ersten Schritt das CLI mit
dem IBM Cloud-Konto verknüpft. Der folgende Befehl setzt der dafür notwendigen API-Endpunkt:

\begin{lstlisting}[language=bash, caption=Setzen des API Targets, label=Setzen des API Targets]
    $ ibmcloud api https://api.ng.bluemix.net
\end{lstlisting}

Anschließend erfolgt ein Login über den folgenden Befehl in der Shell:

\begin{lstlisting}[language=bash, caption=Login über CLI und Single Sign-on, label=Login über CLI und SSO]
    $ ibmcloud login --sso
\end{lstlisting}

Der Parameter \texttt{sso} bewirkt, dass der Login über ein sich öffnendes Browserfenster erfolgt. Dies ermöglicht einen
bequemen Login ohne Kommandozeile.

Nach einem erfolgreichen Login, wird die genutzte Organisation ausgewählt. Dies geschieht über die Eingabe der
entsprechenden Nummer der Organisation, welche am linken Rand zu sehen ist. Der Login-Vogang ist damit abgeschlossen.

Im Weiteren wird eine Instanz des \texttt{Watson Studio}-Service mit dem folgenden Befehl erstellt:

\begin{lstlisting}[language=bash, caption=Instanziierung des Watson Studio Services, label=Instanziierung des Watson Studio Services]
    $ ibmcloud cf create-service Watson-Studio lite sercive_name
\end{lstlisting}

Für den Parameter \texttt{service\_name} wird ein Name eingetragen, unter welchem der Service aufgerufen werden kann.
Über den Befehl \texttt{cf services} können alle instantiierten Services, welche sich in der vorher ausgewählten
Organisation befinden, aufgelistet werden.

\begin{lstlisting}[language=bash, caption=Auflisten aller Services in einer Organisation, label=Auflisten aller Services in einer Organisation]
    $ ibmcloud cf services
\end{lstlisting}

Der erstellte Service sollte mit dem definierten Namen in der Liste der Services erscheinen.

Abschließend wird ein Projekt im Watson Studio angelegt. Dieses bündelt alle Daten und Informationen an einem gemeinsamen
Ort. Um ein Projekt zu erstellen, wird das Watson Studio Dashboard über das IBM Cloud Dashboard aufgerufen.

Ein solches Projekt wird im Watson Studio Dashboard über den Menüpunkt \texttt{New project} angelegt. Nach der Eingabe
eines Namens und der Einrichtung eines Local Storages (dies geschicht über den Mini-Wizzard auf der rechten Seite) kann
die Eingabe über die Schaltfläche \texttt{Create} bestätigt werden.

\subsection{Node.js Runtime}
\label{ssc:nodejs_runtime}
Für die Erstellung der Node.js-Applikation wird eine entsprechende Runtime in der IBM Cloud benötigt. Dafür wird ein
vorkonfigurierter Cloud Foundry-Container mit Node.js-Konfiguration erstellt. Für die Erstellung gibt es ebenfalls zwei
Möglichkeiten.

Im vorangegangenen Kapitel wurden die beiden Möglichkeiten (über den IBM Cloud Katalog und über die Kommandozeile)
ausführlich erläutert. Im Folgenden wird lediglich die Erstellung über das CLI aufgezeigt:

\begin{lstlisting}[language=bash, caption=Instanziierung der Node.js Runtime, label=Instanziierung der Node.JS Runtime]
    $ ibmcloud cf create-service nodejs service_name
\end{lstlisting}

Auch hier wird über den Parameter \texttt{service\_name} der Name für die Applikation vergeben. Die URL, über die
der Container später aufrufbar ist, folgt dem Schema \texttt{https://service\_name.mybluemix.net}.

Dies bedeutet, dass der Name nur ein Mal im kompletten System vergeben sein darf. Eine Meldung gibt Hinweise darauf, ob
die Applikation erfolgreich erstellt werden konnte, oder ob ein anderer Name für die Applikation verwendet werden muss.

\subsection{API Connect}
\label{subsec:apiconnect}
Um im weiteren Verlauf den API Gateway, \textit{API Connect}, nutzen zu können, muss dieser ebenfalls im IBM Cloud
instanziiert werden. Wie auch in den beiden vorangegangenen Kapiteln kann dies über die CLI als auch über den IBM Cloud
Katalog geschehen.

Einfachheitshalber wird im Folgenden die Instanziierung des Services mittels IBM Cloud CLI aufgezeigt.

\begin{lstlisting}[language=bash, caption=Instanziierung von API Connect, label=Instanziierung von API Connect]
$ ibmcloud cf create-service API-Connect lite service_name
\end{lstlisting}

Über den Parameter \texttt{service\_name} wird der Name für den Service definiert. Mit diesem ist er im IBM Cloud
Dashboard auffindbar.

\subsection{Git}
Für die Verwaltung und den späteren, automatisierten Installationsvorgang des geschriebenen Quellcodes wird Git verwendet.
Unter Linux erfolgt die Installation des Programmes über das folgende Kommando:

\begin{lstlisting}[language=bash, caption=Installation von Git, label=Installation von Git]
    $ sudo apt-get install git
\end{lstlisting}

Unter macOS steht ein Grafische Installationsprogramm\footnote{http://sourceforge.net/projects/git-osx-installer} zur
Verfügung.

\subsection{Node.js und npm}
Für die Entwicklung und die damit verbundenen Tests und Probeläufe auf dem Entwicklungsrechner wird ein installiertes
Node.js benötigt. Eine Installation der zur Zeit aktuellsten LTS-Version (8.12.0) erfolgt unter Linux über das folgende
Kommando:

\begin{lstlisting}[language=bash, caption=Installation von Node.js, label=Installation von Node.js]
    $ curl -sL https://deb.nodesource.com/setup_8.x | sudo -E bash -
    $ sudo apt-get install -y nodejs
\end{lstlisting}

Für macOS existiert es ein Installationspaket, welches auf der Download-Seite\footnote{https://nodejs.org/dist/v8.12.0/node-v8.12.0.pkg}
heruntergeladen werden kann.

Bei der Installation von Node.js wird npm automatisch mitinstalliert. Dieser ist der Node.js eigene Paketmanager und wird
für die spätere Installation von Abhängigkeiten für die Applikation benötigt.
\section{Umsetzung}
Hier noch generell was beschreiben. - Zielarchitekturbild darstellen (komplette Architektur)
Was wird im folgenden gemacht.

\colorbox{yellow}{Hier fehlt was}

\subsection{Cloud}
Hier beschreiben, was ich nun in der Cloud umsetzen will und warum man die Cloud eigentlich braucht. Vorteil der
Geschwindigkeit. Dabei würde es insgesamt vier Möglichkeiten geben das umzusetzen. Für welche habe ich mich
entschieden?

- Machine Learning Models (Autoamtisch aus Daten generrieren)\\
- Model flows (SPSS) - Make Deployment\\
- Model flows (SPSS) - Download Model - Import TensorFlow/TensorFlow.js\\
- Notebooks (Python)

\url{https://console.bluemix.net/docs/cli/index.html#overview}

\colorbox{yellow}{Hier fehlt was}

\subsubsection{Daten zusammenstellen}
\colorbox{yellow}{Hier fehlt was}

\subsubsection{Daten importieren}
Nachdem man die Trainingsdaten zusammengestellt hat, kann man diese nun in das Watson Studio importieren. Dazu existiert
in Watson Studio Dashboard einen Menüpunkt mit dem Namen \textit{Assets}.

Dieser beinhaltet alle hochgeladenen, generierten oder gesamelten Daten, Modelle, Dashboards, Notebooks oder Flows. Die
oberste Kategorie, \textit{Data Assets}, listet alle Trainingsdaten in Form von Excel-Tabellen für die Weiterverarbeitung
auf.

Über den Menüpunkt \texttt{New data asset} wird eine neue Datei hochladen. Ein klick auf diesen Menüpunkt öffnet einen
seitlichen Arbeitsbereich. Der Nutzer kann über den Menüpunkt \texttt{Browse} die Dateiauswahl des Betriebssystems
öffnen und seine Datei auswählen. Bei der Auswahl handelt es sich um die im vorangegangenen Kapitel erstellte Datei mit
Trainingsdaten.

Alternativ kann man die Datei auch in das fabrlich hervorgehobene Feld schieben. Der Upload startet damit sofort.

Nach wenigen Sekunden ist die Datei hochgeladen und wird im Bereich \textit{Data assets} angezeigt. Damit ist der
Uploadvorgang erfolgreich abgeschlossen und die Datei kann im nächsten Schritt umgewandelt werden.

\subsubsection{Daten umwandeln}
Da die erstellte Datei mit den Trainigsdaten nun in Watson Studio bereitsteht, ist die Umwandlung in eine CSV-Datei
möglich. Dieser Schritt ist zwingend notwendig, damit die Datei später als Eingabeparameter für das neuronale Netz dienen
kann. Ein anderes Format wird als Eingabe zur Zeit nicht unterstützt.

In Watson Studio liegt mit den \textit{Data flows} ein einfaches Werkzeug bereit, um Dateien in das benötigte CSV-Format
zu überführen. Dazu wird in der Kategorie Data flows über \texttt{New data flow} ein neuer Flow angelegt.

Im folgenden öffnet sich der Wizzard zum Erstellen des Flows. Der Nutzer muss im im linken Bereich die umzuwandelnde Datei
auswählen. Über die Schaltfläche \texttt{Add} wird die Auswahl dann übernommen.

Nun ist der Inhalt der Datei sichtbar. Bei der Ansicht ist darauf zu achten, dass die Werte der einzelnen Spalten als
Dezimal-Zahl interpretiert werden. Sollte diese Einstellung nicht voreingestellt sein, muss dieses Manuell abgeändert
werden.

Dazu klickt man in der entsprechenden Spalte oben auf die drei Punkte (Esintellungen) und selektiert den Eintrag
\texttt{Convert Column}. Im Untermenü muss dann der Wert \texttt{Decimal} bestätigt werden.

Da es sich nun um Dezimal-Zahlen handelt, kann man über die Schaltfläche \texttt{Run data flow} die Konvertierung starten.
Bevor die Konvertierung jedoch startet, zeigt das System eine Übersicht über die einzelnen Schritte, die dazu benötigt
werden, an.

Hier werden zum Beispiel etwaige Konvertierungen zu Dezimal-Zahlen angezeigt oder sonstige Operationen dargestellt. Der
Nutzer kann die Seite bestätigen und der Flow startet.

Nach wenigen Minuten sollte im Watson Studio im Bereich Assets, in der Kategorie Data assets, die neue Datei zur Verfügung
stehen. Dabei trägt die Datei die Dateiendung csv. Die Konvertiedung der Datei ist somit abgeschlossen und sie kann für
das neuronale Netz verwendet werden.

\subsubsection{Modeler flow}
\label{subsub:modeler_flow}
Nach erfolgreicher Konvertierung der Trainingsdaten, können diese für das neuronale Netz genutzt werden. Für die
Erstellung des neuronalen Netzes und den damit verbundenen Parameterübergaben wird der \texttt{Modeler flow} genutzt.

Über die Schaltfläche \texttt{New flow} im Bereich Assets erstellt man diesen und richtet ihn ein. Nachdem man einen Namen
und eine optionale Beschreibung für den Flow eingegeben hat, muss man die Auswahl \textit{Modeler flow} und
\textit{IBM SPSS Modeler} bestätigen. Dabei handelt es sich um den Standard-Flow von Watson Studio.

Der Nutzer bestätigt die Eingabe über die Schaltfläche \texttt{Create}. Nach kurzer Zeit ist der Flow erstellt und es
erscheint ein leerer Arbeitsbereich.

Der Menüpunkt \texttt{Palette} zeigt einen linken Arbeitsbereich an, welcher alle Module, die genutzt werden können,
gruppiert auflistet. In diesem muss man den Baustein \textit{Data asset} in der Gruppe \textit{Import} auswählen. Dieser
ermöglicht den Import der konvertierten Trainingsdaten für das neuronale Netz. Über Drag\&Drop kann der Nutzer den
Baustein in den noch leeren Arbeitsbereich platzieren.

Um den Baustein zu konfigurieren klickt der Entwickler doppelt auf den Baustein. Dies öffnet einen seitlichen
Arbeitsbereich. In diesem kann er über die Schaltfläche \texttt{Change Data Asset} die Datei auswählen, welche die
Trainingssätzen enthält. Über \texttt{Save} wird die Einstellung gespeichert und der Baustein ist fertig konfiguriert.

Im nächsten Schritt muss das neuronale Netz konfiguriert werden. Dazu existiert in der Kategorie \textit{Modeling} das
Modul \textit{Neural Net}. Dies stellt für die Anwendung die beste Alternative dar. Genau wie das Import-Modul kann dieses
mit der Maus auf ein freies Feld des Arbeitsbereiches platziert werden. Somit ist das Modul teil des Prozesses.

Damit das neuronale Netz die importierten Daten nutzen kann, wird eine Verbindung zwischen dem Import-Modul und dem
Neural-Net-Modul aufgebaut. Über einen klick auf den Ausgang des Import-Moduls kann eine Verbindungslinie gestartet werden.

Mit einem klick auf den Eingang des Neural-Net-Modul kann man die Verbindung aufbauen. Die Verbindung wird nun über eine
durchgezogene Linie zwischen den beiden Modulen visualisiert.

Bei einer Verbindung werden die Ausgaben des jeweiligen Modules weiter an die mit einer Linie verbundenen Module
gegeben. Diese Folgemodule können die Werte dann als Eingabevariablen nutzen.

Damit das Neural-Net-Modul konfiguriert werden kann, muss der Nutzer doppelt auf dieses klicken. Damit man die
\textit{Targets} und die \textit{Inputs} selbst definieren kann, muss der Hacken bei \enquote{Use custom field roles}
gesetzt werden.

Die Tabelle~\ref{tab:targets_inputs} auf Seite~\pageref{tab:targets_inputs} zeigt die auszuwählenden Tabellenspalten für
die jewielige Kategorie. Dabei beschreiben die Targets die Variablen, welche durch den Watson Service Vorhergesagt werden
sollen. Die Inputs definieren die Größen, durch welche eine Vorhersage überhaupt möglich ist.

Dem resultierenden, trainierten Model werden zu einem späteren Zeitpunkt die Inputs übergeben und die Targets kommen als
vorhergesagte Rückgabeparameter zurück.

\begin{table}[hb]
    \centering
    \begin{tabular}{|c|c|}
        \hline
        \textbf{Targets} & \textbf{Inputs}\\
        \hline
        \hline
        Leistung & Einlaufbandlänge\\
        \hline
        Druckluft & Wägebandlänge\\
        \hline
        Impuls & Auslaufbandlänge\\
        \hline
        Totzeit & Einlaufbandbreite\\
        \hline
        Position & Wägebandbreite\\
        \hline
        & Auslaufbandbreite\\
        \hline
        & Einlaufbandrolle\\
        \hline
        & Wägebandrolle\\
        \hline
        & Auslaufbandrolle\\
        \hline
        & Produktbreite\\
        \hline
        & Produktlänge\\
        \hline
        & Produkthöhe\\
        \hline
        & Packungsgewicht\\
        \hline
    \end{tabular}
    \caption{Variablen für die Targets und Inputs}
    \label{tab:targets_inputs}
\end{table}

Mit dieser Konfiguration werden 13 Parameter als Eingabevariablen genutzt und fünf Parameter durch das neuronale Netz
vorhergesagt. Vier der Vorhergesagten Parameter (Druckluft, Impuls, Totzeit, Position) beziehen sich auf den
\textit{Pusher} und der Parameter \textit{Leistung} gibt die tatsächliche Bandgeschwindigkeit an, mit der das Band in der
Maschine laufen soll.

Alle anderen Einstellungen des neuronalen Netzes werden im ersten Schritt auf den Standardeinstellungen belassen. Zu einem
späteren Zeitpunkt, wenn sich Testdaten ändern oder Parameter angepasst werden müssen, kann man mit den weiteren
Einstellungen das neuronale Netz anpassen.

Der Flow für das neuronale Netz ist somit fertiggestellt. Mit einem klick auf \texttt{Run} startet das Training des
neuronalen Netzes.

Nach dem erfolgreichem training des neuronalen Netzes erscheint das trainierte Model unterhalb des neuronalen Netzes im
aktuellen Arbeitsbereich. Eine gestrichelte Linie zwischen dem neuronalen Netz und dem trainierten Model zeigt die
Abhängigkeit der beiden Module an.

Für die Weiterverarbeitung und ein späteres Deployment des trainierten Models muss man dieses mit einem Export-Modul
verbinden. Der Export erfolgt über das Modul \textit{Table}.

Das Table-Modul befindet sich in der Kategorie \textit{Outputs} und wird, genau wie alle andere Module, frei auf dem
Arbeitsbereich platziert. Eine Verbindung zwischen dem trainierten Model und der Table ermöglicht den Datenaustausch.
Für die Table ist keine weitere Konfiguration notwendig.

Ein rechtsklick auf das Table-Modul öffnet das Kontextmenü des Moduls. Darüber lässt sich der Punkt
\enquote{Save branch as a model} anklicken. Dieser ermöglicht es, das trainierte Model zu exportieren.

In dem sich öffnenden Fenster muss man den Namen und eine optinale Beschreibung für das neue Model definieren. Mit einem
klick auf den Button \texttt{Save} wird das Model gespeichert und es erscheint im Watson Studio Dashboard.

In der Abbildung~\ref{fig:umsetzung_model_flow} auf Seite~\pageref{fig:umsetzung_model_flow} ist der vollständige Aufbau
des Model flows visualisiert.

\begin{figure}[h]
    \centering
    \includegraphics[scale=0.26]{images/kapitel_3/umsetzung_model_flow.png}
    \caption{Vollständiger Model flow}
    \label{fig:umsetzung_model_flow}
\end{figure}

\subsubsection{Informationen zum Model}
Über einen weiteren rechtsklick auf das trainierte Model kann man über den Menüpunkt \texttt{View Model} wird eine
detaillierte Übersicht über das Model gelangen. Hier finden sich zahlreiche Informationen, die unter anderem aufschluss
über die Genauigkeit des Models geben.

\colorbox{yellow}{Hier fehlt was}

\begin{figure}[h]
    \centering
    \includegraphics[scale=0.26]{images/kapitel_3/model_evaluation.png}
    \caption{Kompletter Model flow}
    \label{fig:umsetzung_model_evaluation}
\end{figure}

\begin{figure}[h]
    \centering
    \includegraphics[scale=0.26]{images/kapitel_3/model_information.png}
    \caption{Kompletter Model flow}
    \label{fig:umsetzung_model_information}
\end{figure}

\begin{figure}[h]
    \centering
    \includegraphics[scale=0.26]{images/kapitel_3/model_predictor.png}
    \caption{Kompletter Model flow}
    \label{fig:umsetzung_model_predictor}
\end{figure}

\begin{figure}[h]
    \centering
    \includegraphics[scale=0.26]{images/kapitel_3/model_network_diagram.png}
    \caption{Kompletter Model flow}
    \label{fig:umsetzung_model_network_diagram}
\end{figure}

\subsubsection{Deployment erstellen}
Das in einem vorangegangenen Kapitel erstellte und trainierte Model kann im weiteren durch ein Deployment über eine
REST-Schnittstelle verfügbar gemacht werden. Dazu ist es erforderlich, das Model in einen \texttt{Web service} zu
installieren. Die spätere Wartung und Verwaltung wird dabei von Bluemix übernommen.

Für die Erstellung des Web Services muss man im Watson Dashboard das erstellte Model auswählen. Das folgende Fenster
zeigt mehrere Informationen zu diesem an. Unter anderem ist sichtbar, welche Eingabe- und Ausgabeparameter für das Model
wichtig sind.

Der Reiter \texttt{Evaluation} zeigt vorangegangene Auswertungen des Models an. Über den Menüpunkt \texttt{Lineage}
werden Verknüpfungen und Abstammungen des Models angezeigt.

Für das Deployment ist der Reiter \texttt{Deployments} wichtig. Dieser verwaltet alle Deployments des Modules. Das Löschen
von älteren Deployments oder das Erstellen von neuen ist hier möglich. Die Schaltfläche \texttt{Add Deployment} öffnet
eine Konfigurationsseite zum Erstellen eines neuen Deployments.

Als Type für das Deployment muss man \textit{Web Service} auswählen. Der Name des Deployments ist das einzige Pflichtfeld
und muss befüllt sein. Anschließend wird das Deployment über \texttt{Save} gespeichert und gestartet. Dieser Vorgang kann
wenige Minuten dauern.

Nachdem das Deployment fertiggestellt ist, wird in der Spalte \textit{Status} der aktuelle Wert \texttt{DEPLOY\_SUCCESS}
angezeigt. Die Informationsseite des Deployments kann man über einen klick auf den Namen öffnen.

Das Deployment ist somit erfolgreich erstellt und kann im Weiteren getestet und genutzt werden.

\subsubsection{Deployment testen}
Ein online Test des eingerichteten Deployments ist über das Watson Studio Dashboard möglich. Dies hat den Vorteil, dass
man das Deployment direkt testen und bei Fehlveralten neu trainiert kann.

Auf der Deploymentseite des gespeicherten Models genügt ein klick auf den Namen um die Deploymentinformationen anzuzeigen.
Hier steht neben diversen Informationen auch der Menüpunkt \texttt{Test} zur Verfügung. In diesem öffnet sich eine
zweispaltige Ansicht.

In der linken Spalte befinden sich die Eingabefelder für das trainierte Model (die Inputs). Nachdem man alle Felder mit
Testwerten gefüllt hat, kann man über die Schaltfläche \texttt{Predict} eine Vorhersage starten.

Nach wenigen Sekunden erscheint auf der rechten Seite ein langes JSON-Object, welches den Rückgabewert des Web Services
enthält. Das erste Array, \textit{fields}, listet alle an den Web-Service gesendeten Felder auf (die Inputs).

Das zweite Array, \textit{values}, die an den Web-Service gesendeten wie auch die Vorhergesgaten Werte (die Inputs und
die Targets). Die letzten fünf Werte des Arrays entsprechen den Vorhersagen des trainierten Models (die Targets).

Sollte das Array \textit{values} kleiner sein als das Array \textit{fields}, war das trainieren des Models nicht
erfolgreich.

Unter dem Menüpunkt \texttt{Implementation} werden wichtige Informationen und erforderliche Schritte aufgezeigt, damit
man die Schnittstelle in das eigene Programm integriert kann. Dabei wird auf verschiedene Programmiersprachen eingegangen.

In Abbildung~\ref{fig:umsetzung_deployment_test} auf Seite~\pageref{fig:umsetzung_deployment_test} ist ein Beispiel für
den online Aufruf des trainierten Models sichtbar.

\begin{figure}[h]
    \centering
    \includegraphics[scale=0.26]{images/kapitel_3/deployment_test.png}
    \caption{Online Test des trainierten Models}
    \label{fig:umsetzung_deployment_test}
\end{figure}

\subsubsection{Aufruf mit Postman}
\label{subsec:Aufruf mit Postman}
Im Weiteren soll das erstellte Model, welches im vorherigen Kapitel als Deployment in einen Web-Service zur Verfügung
gestellt wurde mit Postman\footnote{https://www.getpostman.com} getestet werden.

So ist sichergestellt, dass man das Model auch von extern, nicht nur über das Watson Studio Dashboard, aufrufen kann.
Dies ist für die spätere Entwicklung des Frontends und die Einrichtung des API Connect Services wichtig. Außerdem ist
so eine überprüfung der Übergabeparameter sowie der Rückgabewerte an die Schnittstelle möglich.

Jeder Request an den Web-Service des trainierte Models benötigt einen \textit{Authentication Token} (kurz Auth-Token).
Dieser Token stellt sicher, dass es sich um einen gültigen Aufruf handelt.

Über die REST-Schnittstelle des Watson Studios wird der Token generriert. Dabei handelt es sich um eine andere
Schnittstelle als beim Web-Service. Der Token ist immer nur für maximal vier Stunden gültig.

Um einen Auth-Token zu erstellen, muss man den Watson Studio Benutzername und das zugehörige Passwort an die Schnittstelle
übergeben. Ein Beispielaufruf ist in Listing~\ref{Abruf des Auth-Tokens} auf Seite~\pageref{Abruf des Auth-Tokens}
zu sehen. Der Token ist dabei in einem JSON-Object im Rückgabewert enthalten.

\begin{lstlisting}[language=bash, caption=Abruf des Auth-Tokens, label=Abruf des Auth-Tokens]
$ curl --basic --user USERNAME:PASSWORD https://eu-gb.ml.cloud.ibm.com/v3/identity/token
\end{lstlisting}

In Postman kann man diesen Aufruf über die Eingabe der URL und dem HTTP-Type \texttt{GET} machen. Im Reiter
\textit{Authentication} ist der Type \textit{Basic-Auth} auszuwählen.

Im rechten Bereich sollten dann die beiden Eingabefelder für Benutzername und Passwort erscheinen. Der Nutzer muss die
geforderten Daten eingeben und kann dann den Request mit der Schaltfläche \texttt{Send} starten. Nach wenigen Sekunden
erscheint im Bereich \textit{Body} der Token in einem JSON-Objekt. Ähnlich dem Aufruf mit \textit{curl}.

Um nun das eigentliche Deployment aufzurufen, muss man einen neuen Postman-Tab öffnen. Die URL für den Endpunkt ist im
Deployment des Models zu finden und heißt \texttt{Scoring End-point}.

Nachdem die URL des Postman-Requests definiert ist, kann man als HTTP-Type \texttt{POST} auswählen. Im Bereich
\texttt{Authentication} wird der Typ auf \texttt{Baerer} abgeändert und ermöglicht die Eingabe des Tokens. Dieser Token
entspricht dem Rückgabewert des vorangegangenen Postman-Requests.

Die Auswahl des HTTP-Types \textit{POST} ermöglicht die definition des Bereichs \texttt{Body} für den Request. Dabei
handelt es sich um Parameter, welche an die Schnittstelle geschickt werden. Als Datentyp muss man \textit{raw} und als
Type \textit{JSON (application/json)} auswählen. Im Anhang \ref{sec:postmanTestparameter} auf Seite
\pageref{sec:postmanTestparameter} sind Testparameter zu finden, welche als Eingabe genutzt werden können.

Der Nutzer kann abschließend den Request über den Button \texttt{Send} an den Web Service abschicken. Nach wneigen
Sekunden zeigt Postman den erhaltenen Response des neuronalen Netzes an. Hier sollten auch die Vorhersagen enthalten sein.

Auf der Übersichtsseite des REST-Interfaces des
Deployments\footnote{https://watson\-ml\-api.mybluemix.net/?cm\_mc\_uid=61889453441915363064337}, sind noch weitere
Endpunkte und die dafür benötigten Parameter sowie die Rückgabewerte ersichtlich.

\begin{figure}[h]
    \centering
    \includegraphics[scale=0.26]{images/kapitel_3/deployment_postman.png}
    \caption{Beispielrequest von Postman}
    \label{fig:umsetzung_deployment_postman}
\end{figure}

\subsection{TensorFlow.js}
Das neuronale Netz ist nun erfolgreich in einem Deployment online zur Verfügung gestellt. Außerdem ist die Funktion
über einen Aufruf mittels Postman überprüft worden. Im Weiteren wird das trainierte Model in eine TensorFlow.js
Applikation eingebaut.

Dies hat nun mehrere Vorteile. Zum einen kann man das trainierte Model unabhängig von einer Cloud-Lösung nutzen. Die
TensorFlow-Applikation kann zum Beispiel in einen Cloud Foundry-Container geladen werden und ist somit völlig unabhängig
von der Plattform auf der die Applikation läuft. Auch kann sie modular und schnell in beliebigen Regionen instanziiert
werden.

Ein weiterer Vorteil ist die unabhängigkeit, welche damit erreichbar ist. Die Applikation kann selbst verwaltet,
instanziiert oder auch aktualisiert werden. Auch ist es möglich die TensorFlow-Applikation direkt in die Hauptapplikation
zu integrieren.

\subsubsection{Model Exportieren}
Damit das Model in die TensorFlow-Applikation eingebunden werden kann, muss es aus dem Watson Studio exportiert werden.
Dies erfolgt über den Modeler Flow.

Da das Model in Kapitel \ref{subsub:modeler_flow} auf Seite \pageref{subsub:modeler_flow} komplet trainiert wurde, steht
es im Modeler Flow als gelbes Modul bereit.

Über einen rechtsklick auf das trainierte Model und dem Menüpunkt \texttt{Download Model} kann man das Model auf den
Entwicklungsrechner herunterladen. Bei der Datei handelt es sich um eine pb-Datei. Diese Datei beinhaltet alle wichtigen
Informationen über das gebaute neuronale Netz.

Im weiteren kann mit der Entwicklung des Wrappers für das heruntergeladene neuronale Netz begonnen werden. Dafür wird
der Inhalt der Datei in einer Tensor-Flow-Applikation importiert und ausgewertet.

\subsubsection{Wrapper entwickeln}
\colorbox{yellow}{Hier fehlt was}

Die entwickelte Applikation soll im Weiteren über eine Domain aus dem Internet aufrufbar sein. Eine Installation in einen
Cloud Foundry-Container, welcher in der IBM Cloud läuft, ist dafür notwendig. Im folgenden Kapitel werden die dafür
nötigen Schritte erläutert.

\subsubsection{Toolchain einrichten}
Die fertig entwickelte Node.js-Applikation wird im nächsten Schritt in einen Cloud Foundry-Container installiert. Dies
ermöglicht den Aufruf der Applikation über eine Domain, welche automatisch von der IBM Cloud vergeben wird.

Damit der geschriebene Quellcode nicht nach jeder Änderung manuell mittels \texttt{cf push} in einen Cloud
Foundry-Container geladen werden muss, wird hierfür eine Toolchain aus der IBM Cloud genutzt.

Die Nutzung der Toolchain erfordert ein eingerichtetes Git-Repository. Nach jedem \texttt{commit}, welcher in dieses
Repository geschrieben wird, aktiviert sich die Toolchain selbstständig und lädt den entsprechenden Commit herunter.

Anschließend werden, je nach gewählter und eingerichteter Konfiguration, verschiedene Schritte (Phasen) in der Toolchain
durchlaufen, um die Applikation in einen Cloud Foundry-Container zu installieren.

Dabei ist es möglich, die einzelnen Schritte, welche bei einem Deployment durchlaufen werden, selbst zu definieren, oder
eine vorkonfigurierte Toolchain zu nutzen. Es ist allerdings möglich die vorkonfigurierte Toolchain im Nachgang zu
individualisieren. Sie dient lediglich einem schnelleren Start.

Für die Konfiguration der Toolchain muss man die instanziierte Node.js-Runtime, in welcher die entwickelte Applikation
laufen soll, in dem IBM Cloud Dashboard auswählen.

Auf der dann folgenden Seite, im Tab \texttt{Übersicht} (linke Seite), erscheinen fünf Kacheln mit unterschiedlichen
Informationen. Für die Toolchain ist die Karte mit der Aufschrift \texttt{Continous Delivery} entscheidend. Dort gibt es
einen Button mit der Aufschrift \texttt{Aktivieren}.

Ein Klick auf diesen öffnet die Übersicht und eine visuelle Vorschau der Standardkonfiguration der Toolchain. Nun muss
man einen Name eingetragen und die Region auswählen, in der die Toolchain installiert wird. Da die Standardkonfiguration
vorerst völlig ausreichend ist, kann diese direkt übernommen werden. Dafür genügt ein Klick auf \texttt{Erstellen}.

Nach einem kurzen Ladevorgang ist die Toolchain eingerichtet und vorkonfiguriert. Es erscheinen nun vier Karten für
unterschiedliche Bereiche, in denen die IBM Cloud dem Entwicklungszyklus helfen kann.

Im Bereich \texttt{Nachdenken} wird ein Issue-Tracker konfiguriert, in dem zum Beispiel Bugs (Softwarefehler), welche
in der Software entdeckt werden, eingetragen, verwaltet und diskutiert werden können.

In \texttt{Codieren} stehen gleich zwei Kacheln zur Verfügung. Einerseits das konfigurierte Git-Repository, bei dem es
sich um ein auf IBM-Servern gehostetet GitLab handelt. Andererseits findet sich dort eine Web-IDE, auf Basis von Eclipse
Orion, mit der der Quellcode der Anwendung online editiert werden kann.

Im der letzten Kategorie, \texttt{Bereitstellen}, findet sich die Pipeline, welche im nächsten Schritt näher erläutert
und eingerichtet wird. Mit einem Klick auf die Kachel mit der Aufschrift \texttt{Delivery Pipeline} wechselt man in die
Konfiguration.

Nach dem Laden der Seite erscheinen zwei sogenannte \textit{Phasen} (engl. Stages). Jeder Schritt in der Delivery Pipeline
wird durch eine Phase symbolisiert. In einer Phase können zum Beispiel der Quellcode aus dem Git-Repository geladen, oder
die geschriebenen Tests durchgeführt werden.

Die Standardkonfiguration sieht in der \textit{Build Stage} das Herunterladen des Quellcodes aus dem Git-Repository vor
und in der \textit{Deploy Stage} das Einrichten eines Cloud Foundy-Containers.

Für die Node.js-Applikation reicht diese Konfiguration völlig aus, da keine zusätzlichen Installationen oder Einrichtungen
notwendig sind.

Als nächstes muss der geschriebene Quellcode der Applikation lediglich noch in das, in der Toolchain hinzugefügte,
Git-Repository eingecheckt werden. Die URL für das Repository kann man sich in der Toolchain-Übersicht mit einem Klick
auf die Kachel \texttt{Git} anzeigen.

Nach erfolgreichem \texttt{push} der Anwendung in das Git-Repository startet der deployment Vorgang in der Toolchain
selbstständig. Nach wenigen Minuten ist die Anwendung über ihre URL, welche in der Node.js Runtime definiert ist,
aufrufbar.

Da die Applikation nun im Internet in einem Cloud Foundry-Container zur Verfügung steht, kann diese im nächsten Schritt
mit dem API Connect Service verbunden werden. Dieser Schritt ist nötig, damit die Schnittstelle der Applikation vom
Frontend, welches in Kapitel~\ref{subsec:webseite} auf Seite~\pageref{subsec:webseite} beschrieben wird, und auch über
Postman vereinfacht aufgerufen werden kann.

\begin{figure}[h]
    \centering
    \includegraphics[scale=0.26]{images/kapitel_3/toolchain_pipeline.png}
    \caption{Übersicht der Toolchain-Konfiguration}
    \label{fig:umsetzung_toolchain_pipeline}
\end{figure}

\subsection{API Connect}
Wie in Kapitel \ref{subsec:Aufruf mit Postman} auf Seite \pageref{subsec:Aufruf mit Postman} beschrieben, benötigt das
REST-Interface des erstellten Deployments einen Auth-Token. Dieser kann nur über den Aufruf einer anderen
REST-Schnittstelle zur Verfügung gestellt werden.

Damit man diese Schritte vereinfachen kann und das im weiteren Verlauf erstellte Frontend ebenfalls das Deployment aufrufen
kann, werden beide Abfragen in API Connect gebündelt.

Ein weiterer Grund für das Bündeln der beiden Anfragen an den Watson Service ist das mitschicken des Benutzernamens und
des zugehörigen Passwortes zum generieren des Tokens. Damit diese Daten nicht in der zu entwickelnden Anwendung hinterlegt
werden müssen, können diese Zentral in API-Connect gespeichert werden.

Das hat den Vorteil, dass man sie bei Bedarf nur an einer zentralen Stelle abändern muss. Desweiteren ist es relativ
einfach Daten auszulesen, die von der Anwendung an die Schnittstelle geschickt werden. Dies ermöglicht es Angreifern die
Daten für andere Zwecke zu nutzen.

Ein weiterer Vorteil für die Nutzung von API Connect ist die bündelung von mehreren Schnittstellen um Vorhersagen zu
beziehen. Aktuell ist es Möglich, die Vorhersage aus dem Deployment des Watson Studio Models zu beziehen oder die
Entwickelte Node.js-Applikation mit TensorFlow.js zu nutzen.

Mittels API Connect kann man die Aufrufe optimal verteilen oder eigene Routen für die jeweile Schnittstelle bauen.
Außerdem können Änderungen schnell und kompfortabel in einem Online-Editor angepasst werden.

\subsubsection{API Connect einrichten}
Der in Kapitel \ref{subsec:apiconnect} auf Seite \pageref{subsec:apiconnect} eingerichtete Service wird im folgenden
konfiguriert und eingerichtet. Dazu muss er aus dem IBM Cloud Dashboard heraus aufgerufen werden.

\colorbox{yellow}{Hier fehlt was}

In der Abbildung \ref{fig:umsetzung_api_connect} auf Seite \pageref{fig:umsetzung_api_connect} ist der fertig
eingerichtete API Connect Flow dargestellt.

\begin{figure}[h]
    \centering
    \includegraphics[scale=0.26]{images/kapitel_3/api_connect.png}
    \caption{Kompletter API Connect flow}
    \label{fig:umsetzung_api_connect}
\end{figure}
\section{Abschluss}
Das neuronale Netz ist an dieser Stelle fertig trainiert und erfolgreich als Modell exportiert. Auch kann man es als
Deployment über einen Webdienst aufrufen und Vorhersagen triggern.

Mittels internem Onlinetest und über das externe Programm \textit{Postman} wurden der Aufruf des Webdienstes überprüft
und die Parameter verifiziert.

Desweiteren ist das trainierte Modell erfolgreich in einen Node.js Wrapper eingebaut, der eine TensorFlow.js
Anwendung beherbergt. Diese Anwendung lädt beim Ausführen das aktuelle Modell in den Speicher und parametrisiert es
mit den Eingabevariablen.

Nach kurzer Zeit werden die vorhergesagten Parameter als Rückgabewert über den Request zurückgegeben, und eine Anfrage
wurde so erfolgreich durchgeführt.

Sowohl der Webdienst als auch der Node.js Wrapper sind in einem API-Gateway vereint, sodass Aufrufe an die beiden
Endpunkte vereinfacht werden können. Auch Authentifizierung und Sicherheitsmechanismen laufen über den API-Gateway, um
eine Anpassung zu vereinfachen.

Um dem Endnutzer eine vereinfachte Möglichkeit zu bieten, Anfragen an die beiden Vorhersagemodelle zu schicken, wird
im nächsten Schritt ein Frontend dafür eingerichtet. Mit diesem ist es möglich, die Inputs für die Bosch KWE einzugeben
und die vorhergesagten Parameter schön darzustellen.

Außerdem ist so ein Test der kompletten Architektur mit Webdienst, Node.js Wrapper mit integrierter TensorFlow.js
Applikation und API Connect möglich.

Über das Frontend soll ein Request an den API-Gateway gesendet werden, welcher dann weiter an ein Vorhersagemodell
geschickt wird. Die Antwort des Modells gelangt dann zurück zum API-Gateway und dann wieder zurück an das Frontend.

In einem Aufruf sind alle bisher gebauten Komponenten involviert, sodass durch einen Test das Zusammenspiel der
Kommunikation der einzelnen Komponenten möglich ist.

Die dafür notwendige Umsetzung wird in Kapitel~\ref{ch:client} ab Seite~\pageref{ch:client} behandelt und Schritt für
Schritt realisiert.

\chapter{Client}
Hier beschreiben, was im folgenden Kapitel gemacht werden soll? Was ist das Ziel? Was muss vorbereitet werden?
Es soll der Client aufgebaut werden, welcher dann mit der REST-Schnittstelle kommuniziert.
\\ \\
Ich habe eine Webseite auf Basis von Angular mit Angular-Material gebaut. - Mit der REST-Schnittstelle von Machine Learning
kommunizieren lassen - Auch mit der Schnittstelle von Tensorflow kommunizieren lassen - Zwei Smartphone-Apps aufgebaut,
welche die Webseite in einem WebViewer laden und anzeigen - Vorkehrungen getroffen, damit die Applikation in der Cloud
gebaut und gehostet werden kann (Toolchain, Git) - Bild mit schematischer Architektur (Markiert, was nun gemacht wird.
Rest ausgegraut)
\\ \\
1 - 2 Seiten
\section{Analyse}
Dieses Kapitel soll verschiedene Umsetzungsmöglichkeiten für die Webseite als auch für die Smartphone-Apps analysieren
und eine Einleitung für die präferierte Möglichkeit geben.

Ziel ist es für sowohl die Webseite als auch für die Smartphone-Apps eine optimale Möglichkeit zu finden, diese
umzusetzen, zu verwalten und in Betrieb zu nehmen.

Mit den Ergebnissen wird in den darauffolgenden Kapiteln fortgesetzt und sie werden Schritt für Schritt umgesetzt.

%% TODO noch schreiben
\subsection{Webseite}
Wir könnten eine Webseite aufbauen. Am besten responsive.

\colorbox{yellow}{Hier fehlt was}

%% TODO noch schreiben
\subsection{App}
Wir könnten aber auch eine Native App bauen, mit der wir alles darstellen.

\colorbox{yellow}{Hier fehlt was}

%% TODO noch schreiben
\subsection{Hybrid}
Wie könnten wir das noch machen? WebViewer-App? Also eine Mischung aus beidem.

\colorbox{yellow}{Hier fehlt was}
\section{Vorbereitung}
Damit im Weiteren das Frontend und die Smartphone-Apps entwickelt und gebaut werden können, werden drei Programme auf
dem Entwicklungsrechner installiert und ein Services in der IBM Cloud instanziiert. Die Vorgehensweise wird im
folgenden Kapitel erläutert.

\subsection{Angular CLI}
Für die Erstellung, Verwaltung und Aktualisierung einer Angular-Applikation, empfiehlt es sich das hauseigene Command
Line Interface (CLI) zu nutzen. Über den folgenden Befehl wird dieses auf einem Linux oder macOS Rechner installiert.

\begin{lstlisting}[language=bash, caption=Installation des Angular CLI, label=ls:vorbereitung_angularcli]
    $ npm install -g @angular/cli
\end{lstlisting}

Anschließend steht die Applikation über das Kommando \texttt{ng} zur Verfügung. Dies ist eine Abkürzung für
\textit{A\textbf{ng}ular}.

\subsection{Android Studio}
Für die Entwicklung einer Android-App stehen zahlreiche Tools zur Verfügung. Eine der bekanntesten ist das Android
Studio. Bei diesem Programm handelt es sich um eine IDE auf Basis von IntelliJ Community
Edition\footnote{https://www.jetbrains.com/idea} und ist kostenlos.

Die Installation von Android Studio erfolgt durch das herunterladen des jeweiligen Installationspaketes auf der
Installationsseite\footnote{https://developer.android.com/studio/install}.

Nach erfolgreicher Installation kann die IDE gestartet und ein neues Projekt angelegt werden.

\subsection{Xcode}
iOS-Apps können lediglich durch das Apple eigene Xcode entwicklet werden. Die Insallation kann nur auf einem macOS
erfolgen und dort durch den internen App Store\footnote{https://itunes.apple.com/de/app/xcode/id497799835}.

Nach erfolgreicher Installation erscheint die IDE im Dock und ist bereit zur Nutzung.

\subsection{Node.js Runtime}
Damit das erstellte Frontend online aufgerufen werden kann, wird in der IBM Cloud eine zusätzliche Node.js Runtime
benötigt. Genau wie in Kapitel~\ref{ssc:nodejs_runtime} auf Seite~\pageref{ssc:nodejs_runtime} beschrieben, ist die
Einrichtung und Konfiguration der Runtime über den folgenden Befehl möglich.

\begin{lstlisting}[language=bash, caption=Instanziierung der Node.js Runtime, label=ls:vorbereitung_nodejsdashboard]
    $ ibmcloud cf create-service nodejs service_name
\end{lstlisting}

Anschließend steht die Runtime in der Kategorie \textit{Cloud Foundry-Anwendungen} im IBM Cloud Dashboard zur Verfügung
und kann genutzt werden.
\section{Umsetzung}
Hier noch generell was beschreiben. - Zielarchitekturbild darstellen (komplette Architektur)

\colorbox{yellow}{Hier fehlt was}

\begin{figure}[h]
    \centering
    \includegraphics[width=\textwidth]{images/kapitel_4/architektur_uebersicht.pdf}
    \caption{Übersicht über die Zielimplementierung}
    \label{fig:umsetzung_zielarchitektur_4}
\end{figure}

\subsection{Webseite}
\label{subsec:webseite}
Bauen und erstellen der eigentlichen Webseite
Mockups werden auch erstellt. Wie sehen die aus etc.

\begin{figure}[h]
    \centering
    \includegraphics[width=\textwidth]{images/kapitel_4/website_input.png}
    \caption{Kompletter Model flow}
    \label{fig:umsetzung_website_input}
\end{figure}

\begin{figure}[h]
    \centering
    \includegraphics[width=\textwidth]{images/kapitel_4/website_output.png}
    \caption{Kompletter Model flow}
    \label{fig:umsetzung_website_output}
\end{figure}

\colorbox{yellow}{Hier fehlt was}

\subsubsection{Mockups erstellen}
\colorbox{yellow}{Hier fehlt was}

\subsubsection{Webseite umsetzen}
\colorbox{yellow}{Hier fehlt was}

\subsubsection{Offline-Mode}
\colorbox{yellow}{Hier fehlt was}

\subsubsection{Toolchain einrichten}
\colorbox{yellow}{Hier fehlt was}

\subsection{Smartphone App}
Hier muss ein bisschen Beschreibung hin.

\colorbox{yellow}{Hier fehlt was}

\subsubsection{Android}
Bauen und erstellen der Apps

\colorbox{yellow}{Hier fehlt was}

\subsubsection{iOS}
Bauen und erstellen der Apps

\colorbox{yellow}{Hier fehlt was}
\section{Abschluss}
Sowohl das Frontend als auch die Smartphone-Apps für Android und iOS sind nun erfolgreich entwickelt und getestet. In
der Abbildung~\ref{fig:umsetzung_zielarchitektur_4} auf Seite~\pageref{fig:umsetzung_zielarchitektur_4} ist die
Gesamtarchitektur mit dem Frontend auf der linken Seite und dem Backend auf der rechten Seite aufgezeigt.

Die Kommunikation der beiden Module erfolgt über den in der Mitte angelegten API Connect Service in der IBM Cloud,
welcher als API-Gateway fungiert und anfallende Anfragen um Informationen erweitert und anschließend weiterleitet.

Die Architektur ist somit erfolgreich umgesetzt und kann von einem Endnutzer genutzt werden, um Vorhersagen für die
aktuellen Bosch Wiegeeinheiten zu erhalten.

Im nächsten Schritt soll die erstellte Architektur nun auf weitere Maschinen angewandt werden. Um dies zu zeigen, wird
im Kapitel~\ref{ch:adaptierbarkeit} ab Seite~\pageref{ch:adaptierbarkeit} die Adaptierbarkeit der Architektur
aufgezeigt.

Damit soll erreicht werden, dass weitere Maschinen in der Produktpalette der Robert Bosch GmbH die Vorteile von
künstlicher Intelligenz im Bereich der Parameteroptimierung erfahren können.

Dafür werden weitere neuronale Netze für die einzelnen Maschinen angelegt und das Frontend dahingehend erweitert,
dass neue Daten eingegeben werden können.

\begin{figure}[h]
    \centering
    \includegraphics[width=\textwidth]{images/kapitel_4/architektur_uebersicht.pdf}
    \caption{Übersicht der Zielarchitektur}
    \label{fig:umsetzung_zielarchitektur_4}
\end{figure}

\chapter{Adaptierbarkeit}
\label{ch:adaptierbarkeit}
Die in Kapitel~\ref{ch:neuronalesNetz} ab Seite~\pageref{ch:neuronalesNetz} aufgebaute und erstellte Architektur soll
zusammen mit dem in Kapitel~\ref{ch:client} ab Seite~\pageref{ch:client} implementierten Frontend durch weitere
Maschinen oder Module erweitert und verbessert werden.

Eine Untersuchung soll zeigen, ob sowohl die Architektur als auch das Frontend so modular entwickelt wurden, dass sie
problemlos um weitere Maschinen oder Module erweitert werden können. Eventuelle Probleme im System oder in der
Architektur sollen gelöst werden.

Auch soll so auf neue Versionen oder Anpassungen der bereits implementierten Maschine reagiert werden können.

Für dieses Ziel werden im folgenden Kapitel alle notwendigen Schritte zum Einbinden einer Schlauchbeutelmaschine
behandelt und erläutert. Anschließend werden diese Schritt für Schritt umgesetzt.

Die Abbildung~\ref{fig:schematische_architektur_5} auf Seite~\pageref{fig:schematische_architektur_5} zeigt eine
schematische Erweiterung des Systems. Dort ist sehr gut ersichtlich, dass der API Connect Service eine zentrale Rolle
in der Erweiterbarkeit einnimmt.

\begin{figure}[h]
    \centering
    \includegraphics[width=\textwidth]{images/kapitel_5/architektur_schematisch.pdf}
    \caption{Schematische Darstellung der Adaptierbarkeit}
    \label{fig:schematische_architektur_5}
\end{figure}

\section{Schlauchbeutelmaschine}
Mit diesem Beispiel kann man die bestehende Architektur durch eine weitere Maschine ergänzen. Es ist problemlos möglich
auch Weitere oder andere Maschinen oder gar Module einer Maschine einzurichten. Allerdings soll mit diesem Beispiel die
allgemeine Vorgehensweise erklärt werden und dieses Beispiel veranschaulicht dies.

In Abbildung~\ref{fig:siegelmaschinen_vffs} auf Seite~\pageref{fig:siegelmaschinen_vffs} ist der schematische Aufbau der
vertikalen Schlauchbeutelmaschine abgebildet.

\begin{figure}[h]
    \centering
    \includegraphics[scale=1]{images/kapitel_5/vffs.jpg}
    \caption{Aufbau einer vertikalen Schlauchbeutelmaschine~\cite{online_grundlagen_boschkwe}}
    \label{fig:siegelmaschinen_vffs}
\end{figure}

Dort ist gut zu sehen, dass im oberen Bereich der Abbildung, unter der Nummer 2, dass zu befüllende Produkt einlegt
wird. Dies fällt dann durch das Rohr mit der Nummer 3. Die Folie umschließt das Produkt anschließend und diese wird mit
den Siegelbacken (Nummer 5) versiegelt. Bei Nummer 6 trennt ein \textit{Trennmesser} den befüllten Beutel von der
Endlosfolie ab und so kann der befüllte Beutel weiterverarbeitet werden.

Die wichtigsten Einstellparameter der Maschine betreffen die Siegelung des Beutels. Da zahlreiche Folienarten von
verschiedenen Anbietern existieren ist es nicht immer einfach, die Folie richtig und korrekt zu versiegeln.

Die \textit{Siegelnahtfestigkeit} gibt an, welche Kraft notwendig ist, um die verschlossene Siegelnaht wieder zu öffnen.
Dies ist insbesondere dann Wichtig, wenn es sich bei dem verpackten Produkt um ein Endkunden-Produkt handelt. Der Käufer
sollte die Verpackung unter normalem Kraftaufwand öffnen können, damit er an das Produkt gelangen kann.

Allerdings darf die Siegelnaht nicht zu schwach sein, da sonst Luft oder andere Gase in die Verpackung eindringen
könnten. Für die vertikalen Schlauchbeutelmaschine ist es entscheidend die richtige Siegelnahtfestigkeit zu erzeugen.

Zur Erzeugung einer Siegelnaht sind Daten zur \textit{Siegeltemperatur}, \textit{Siegelzeit} und \textit{Folienart}
interessant. Bei der Folie spielt der Aufbau der Folie (anzahl der Lagen und genutzte Materialien) eine Rolle.

Der genaue Aufbau von Folien wird von den entsprechenden Herstellern oft geheim gehalten und man muss ihn entweder
händisch ermitteln oder auf Erfahrungen von Mitarbeitern zurückgreifen.

Aus diversen Tests kann man anhand der eingestellten Parameter für Siegeltemperatur und Siegelzeit und der genutzten
Folie die Siegelnahtfestigkeiten ableiten, indem man den Kraftaufwand zum öffnen der Siegelnaht misst.

Für das weitere Beispiel ist es nun interessant die Siegelnahtfestigkeit aus einem neuronalen Netz zu ermitteln, sofern
die Zusammensetzung der Folie sowie die Siegeltermperatur und Siegelzeit feststeht.

Ein weiteres Beispiel könnte sein, die Siegelnahtfestigkeit und genutzte Folie vorzugeben und das neuronale Netz
ermitteln zu lassen, welche Temperatur und welche Zeit für das Siegeln eingeplant werden muss. Dieses Beispiel wird
aktuell aber nicht weiter verfolgt.

\subsection{Daten zusammenstellen}
Um an Testdaten für das Aufbauen eines neuronalen Netzes für diesen Maschinentyp zu kommen, sind zahlreiche
zeitintensive Tests und Versuche notwendig.

Für die Versuche muss man verschiedene Temperaturen mit verschiedenen Siegelzeiten mit verschiedenen Folien kreuzen, was
einen enormen Zeitaufwand bedeutet.

Durch eine Zugprüfmaschine kann man anschließend die Schälfestigkeit bestimmt -- die Kraft, die bei einer schälenden
Beanspruchung aufgewendet werden muss -- um eine Naht zu öffnen. Die Prüflinge sollten dabei eine Breite von 15mm und
eine Länge 40mm aufweisen.

Herr Felix Kruppa von der Robert Bosch GmbH hat während seiner Dissertation einen Siegelteststand aufgebaut, welcher
verschiedene Siegelnähte mit unterschiedlichen Siegelzeiten und "~temperaturen erzeugen kann.

Für den Aufbau seines Teststandes durchlief er zahlreiche Tests, welche er mit einem Protokoll dokumentierte. Diese
Daten können für die Erstellung des neuronalen Netzes fungieren, da er ebenfalls die resultierende Siegelnahtfestigkeit
ermittelte und in seiner Dokumentation festhielt.

In Abbildung~\ref{fig:siegelmaschinen_vffs_simulator} auf Seite~\pageref{fig:siegelmaschinen_vffs_simulator} ist der
aufgebaute Siegelteststand von Herrn Felix Kruppa mit Beschriftungen zu sehen.

\begin{figure}[h]
    \centering
    \includegraphics[width=\textwidth]{images/kapitel_5/vffs_simulator.png}
    \caption{Aufbau des Siegelteststandes}
    \label{fig:siegelmaschinen_vffs_simulator}
\end{figure}

Die kombinierten Daten aus den Testdurchläufen von Herrn Kruppa sind im Anhang~\ref{sec:schlauchbeutelmaschine}
auf Seite~\pageref{sec:schlauchbeutelmaschine} zu sehen und man kann diese direkt für das Trainieren des neuronalen
Netzes in der Cloud oder auch offline nutzen.

\subsection{Neuronales Netz trainieren}
Mit Hilfe der zusammengestellten Testdaten kann man nun das neuronale Netz trainieren. Das Training funktioniert in
gleicher Weise wie auch in Kapitel~\ref{subsub:modeler_flow} ab Seite~\pageref{subsub:modeler_flow} mit dem Modeler
Flow in der IBM Cloud.

Dazu werden die Testdaten in den Watson Studio Service hochgeladen und anschließend mit einem Data Refinery Flow in eine
CSV-Datei umgewandelt. Der Machine Learning Service der IBM Cloud kann aktuell lediglich mit CSV-Dateien zum Trainieren
von neuronalen Netzen umgehen.

Auch kann man die Datei schon auf dem Entwicklungsrechner in eine CSV-Datei konvertieren um sich die Konvertierung im
Service zu sparen. So muss man keinen neuer Flow für die Data Refinery anlegen. Es spielt keine Rolle wo man die Datei
konvertiert.

Anschließend muss man einen neuen Modeler Flow anlegen und diesen genau gleich dem Flow in
Abbildung~\ref{fig:umsetzung_model_flow} auf Seite~\pageref{fig:umsetzung_model_flow} aufbauen.

In diesem Fall ändern sich aber die Parameter für die Eingabe"~ und Ausgabeparameter wie in
Tabelle~\ref{tab:targets_inputs_siegeln} auf Seite~\pageref{tab:targets_inputs_siegeln} zu sehen.

\begin{table}[h]
    \centering
    \begin{tabular}{|c|c|}
        \hline
        \textbf{Targets} & \textbf{Inputs}\\
        \hline
        \hline
        Mittelwert & Eindringverhalten\\
        \hline
        & Temperatur\\
        \hline
        & Vorwärmphase\\
        \hline
        & Eindringphase\\
        \hline
        & Siegelphase\\
        \hline
        & End-Siegel-Position\\
        \hline
    \end{tabular}
    \caption{Variablen für die Targets und die Inputs der Schlauchbeutelmaschine}
    \label{tab:targets_inputs_siegeln}
\end{table}

Nach erfolgreicher Konfiguration der einzelnen Module kann man das neuronale Netz trainieren lassen um das Modell zu
erhalten. Diese Modell kann man dann im Weiteren in einem Deployment zur Verfügung stellen.

\subsection{Deployment erstellen}
Um das Deployment für das trainierte Modell zu erstellen, muss man das Modell im Modeler Flow mit der rechten Maustaste
auswählen und \enquote{Save branch as a model} auswählen. Dies speichert das trainierte Modell als eigenständiges Modell
im Watson Studio ab.

Dies kann man nun über das Watson Studio Dashboard auswählen und über den Reiter \texttt{Deployments} auf alle
Webservices des einen Modells zugreifen. Dort sind aktuell aber noch keine sichtbar.

Über die Schaltfläche \texttt{Add Deployment} legt man ein neues an. Dies geschieht genau wie in
Kapitel~\ref{subsec:deployment_erstellen} ab Seite~\pageref{subsec:deployment_erstellen} beschrieben.

Nach der Eingabe eines Namens und der Speicherung des Deployments erscheint es in der vorher leeren Liste. Damit das das
trainierte Modell in einem Webservice zur Verfügung getsellt.

In der Abbildung~\ref{fig:siegelmaschinen_deployment} auf Seite~\pageref{fig:siegelmaschinen_deployment} ist die
komplette Konfiguration des Deployments für den Webservice ersichtlich.

\begin{figure}[h]
    \centering
    \includegraphics[width=\textwidth]{images/kapitel_5/vffs_deployment.png}
    \caption{Deployment der Schlauchbeutelmaschine}
    \label{fig:siegelmaschinen_deployment}
\end{figure}

Nun kann man in einem weiteren Schritt den API Connect Service um diesen Endpunkt erweitern, um auch Vorhersagen von
diesem neuronalen Netz zu beziehen um die Siegelnahtfestigkeit zu ermitteln.

\subsection{API Connect erweitern}
Der API Connect Service fungiert als Knotenpunkt zwischen Frontend und den möglichen Vorhersage-Services. Damit das
Frontend auf neue Services vereinfacht zugreifen kann, muss man hier einen neuen Endpunkt anlegen. Dies kann man
entweder über die Erweiterung der JSON-Konfiguration oder über den grafischen Editor machen.

Der einfachere Weg ist die Anpassung der Endpunkt über den grafischen Editor. Damit man den neuen Service auch über den
API Service nutzen kann, muss man im Reiter \texttt{Gestalten} in der Konfiguration \texttt{Pfade} einen neuen anlegen.

Diesen kann man beliebig benamen. Ein Beispiel hierfür ist \textit{predict-seal}. Darüber hinaus muss der Pfad eine
Operation besitzen. Diese muss, genau wie alle anderen Operationen der anderen Pfade, \textit{POST} sein, damit das
Frontend Daten an den Endpunkt schicken kann.

Da der neue Pfad angelegt ist, kann man die Konfiguration speichern und in den Reiter \texttt{Assemblieren} wechseln.
Der \textit{Switch} für die Verzweigung je nach gewünschtem Pfad muss man um den gerade eingerichteten neuen Pfad
erweitern. Somit beinhaltet die Switch-Operation drei Verzweigungen.

Der Inhalt für den neuen Pfad kann man direkt vom Pfad \textit{predict-watson} übernehmen und lediglich im
\textit{Get Prediction} Modul definierte URL für den Webservice anpassen.

Die Abbildung~\ref{fig:siegelmaschinen_apiconnect} auf Seite~\pageref{fig:siegelmaschinen_apiconnect} veranschaulicht
den angepassten Modeler Flow um den neuen Endpunkt.

\begin{figure}[h]
    \centering
    \includegraphics[width=\textwidth]{images/kapitel_5/vffs_apiconnect.png}
    \caption{Angepasster API Connect Service}
    \label{fig:siegelmaschinen_apiconnect}
\end{figure}

Damit ist der API Connect Service fertig eingerichtet und man kann den neuen Endpunkt zum Beispiel vom Frontend aus
abrufen. Auch ist es Möglich den Aufruf mittels Postman zu überprüfen. Ein Test sollte die Funktionsweiße des Endpunktes
sicherstellen.

\subsection{Frontend erweitern}
Im letzten Schritt der Adaptierung der Architektur auf eine neue Maschine oder auf ein neues Modul muss man das Frontend
um die gewünschten Eingaben der Parameter erweitern, damit das Frontend die Vorhersage auf dem Webservice starten und
das Ergebnis darstellen kann.

Dafür kann man eine neue Angular-Komponente anlegen und diese dann mit dem gewünschten neuen Design versehen. Das
Anlegen einer Komponente in Angular geschieht am Einfachsten über ein Kommando der Angular-CLI.

\begin{lstlisting}[caption=Erstellen einer neuen Komponente, label=ls:schlauchbeutelmaschine_component]
    ng generate component name
\end{lstlisting}

Der Parameter \textit{Name} dient als Platzhalter und man muss ihn mit einem Namen für die Komponente ersetzen. Dieser
Name darf durch keine andere Komponente vergeben sein. Die neue Komponente erscheint als Unterordner im Ordner
\texttt{app}.

Die neue Komponente besteht aus den vier Dateien \textit{*.component.css} für die Definition von CSS-Eigenschaften,
\textit{*.component.html} zur Konfiguration der Darstellung, \textit{*.component.spec.ts} für das Schreiben von Tests
und \textit{*.component.ts} für die eigentliche Logik.

Der Inhalt der vier Dateien kann man aus der Komponente für die Waage im Ordner \texttt{/app/scales} kopieren und
anschließend Stück für Stück anpassen.

In der HTML-Datei muss man das Design auf die neuen Eingabefelder abändern. Dies ist Wichtig, da andere Parameter für
den Endpunkt notwendig sind.

In der TS-Datei, in der die Logik der Komponente definiert ist, muss man den REST-Endpunkt auf den neuen Namen abändern
und die übergebenen Parameter anpassen, damit die Vorhersage auch funktionieren kann.

Auch ist entscheidend, dass man die Abfrage zu leeren Felder auf die richtige Benamung und die definition von
zufälligen Eingabeparameter auf logische Werte anpasst. So ist es für einen Endnutzer möglich schnell einen Test über
das Frontend zu starten.

Die CSS-Datei sowie die SPEC-Datei bedürfen keiner zwingenden Änderung, da sie für die Funktionsfähigkeit der Abfrage an
das Backend keine Rolle spielen. Allerdings sollte man zu einem späteren Zeitpunkt den Test für die Komponente
berichtigen.

\subsection{Abschluss}
An diesem Punkt ist die Adaption der erstellen Architektur auf die Siegelmaschine abgeschlossen und man kann damit
beginnen, neue Vorhersagen des neuen Webservices durch das Frontend zu beziehen. Die Ergebnisse werden anschließend
ebenfalls auf dem Frontend angezeigt.

Für alle weiteren Anpassungen oder Erweiterungen der Architektur muss man lediglich diese fünf relativ einfachen
Schritte wiederholen und um die neuen Einstellungen anpassen.

Damit ist gezeigt, dass die gebaute Architektur so modular ist, um sie problemlos zu erweitern und neue Funktionen
darin einzubauen. Dies galt es als Teilaufgabe in der Arbeit zu zeigen und beweisen.
\chapter{Ausblick}
\label{ch:ausblick}
Dieses Kapitel soll einen Überblick über weitere größere Funktionen und Erweiterungen geben, durch die die Architektur,
die Webanwendung und die Smartphone-Apps erweitert und verbessert werden können.

Außerdem sollen Möglichkeiten aufgezeigt werden, welche die Arbeit mit Maschinen und künstlicher Intelligenz
vereinfacht. Dazu zählt unter anderem eine Möglichkeit, eine direkte Verbindung zwischen Maschine und künstlicher
Intelligenz herzustellen.

Auch sind Themen, welche die klassische Container-Architektur verbessern für Erweiterungen interessant oder
Möglichkeiten zur schnelleren Vorhersage von Daten und eingabe von Daten unter erschwerten Bedingungen.

%% TODO noch schreiben
\section{Eine direkte Verbindung}
Daten in Maschine generrieren, IoT hochladen, in Maschine Learning über Connections einbauen und immer wieder
trainieren.

Allerdings sind automatisierte Tests nicht immer und für jede Anwendung geeignet. Michael Lüttel von der Deutschen 
Flugsicherung sagte auf der iqnite-Konferenz in Düsseldorf \enquote{Automatisierung macht nur dann Sinn, wenn sie mehr 
Aufwände einspart als sie selbst erzeugt.}\footnote{https://www.qz-online.de/news/uebersicht/nachrichten/vor-und-nachteile-von-automatisierten-software-tests-890130.html}

\colorbox{yellow}{Hier fehlt was}

\begin{figure}[h]
    \centering
    \includegraphics[width=\textwidth]{images/kapitel_6/architektur_uebersicht.pdf}
    \caption{Übersicht der Zielarchitektur}
    \label{fig:ausblick_uebersicht}
\end{figure}

%% TODO noch schreiben
\section{Function as a Service}
Wie könnte man Functions/Lambda einbauen. FaaS

\colorbox{yellow}{Hier fehlt was}

%% TODO noch schreiben
\section{Machine Learning für Smartphones}
Wie könnte man das Feature nutzen und was würde das bringen

\colorbox{yellow}{Hier fehlt was}

%% TODO noch schreiben
\section{Offline Modus}
Ein Offline-Mode für die Website mit TensorFlow.js.

\colorbox{yellow}{Hier fehlt was}

%% TODO noch schreiben
\section{AI OpenScale}
\label{ai_openscale}
Hiermit kann man veranschaulichen, warum und wie die AI auf das Ergebnis gekommen ist. Das kann man mit dem Deployment
und mit dem TensorFlow.js Ding machen. Dafür braucht man eine Datenbank.

\colorbox{yellow}{Hier fehlt was}

%% TODO nosch schreiben
\section{Skalierbarkeit}
Wie könnte man die Anwendung skalieren?

\colorbox{yellow}{Hier fehlt was}

%% TODO noch schreiben
\section{Audit mit Blockchain}
Für die Waage muss Audit gemacht werden. Könnte man Blockchian da nicht noch einbauen?

\colorbox{yellow}{Hier fehlt was}

%% TODO noch schreiben
\section{Nutzen von IoT}
Nutzen von IoT

\colorbox{yellow}{Hier fehlt was}

\begin{figure}[h]
    \centering
    \includegraphics[width=\textwidth]{images/kapitel_6/iot_waage.pdf}
    \caption{Übersicht der Zielarchitektur}
    \label{fig:ausblick_iot}
\end{figure}

%% TODO noch schreiben
\subsection{Daten einlesen}
Wie könnten die Daten, welche durch das Netzwerk herausgefunden werden, auch automatisch in die Machine eingegeben werden?

\colorbox{yellow}{Hier fehlt was}

%% TODO noch schreiben
\subsection{Daten auslesen}
Wie könnte man Daten der Maschine auslesen, damit man sie nutzen kann um das Neuronale Netz weiter zu verbessern?

\colorbox{yellow}{Hier fehlt was}
\chapter{Zusammenfassung}
\label{ch:zusammenfassung}
Hier noch mal alles zusammenfassen was ich vorher so geschrieben habe! Was macht meine Anwendung? Was habe ich gebaut?
Welche Schritte habe ich absolviert? Abschließendes Fazit -> Ein intelligenter Spruch über iOT, Machine Learning
\\ \\
Hypothese am Anfang, dass ich das Problem mit Machine Learning lösen will... Hat geklappt. Ich konnte zeigen das es geht.

\colorbox{yellow}{Hier fehlt was}

% Alle Einträge der BibTeX Datenbank zitieren
\nocite{*}

% Einstellen des Bibliography-Stils für das Literaturverzeichnis
\bibliographystyle{abbrvdin}

% Auswahl der BibTeX Datenbank für das Literaturverzeichnis
\bibliography{literatur}

% Abbildungsverzeichnis ausgeben
\listoffigures

% Tabellenverzeichnis ausgeben
\listoftables

% Das Listings-Verzeichnis scheint mit manchen Versionen vom Koma-Script
% bzw. des Listings Pakets nicht ohne weiteres zu funktionieren
\lstlistoflistings

% Anhang beginnen (Formatierung umschalten)
\appendix
\chapter{Anhang}
\label{ch:anhang}

%% TODO noch hinzufügen
\section{Konfiguration des API Connect Services}
\label{sec:konfigurationAPIConnect}
\colorbox{yellow}{Hier fehlt was}

\section{Datensatz der KWE Waage}
\label{sec:scaleData}
Die größe des Dokuments und die Anzahl der Zeilen und Spalten würde den hiesigen Platz sprengen. Desshalb liegt das
Dokument auf der beigelegten CD mit dem Namen \texttt{datensatz\_kwe.xlsx} bereit.

\section{Datensatz der Siegelmaschine}
\label{sec:hariboSiegel}
Die größe des Dokuments und die Anzahl der Zeilen und Spalten würde den hiesigen Platz sprengen. Desshalb liegt das
Dokument auf der beigelegten CD mit dem Namen \texttt{datensatz\_siegelmaschine.xlsx} bereit.

\newpage

\section{Postman Testparameter}
\label{sec:postmanTestparameter}

\begin{lstlisting}[language=json, caption=Testparameter für Postman, label=ls:anhang_postman, escapeinside=``]
    {   "fields": [
            "Einlaufbandl`\textcolor{blue}{ä}`nge",
            "W`\textcolor{blue}{ä}`gebandl`\textcolor{blue}{ä}`nge",
            "Auslaufbandl`\textcolor{blue}{ä}`nge",
            "Einlaufbandbreite",
            "W`\textcolor{blue}{ä}`gebandbreite",
            "Auslaufbandbreite",
            "Einlaufbandrolle",
            "W`\textcolor{blue}{ä}`gebandrolle",
            "Auslaufbandrolle",
            "Produktbreite",
            "Produktl`\textcolor{blue}{ä}`nge",
            "Produkth`\textcolor{blue}{ö}`he",
            "Leistung",
            "Packungsgewicht",
            "Bandgeschwindigkeit",
            "Position",
            "Totzeit",
            "Impuls",
            "Druckluft"
        ],
        "values": [
            [
                415,
                360,
                470,
                250,
                250,
                250,
                29,
                29,
                29,
                240,
                150,
                20,
                220,
                275.86,
                null,
                null,
                null,
                null,
                null
            ]
        ]}
\end{lstlisting}

\newpage

\section{Konfiguration des Service Worker}
\label{sec:serviceWorkerConfig}

\begin{lstlisting}[language=json, caption=Konfiguration des Service Workers, label=ls:anhang_serviceworker]
    {
        "index": "/index.html",
        "assetGroups": [
            {
                "name": "app",
                "installMode": "prefetch",
                "updateMode": "prefetch",
                "resources": {
                    "files": [
                        "/favicon.ico",
                        "/index.html",
                        "/*.css",
                        "/*.js"
                    ]
                }
            },
            {
                "name": "assets",
                "installMode": "prefetch",
                "updateMode": "prefetch",
                "resources": {
                    "files": [
                        "/assets/**"
                    ]
                }
            },
            {
                "name": "fonts",
                "installMode": "lazy",
                "updateMode": "lazy",
                "resources": {
                    "urls": [
                        "https://fonts.googleapis.com/**",
                        "https://fonts.gstatic.com/**"
                    ]
                }
            }
        ]
    }
\end{lstlisting}

% Ausgabe des Wortindex
%\clearpage
%\addcontentsline{toc}{chapter}{Index}
%\printindex

\end{document}
\documentclass[
BCOR=5mm,           % Binderkorrektur von 5mm vorsehen
DIV10,              % Seite in X Kästchen einteilen (Siehe Koma-Script Guide)
%DIVcalc,           % Besten DIV Wert berechnen (Siehe Koma-Script Guide)
fontsize=11pt,      % Schriftgröße 11 Punkte
oneside,            % Einseitig
parskip,            % Paragraphen nicht einrücken
headsepline,        % Kopfzeile nach unten durch Linie abgrenzen (scrheadings)
%footbotline,       % Fußzeile nach unten durch Linie abgrenzen (scrheadings)
plainheadsepline,   % Kopfzeile nach unten durch Linie abgrenzen (scrplain)
plainfootbotline,   % Fußzeile nach unten durch Linie abgrenzen (scrplain)
%headtopline,       % Kopfzeile nach oben durch Linie abgrenzen (scrheadings)
footsepline,        % Fußzeile nach oben durch Linie abgrenzen (scrheadings)
plainheadtopline,   % Kopfzeile nach oben durch Linie abgrenzen (scrplain)
plainfootsepline,   % Fußzeile nach oben durch Linie abgrenzen (scrplain)
headexclude,        % Kopfzeile nicht als Teil des Inhalts setzen
footexclude,        % Fußzeile nicht als Teil des Inhalts setzen
%bibtotocnumbered,  % Literaturverzeichnis nummeriert ins Inhaltsverzeichnis aufnehmen
bibtotoc,           % Literaturverzeichnis ins Inhaltsverzeichnis aufnehmen
%liststotocnumbered,% Sonstige Verzeichnise nummeriert ins Inhaltsverzeichnis aufnehmen
liststotoc,         % Sonstige Verzeichnise ins Inhaltsverzeichnis aufnehmen
idxtotocnumbered    % Index nummeriert ins Inhaltsverzeichnis aufnehmen
%idxtotoc           % Index ins Inhaltsverzeichnis aufnehmen
]{scrbook}          % Koma-Script Klasse zum setzen eines Buchs

% Die "Standard-Header" für deutsche Dokumente
\usepackage[utf8]{inputenc}
\usepackage[T1]{fontenc}         % T1 Schriften verwenden (sieht besser aus)
\usepackage[ngerman]{babel}      % Neue deutsche Rechtschreibung und Übersetzungenq

% Für URLs und Path
\usepackage[hyphens,spaces,obeyspaces]{url}

% "Schönere" Schriften einbinden
\usepackage{mathpazo}            % Serifen-Font mit passendem Math-Font
\usepackage[scaled=.95]{helvet}  % Serifenloser Font passend zu mathpazo
\usepackage{courier}             % "Schönerer" Festbreiten-Font

\usepackage[babel,german=quotes]{csquotes}

% Koma-Script Paket zum setzen vom Kopf- und Fußzeilen einbinden
\usepackage{scrlayer-scrpage}
% Seitenstil für normale Seiten auf scrheadings setzen
% Für Kapitelanfang und ähnliches wird scrplain verwendet
\pagestyle{scrheadings}
% Kopf- und Fußzeile löschen
\clearscrheadfoot
% Automarkierungen verwenden \automark[rechts]{links}
% Statt \leftmark und \rightmark kann dann bei
% Koma-Script einfach \headmark verwendet werden
\automark[section]{chapter}
% Kopfzeile für scrplain und scrheadings setzen
% \*head[scrplain]{scrheadings}
%\ihead[Innen]{Innen}
%\chead[Mitte]{Mitte}
\ohead[\sffamily\scshape\bfseries\large\headmark]
{\sffamily\scshape\bfseries\large\headmark}
% Fußzeile für scrplain und scrheadings setzen
% \*foot[scrplain]{scrheadings}
%\ifoot[Innen]{Innen}
%\cfoot[Mitte]{Mitte}
\ofoot[\sffamily\thepage]{\sffamily\thepage}
% Trennlinien für Kopf- und Fußzeile formatieren
% Siehe Optionen der Dokumentklasse
%\setheadtopline{2pt}
\setheadsepline{.4pt}
\setfootsepline{.4pt}
%\setfootbotline{2pt}

% Paket zum Einbinden von Quellcode als Listings
% Hinweis: UTF-8 Encoding führt zu Problemen mit Umlauten
\usepackage{listings}
\usepackage{xcolor}
\usepackage{soul}

% Formatierung der Listings
\lstset{
    captionpos=b,                % Beschriftung unterhalb (bottom)
    frame=trbl,                  % Rahmen zeichnen (top, right, bottom, left)
    basicstyle=\small\ttfamily,  % Festbreitenschrift verwenden (small)
    breaklines=true,
    showstringspaces=false,
    commentstyle=\color{red},
    keywordstyle=\color{blue},
    emphstyle={\color{blue}},
    emph={
        bash, curl, ibmcloud, sudo, npm
    }
}

% Paket für definierte Übersetzungen einbinden
\usepackage[ngerman]{translator}

% Paket für Stichwort- Abkürzungs- und sonstige Verzeichnisse einbinden
\usepackage[
nonumberlist, % Keine Seitenzahlen anzeigen
acronym,      % Abkürzungsverzeichnis erstellen
%toc,          % In Inhaltsverzeichnis aufnehmen
%section       % Verzeichniseintrag als Section
]{glossaries}

% Ein eigenes Verzeichnis definieren (Smbolverzeichnis)
% Das Stichwort- und Abkürzungsverzeichnis wird analog vordefiniert
% Siehe makeindex Aufrufe - Hier werden die Dateiendungen festgelegt
\newglossary[slg]{symbolslist}{syi}{syg}{Symbolverzeichnis}

% Den Punkt am Ende der Beschreibung deaktivieren
% \renewcommand*{\glspostdescription}{}

% Stichwort-, Abkürzungs- und Symbolverzeichnis erzeugen
\makeglossaries

% Paket für Wortindex einbinden
\usepackage{makeidx}

% Wortindex erzeugen
\makeindex

% Paket zum generieren von Blindtext
\usepackage{blindtext}

% Paket zum Einbinden von Bildern
\usepackage{graphicx}

%
% WORKAROUND, damit lstlistoflistings funktioniert:
% Quelle: http://www.komascript.de/node/477
%
\makeatletter
\@ifundefined{float@listhead}{}{%
    \renewcommand*{\lstlistoflistings}{%
        \begingroup
    	    \if@twocolumn
                \@restonecoltrue\onecolumn
            \else
                \@restonecolfalse
            \fi
            \float@listhead{\lstlistlistingname}%
            \setlength{\parskip}{\z@}%
            \setlength{\parindent}{\z@}%
            \setlength{\parfillskip}{\z@ \@plus 1fil}%
            \@starttoc{lol}%
            \if@restonecol\twocolumn\fi
        \endgroup
    }%
}
\makeatother

% Andere packages
\usepackage{hyperref}
\usepackage{eurosym}

% PDF-Metadaten
\AfterPreamble{
    \hypersetup {
        pdftitle = {Künstliche Intelligenz zur Parameteroptimierung in der Fertigung},
    	pdfsubject = {Künstliche Intelligenz zur Parameteroptimierung in der Fertigung},
    	pdfauthor = {Lars Helmuth Probst},
    	pdfkeywords = {Master, Lars, Helmuth, Probst},
    	pdfcreator = {Lars Helmuth Probst},
    	pdfproducer = {Lars Helmuth Probst},
    	pdfstartpage = 1,
    	pdffitwindow = true,
    	pdfpagelayout = SinglePage
    }
}

% Numbering
\setcounter{secnumdepth}{5}
\setcounter{tocdepth}{5}

\lstset{aboveskip=20pt,belowskip=0pt}

% Define language javascript
\lstdefinelanguage{JavaScript}{
  keywords={typeof, new, true, false, catch, function, return, null, catch, switch, var, if, in, while, do, else, case, break, apim, let, string},
  keywordstyle=\color{blue}\bfseries,
  ndkeywords={class, export, boolean, throw, implements, import, this},
  ndkeywordstyle=\color{darkgray}\bfseries,
  identifierstyle=\color{black},
  sensitive=false,
  comment=[l]{//},
  morecomment=[s]{/*}{*/},
  commentstyle=\color{purple}\ttfamily,
  stringstyle=\color{red}\ttfamily,
  morestring=[b]',
  morestring=[b]",
  numbers=left,
  stepnumber=1
}

% Define language html
\lstdefinelanguage{HTML}{
    sensitive=true,
    keywords={%
    % JavaScript
    typeof, new, true, false, catch, function, return, null, catch, switch, var, if, in, while, do, else, case, break,
    % HTML
    html, title, meta, style, head, body, script, canvas, domain, allowedOrigins, allowedMethods, allowedHeaders, allowCredentials, maxAge,
    % CSS
    border:, transform:, -moz-transform:, transition-duration:, transition-property:,
    transition-timing-function:
    },
    % http://texblog.org/tag/otherkeywords/
    otherkeywords={<, >, \/},
    ndkeywords={class, export, boolean, throw, implements, import, this, cors},
    comment=[l]{//},
    % morecomment=[s][keywordstyle]{<}{>},
    morecomment=[s]{/*}{*/},
    morecomment=[s]{<!}{>},
    morestring=[b]',
    morestring=[b]",
    alsoletter={-},
    alsodigit={:}
}

% Define language json
\lstdefinelanguage{json}{
    string=[s]{"}{"},
    stringstyle=\color{blue},
    comment=[l]{:},
    commentstyle=\color{black},
}

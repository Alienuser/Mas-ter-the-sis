\chapter{Zusammenfassung}
\label{ch:zusammenfassung}
Im Rahmen dieser Arbeit wurde ein neuronales Netz auf Basis von historischen Daten trainiert, welches der Robert Bosch
GmbH dabei hilft, die produzierten Verpackungsmaschinen schneller an den Endkunden auszuliefern, indem die
einzustellenden Formatparameter der Maschine durch präzise Vorhersagen und nicht durch langes Testen eines Mitarbeiters
ermittelt werden können.

Gerade durch die modularität der erstellten Architektur können Vorhersagen auch für weitere Maschinen erfolgen. Dabei
können Teile der Anwendung sowohl in einer beliebigen Cloud als auch im eigenen Rechenzentrum laufen ohne größeren
Entwicklungsaufwand zu generieren.

Die Basis für die Anwendung bildet das neuronale Netz sowie das darauf trainierte Modell, welches in Echtzeit Vorhersagen
tätigen kann und über mehrere Schnittstellen erreichbar ist.

Zum Start der Arbeit wurden historischen Daten gesammelt um anschließend ein neuronale Netz zu trainieren. Das daraufhin
erstellte Deployment des Modells mit eigener Schnittstelle erfolgte automatisiert aus der Cloud heraus.

Zusätzlich ist das extrahierte Modell in einem TensorFlow-Wrapper nutzbar, um es unabhängig von einer Cloud zu
betreiben. Auch kann so getestet werden, ob Vorhersagen für die gleichen Parameter in beiden Deployments übereinstimmen.

Ein Cloud Service für die Bündelung der Anfragen an die beiden Endpunkte ist in einem weiteren Schritt eingerichtet
worden. Über diesen ist es Möglich, die Anfragen der Frontends zu normieren und Benutzerdaten zum Watson Deployment zu
verschleiern.

Abschließend bleibt mir zu sagen, dass ich die in der Einleitung erwähnte Meinung von Amy Webb in so weit unterstütze,
dass mit Künstlicher Intelligenz viele neue Möglichkeiten zur Lösung von aktuellen Problemen entstehen.

Laut einer aktuellen IDC-Studie~\cite{article_zusammenfassung_idc} arbeiten und beschäftigen sich mehr als 50\% der
deutschen Unternehmen mit der Künstlichen Intelligenz allerdings bremst auch hier \enquote{Der Fachkräftemangel (\ldots)
den Fortschritt vieler Unternehmen aus.}.
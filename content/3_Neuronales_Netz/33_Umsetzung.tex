\section{Umsetzung}
Hier noch generell was beschreiben. - Zielarchitekturbild darstellen (komplette Architektur)
Was wird im folgenden gemacht.

\colorbox{yellow}{Hier fehlt was}

\subsection{Cloud}
Hier beschreiben, was ich nun in der Cloud umsetzen will und warum man die Cloud eigentlich braucht. Vorteil der
Geschwindigkeit. Dabei würde es insgesamt vier Möglichkeiten geben das umzusetzen. Für welche habe ich mich
entschieden?

- Machine Learning Models (Autoamtisch aus Daten generrieren)\\
- Model flows (SPSS) - Make Deployment\\
- Model flows (SPSS) - Download Model - Import TensorFlow/TensorFlow.js\\
- Notebooks (Python)

\url{https://console.bluemix.net/docs/cli/index.html#overview}

\colorbox{yellow}{Hier fehlt was}

\subsubsection{Daten zusammenstellen}
\colorbox{yellow}{Hier fehlt was}

\subsubsection{Daten importieren}
\colorbox{yellow}{Hier fehlt was}

\subsubsection{Daten umwandeln}
Data flows
\colorbox{yellow}{Hier fehlt was}

\subsubsection{Watson Maschine Learning einrichten}
\colorbox{yellow}{Hier fehlt was}

\subsubsection{Neuronales Netz aufbauen}
\colorbox{yellow}{Hier fehlt was}

\subsubsection{Modell trainieren}
\colorbox{yellow}{Hier fehlt was}

\subsubsection{Modell exportieren}
Tables
\colorbox{yellow}{Hier fehlt was}

\subsubsection{Deployment erstellen}
\colorbox{yellow}{Hier fehlt was}

\subsubsection{Deployment testen}
\colorbox{yellow}{Hier fehlt was}

Übersicht über die REST-Scbnittstelle:
\url{https://watson-ml-api.mybluemix.net/#!/Deployments/post_v3_wml_instances_instance_id_published_models_published_model_id_deployments_deployment_id_online}

\subsubsection{Aufruf mit Postman}
\label{subsec:Aufruf mit Postman}
Im Weiteren soll das erstellte Model, welches über ein Deployment in einem Web-Service zur Verfügung gestellt wurde,
mit Postman\footnote{https://www.getpostman.com} getestet werden.

So ist sichergestellt, dass das Model auch von extern, nicht nur über das Watson Studio Dashboard, aufrufbar ist.
Dies ist für die spätere Entwicklung des Frontends und die Einrichtung des API Connect Services wichtig. Außerdem können
so die Übergabeparameter und die Rückgabewerte der Schnittstelle überprüft werden.

Um auf die Schnittstelle des Web-Services zugreifen zu können, muss bei jedem Request an diesen ein
\textit{Authentication Token} (kurz Auth-Token) mitgeschickt werden. Dieser Token stellt sicher, dass es sich um einen
gültigen Aufruf handelt.

Der Token kann über eine REST-Schnittstelle des Watson Studio erstellt werden. Dabei handelt es sich um eine andere
Schnittstelle als beim Web-Service. Der Token ist immer nur für maximal vier Stunden gültig.

Um einen Auth-Token zu erstellen, muss der Watson Studio Benutzername und das zugehörige Passwort an die Schnittstelle
übergeben werden. Ein Beispielaufruf ist in Listing~\ref{Abruf des Auth-Tokens} auf Seite~\pageref{Abruf des Auth-Tokens}
zu sehen. Der Token ist dabei in einem JSON-Object im Rückgabewert.

\begin{lstlisting}[language=bash, caption=Abruf des Auth-Tokens, label=Abruf des Auth-Tokens]
$ curl --basic --user USERNAME:PASSWORD https://eu-gb.ml.cloud.ibm.com/v3/identity/token
\end{lstlisting}

In Postman kann dieser Aufruf durch die Eingabe der URL und dem HTTP-Type \texttt{GET} gemacht werden. Im Reiter
\texttt{Authentication} muss als Type \texttt{Basic-Auth} ausgewählt werden.

Im rechten Bereich sollten dann die beiden Eingabefelder für Benutzername und Passwort erscheinen. Nach der Eingabe der
geforderten Daten kann mit \texttt{Send} der Request abgeschickt werden und nach wenigen Sekunden sollte der Token in
einem JSON-Object als Rückgabeparameter erscheinen.

Anschließend kann ein neuer Postman-Tab geäffnet werden, welcher das eigentliche Deployment aufrufen soll. Der
REST-Endpunkt ist in Watson Studio unter Deployments zu finden und heißt \texttt{Scoring End-point}.

Dieser Scoring End-point kann als URL in Postman eingetragen und als HTTP-Type muss \texttt{POST} ausgewählt werden. Im
Bereich \texttt{Authentication} wird der Type mit \texttt{Baerer} angegeben. Somit kann im rechten Bereich nicht der
Benutzername und das Passwort angegeben werden, sondern gespeicherte Token.

Durch die Auswahl des HTTP-Types \textit{POST} kann der \texttt{Body} des Requests definiert werden. Dabei handelt es
sich um Parameter, welche an die Schnittstelle mitgeschickt werden. Für das Deployment ist es wichtig, dass die Daten
in Postman als \texttt{raw} und Type \texttt{JSON (application/json)} geschickt werden. Nun können in das Feld eigene
Testparameter eingetragen werden. Im Anhang \ref{sec:postmanTestparameter} auf Seite \pageref{sec:postmanTestparameter}
findet sich dafür ein Beispiel.

Abschließend kann mit Button \texttt{Send} der Request abgeschickt werden. Nach wenigen Sekunden erhält man den Response
des neuronalen Netzes als Rückgabeparameter angezeigt. Hier sollten auch die Vorhersagen enthalten sein.

\subsection{TensorFlow.js}
Da die Watson Studio Applikation nun erfolgreich in einem Deployment zur Verfügung gestellt und der Aufruf über
Postman verifiziert wurde, kann nun das trainierte Model in eine TensorFlow.js Applikation eingebettet werden.

Einrichten der Node.js-Umgebung auf dem PC. Dann entwicklen des ganzen. Dazu muss das trainierte Netz, welches im
vorangegangenen Kapitel erstellt wurde heruntergeladen werden. Dann in TensorFlow.js einbinden und nutzen. Also einen
Wrapper bauen. Gerade interessant für On The Edge Sachen oder für Mobile.

\colorbox{yellow}{Hier fehlt was}

\subsubsection{Modell Downloaden}
\colorbox{yellow}{Hier fehlt was}

Damit die Applikation über eine Domain aus dem Internet aus aufrufbar ist, muss diese zum Beispiel in einen Cloud
Foundry-Container, welcher in der IBM Cloud läuft, installiert werden. Im folgenden Kapitel werden die dafür nötigen
Schritte erläutert.

\subsection{Toolchain einrichten}
Da die Node.js-Applikation fertig entwickelt ist, soll diese in einem Cloud Foundry-Container installiert und über eine
Domain abrufbar gemach werden. Damit dies nicht manuell nach jeder Änderung im Quellcode über \texttt{cf push} gemacht
werden muss, kann eine Toolchain aus der IBM Cloud genutzt werden.

Für die Nutzung der Toolchain wird ein Git-Repository angelegt. Nach jedem \texttt{commit}, welcher in dieses Repository
geschrieben wird, aktiviert sich die Toolchain selbstständig und lädt den entsprechenden Commit herunter.

Anschließend werden, je nach gewählter und eingerichteter Konfiguration, verschiedene Schritte (Phasen) in der Toolchain
durchlaufen, um die Applikation in einen Cloud Foundry-Container zu installieren.

Dabei können die einzelnen Schritte, welche beim Deployment durchlaufen werden, selbst definiert, oder eine
vorkonfigurierte Toolchain genutzt werden. Die vorkonfigurierte Toolchain kann im Nachgang allerdings individuell
angepasst werden und dient lediglich einem schnelleren Start.

Für die Konfiguration der Toolchain muss die instanziierte Node.js-Runtime, in welcher die entwickelte Applikation laufen
soll, in dem IBM Cloud Dashboard ausgewählt werden.

Auf der folgenden Seite, im Tab \texttt{Übersicht} (linke Seite), erscheinen fünf Kacheln mit unterschiedlichen Informationen.
Für die Toolchain ist die Karte mit der Aufschritf \texttt{Continous Delivery} entscheidend. Dort gibt es einen Button
mit der Aufschrift \texttt{Aktivieren}.

Ein Klick auf diesen öffnet die Übersicht und eine visuelle Vorschau der Standardkonfiguration der Toolchain. Nun kann ein
Name eingetragen und die Region ausgewählt werden, in der die Toolchain installiert werden soll. Da die Standardkonfiguration
vorerst völlig ausreichend ist, kann diese direkt übernommen werden. Dafür genügt ein Klick auf \texttt{Erstellen}.

Nach einem kurzen Ladevorgang wurde die Toolchain eingerichtet und vorkonfiguriert. Es erscheinen nun vier Karten für
unterschiedliche Bereiche, in denen die IBM Cloud dem Entwicklungszyklus helfen kann.

Im Bereich \texttt{Nachdenken} wird ein Issue-Tracker konfiguriert, in dem zum Beispiel Bugs (Softwarefehler), welche
in der Software entdeckt werden, eingetragen, verwaltet und diskutiert werden können.

In \texttt{Codieren} stehen gleich zwei Kacheln zur Verfügung. Einerseits das konfigurierte Git-Repository, bei dem es
sich um ein auf IBM-Servern gehostetet GitLab handelt. Andererseits findet sich dort eine Web-IDE, auf Basis von Eclipse
Orion, mit der der Quellcode der Anwendung online editiert werden kann.

Im der letzten Kategorie, \texttt{Bereitstellen}, findet sich die Pipeline, welche im nächsten Schritt näher Erläutert
und eingerichtet wird. Ein Klick auf die Kachel mit der Aufschrift \texttt{Delivery Pipeline} öffnet diese.

Nach dem Laden der Seite erscheinen zwei sogenannte \texttt{Phasen} (engl. Stages). Jeder Schritt in der Delivery Pipeline
wird durch eine Phase symbolisiert. In einer Phase können zum Beispiel der Quellcode aus dem Git-Repository geladen, oder
die geschriebenen Tests durchgeführt werden.

Die Standardkonfiguration sieht in der \texttt{Build Stage} das Herunterladen des Quellcodes aus dem Git-Repository vor
und in der \texttt{Deploy Stage} das Einrichten eines Cloud Foundy-Containers.

Für die Node.js-Applikation reicht diese Konfiguration völlig aus, da keine zusätzlichen Installationen oder Einrichtungen
notwendig sind.

Als nächstes muss der geschriebene Quellcode der Applikation lediglich noch in das, in der Toolchain hinzugefügte,
Git-Repository eingecheckt werden. Die URL für das Repository kann in der Toolchain-Übersicht mit einem Klick auf die
Kachel \texttt{Git} angezeigt werden.

Nach erfolgreichem \texttt{push} der Anwendung in das Git-Repository startet der deployment Vorgang in der Toolchain
selbstständig. Nach wenigen Minuten sollte die Anwendung über ihre URL, welche in der Node.js Runtime definiert ist,
aufrufbar sein.

Da die Applikation nun im Internet in einem Cloud Foundry-Container zur Verfügung steht, kann diese im nächsten Schritt
mit dem API Connect Service verbunden werden. Dieser Schritt ist nötig, damit die Schnittstelle der Applikation vom
Frontend, welches in Kapitel~\ref{subsec:webseite} auf Seite~\pageref{subsec:webseite} beschrieben wird, und über Postman
vereinfacht aufgerufen werden kann.

\subsection{API Connect}
Wie in Kapitel \ref{subsec:Aufruf mit Postman} auf Seite \pageref{subsec:Aufruf mit Postman} beschrieben, benötigt das
REST-Interface des erstellten Deployments einen Auth-Token. Dieser kann nur über den Aufruf einer anderen
REST-Schnittstelle zur Verfügung gestellt werden.

Damit dies vereinfacht werden und das im weiteren Verlauf erstellte Frontend ebenfalls das Deployment aufrufen kann,
werden beide Abfragen in API Connect gebündelt.

Ein weiterer Grund für das Bündeln der beiden Anfragen an den Watson Service ist das mitschicken des Benutzernamens und
des zugehörigen Passwortes zum generrieren des Tokens. Damit diese nicht in der zu entwickelnden Anwendung hinterlegt
werden müssen, können diese Zentral in API-Connect gespeichert werden.

Das hat den Vorteil, dass sie bei Bedarf nur an einer Stelle abgeändert werden müssen. Auch werden sie nicht in einer
Anwendung hinterlegt, in der sie relativ einfach abgefangen und zu anderen Zwecken genutzt werden können.

Ein weiterer Vorteil für die Nutzung von API Connect ist die Tatsache, dass es zwei Möglichkeiten gibt, Vorhersagen zu
beziehen. Nebem dem Deployment des Watson Models kann auch die Schnittstelle der Node.js-Applikation mit TensorFlow.js
genutzt werden.

Mittels API Connect können Aufrufe optimal verteilt und Änderungen können schnell und komfortabel mit einem Online-Editor
angepasst werden.

\subsubsection{API Connect einrichten}
Da der Service, wie in Kapitel \ref{subsec:apiconnect} auf Seite \pageref{subsec:apiconnect} beschrieben, schon
eingerichtet wurde, kann dieser nun über IBM Bluemix Dashboard aufgerufen werden.

\colorbox{yellow}{Hier fehlt was}
\section{Umsetzung}
Hier noch generell was beschreiben. - Zielarchitekturbild darstellen (komplette Architektur)

\subsection{Cloud}
Wie habe ich das nun in der Cloud umgesetzt?\\
Es gibt vier Möglichkeiten:\\
- Machine Learning Models (Autoamtisch aus Daten generrieren)\\
- Model flows (SPSS) - Make Deployment\\
- Model flows (SPSS) - Download Model - Import Tensorflow/Tensorflow.JS\\
- Notebooks (Python)\\

Wenn fertig, kann man das auch mal mit cURL probieren das Deployment. Ob das auch geht!

\url{https://console.bluemix.net/docs/cli/index.html#overview}

\subsection{Tensorflow}
Einrichten der NodeJS-Umgebung auf dem PC. Dann entwicklen des ganzen. Dazu muss das trainierte Netz, welches im vorangegangenen
Kapitel erstellt wurde heruntergeladen werden. Dann in Tensorflow.JS einbinden und nutzen. Also einen Wrapper bauen.

Gerade interessant für On The Edge Sachen oder für Mobile.

\subsubsection{Toolchain einrichten}
Der einfachste Weg, seinen geschriebenen Quellcode in der Nod.JS-Runtime (Cloud Foundry-Container) zu installieren ist es,
diesen mittels DevOPS über eine Toolchain automatisiert zu installieren.

\subsection{API Connect}
Aufbauen von API Connect. Warum braucht man das? CORS-Problem. Beide Applikationen müssen abgefangen werden können. Also
auch zwei Routen. Das ist wichtig. Wie sieht das dann aus etc.
\\ \\
While this mechanism works for smaller teams and projects I’d guess that at some point you’d want API management
capabilities so that developers don’t have to have the credentials of the machine learning service and so that you can
better track the REST API invocations.
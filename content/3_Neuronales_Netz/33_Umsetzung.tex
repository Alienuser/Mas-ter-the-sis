\section{Umsetzung}
Hier noch generell was beschreiben. - Zielarchitekturbild darstellen (komplette Architektur)

\subsection{Cloud}
Wie habe ich das nun in der Cloud umgesetzt?\\
Es gibt vier Möglichkeiten:\\
- Machine Learning Models (Autoamtisch aus Daten generrieren)\\
- Model flows (SPSS) - Make Deployment\\
- Model flows (SPSS) - Download Model - Import Tensorflow/Tensorflow.JS\\
- Notebooks (Python)\\

Wenn fertig, kann man das auch mal mit cURL probieren das Deployment. Ob das auch geht!

\url{https://console.bluemix.net/docs/cli/index.html#overview}

\subsection{Tensorflow}
Einrichten der NodeJS-Umgebung auf dem PC. Dann entwicklen des ganzen. Dazu muss das trainierte Netz, welches im vorangegangenen
Kapitel erstellt wurde heruntergeladen werden. Dann in Tensorflow.JS einbinden und nutzen. Also einen Wrapper bauen.

Gerade interessant für On The Edge Sachen oder für Mobile.

\subsection{Toolchain einrichten}
Da die Node.JS-Applikation nun entwickelt ist, soll diese in einem Cloud Foundry-Container installiert und abrufbar
gemacht werden. Damit dies nicht händisch über \path{cf push} gemacht werden muss, kann eine Toolchain aus der IBM Cloud
genutzt werden.

Für die Nutzung der Toolchain wird ein Git-Repository angelegt. Nach jedem \path{commit}, welcher in dieses Repository
geladen wird, aktiviert sich die Toolchain selbstständig und lädt den aktuellsten Quellcode herunter.

Anschließend werden, je nach gewählter und eingerichteter Konfiguration, verschiedene Steps durchlaufen, um die Applikation
in einen Cloud Foundry-Container zu installieren.

Für die Konfiguration der Toolchain muss die instanziierte Node.JS-Runtime, in welcher die entwickelte Applikation laufen
soll, in der IBM Cloud ausgewählt werden.

\subsection{API Connect}
Aufbauen von API Connect. Warum braucht man das? CORS-Problem. Beide Applikationen müssen abgefangen werden können. Also
auch zwei Routen. Das ist wichtig. Wie sieht das dann aus etc.
\\ \\
While this mechanism works for smaller teams and projects I’d guess that at some point you’d want API management
capabilities so that developers don’t have to have the credentials of the machine learning service and so that you can
better track the REST API invocations.
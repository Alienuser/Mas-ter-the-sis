\section{Umsetzung}
Hier noch generell was beschreiben. - Zielarchitekturbild darstellen (komplette Architektur)

\subsection{Cloud}
Wie habe ich das nun in der Cloud umgesetzt?\\
Es gibt vier Möglichkeiten:\\
- Machine Learning Models (Autoamtisch aus Daten generrieren)\\
- Model flows (SPSS) - Make Deployment\\
- Model flows (SPSS) - Download Model - Import Tensorflow/Tensorflow.JS\\
- Notebooks (Python)\\

Wenn fertig, kann man das auch mal mit cURL probieren das Deployment. Ob das auch geht!

\url{https://console.bluemix.net/docs/cli/index.html#overview}

\subsection{Tensorflow}
Einrichten der NodeJS-Umgebung auf dem PC. Dann entwicklen des ganzen. Dazu muss das trainierte Netz, welches im vorangegangenen
Kapitel erstellt wurde heruntergeladen werden. Dann in Tensorflow.JS einbinden und nutzen. Also einen Wrapper bauen.

Gerade interessant für On The Edge Sachen oder für Mobile.

Damit die Applikation auch im Internet zur Verfügung steht und vom Frontend über das API Connect aufgerufen werden kann,
muss diese in einen Cloud Foundry-Container, welcher in der IBM Cloud laufen wird, installiert werden. Im folgenden
Kapitel werden die dafür nötigen Schritte erläutert.

\subsection{Toolchain einrichten}
Da die Node.JS-Applikation nun entwickelt ist, soll diese in einem Cloud Foundry-Container installiert und abrufbar
gemacht werden. Damit dies nicht händisch über \path{cf push} gemacht werden muss, kann eine Toolchain aus der IBM Cloud
genutzt werden.

Für die Nutzung der Toolchain wird ein Git-Repository angelegt. Nach jedem \path{commit}, welcher in dieses Repository
geladen wird, aktiviert sich die Toolchain selbstständig und lädt den entsprechenden Commit herunter.

Anschließend werden, je nach gewählter und eingerichteter Konfiguration, verschiedene Toolchain-Schritte durchlaufen, um
die Applikation in einen Cloud Foundry-Container zu installieren.

Dabei können die einzelnen Schritte selbst definiert, oder eine vorkonfigurierte Toolchain genutzt werden.

Für die Konfiguration der Toolchain muss die instanziierte Node.JS-Runtime, in welcher die entwickelte Applikation laufen
soll, in der IBM Cloud ausgewählt werden.

Auf der folgenden Seite, im Tab \path{Übersicht}, erscheinen 5 Karten mit unterschiedlichen Informationen. Für die
Toolchain ist die Karte mit der Aufschritf \path{Continous Delivery} entscheiden. Dort erscheint ein Button mit der
Aufschrift \path{Aktivieren}.

Ein klick auf diesen öffnet die Übersicht und eine visuelle Vorschau der Standardkonfiguration der Toolchain. Nun kann ein
Name eingetragen und die Region ausgewählt werden. Da die Standardkonfiguration vorerst völlig ausreichend ist, kann diese
direkt übernommen werden. Dafür genügt ein klick auf \path{Erstellen}.

Nach einem kurzen Ladevorgang wurde die Toolchain eingerichtet und vorkonfiguriert. Es erscheinen nun vier Karten für
unterschiedliche Bereiche, in denen die IBM Cloud dem Entwicklungszyklus helfen kann.

Im Bereich \path{Nachdenken}, wir dein Issue-Tracker konfiguriert, in dem Probleme mit der Anwendug eingetragen, verwaltet
und gelöst werden können.

In \path{Codieren} können auf zwei Kacheln zugegriffen werden. Einerseits auf das konfigurierte Git-Repository, bei dem
es sich um ein auf IBM-Servern gehostetet GitLab handelt. Andererseits findet sich dort eine Web-IDE, auf Basis von
Eclipse Orion.

Im der letzten Kategorie, \path{Bereitstellen}, findet sich die Toolchain, welche im nächsten Schritt Konfiguriert werden
muss. Ein klick auf die Kachel \path{Delivery Pipeling} öffnet diese.

Nach dem Laden der Seite erscheinen zwei sogenannte \path{Phasen}. Jeder Schritt in der Delivery Pipeline wird durch eine
Phase symbolisiert. In einer Phase kann zum Beispiel der Quellcode aus dem Git-Repository geladen, oder die
geschriebenen Tests können durchgeführt werden. Die Standardkonfiguration sieht in der \path{Build Stage} das herunterladen
des Quellcodes aus dem Git-Repository vor und in der \path{Deploy Stage} das Einrichten eines Cloud Foundy-Containers.

Für die Node.JS-Applikation reicht diese Konfiguration völlig aus, da keine zusätzlichen Installationen durchgeführt
werden müssen.

Da die Applikation nun im Internet über ein Cloud Foundry-Container zur Verfügung steht, kann diese im nächsten Schritt
mit dem API Connect Service verbunden werden. Dieser Schritt ist nötig, damit die Schnittstelle der Applikation vom
Frontend, welches in Kapitel~\ref{subsec:webseite} auf Seite~\pageref{subsec:webseite} beschrieben wird, aufgerufen werden
kann.

\subsection{API Connect}
Aufbauen von API Connect. Warum braucht man das? CORS-Problem. Beide Applikationen müssen abgefangen werden können. Also
auch zwei Routen. Das ist wichtig. Wie sieht das dann aus etc.
\\ \\
While this mechanism works for smaller teams and projects I’d guess that at some point you’d want API management
capabilities so that developers don’t have to have the credentials of the machine learning service and so that you can
better track the REST API invocations.
\section{Analyse}
\label{sec:analyse}
In diesem Kapitel geht es um die Analyse verschiedener Ansätze zur Umsetzung eines neuronalen Netzes und den damit
verbundenen Trainings und Tests. Dafür existieren zum jetzigen Zeitpunkt mehrere Möglichkeiten.

Im Folgenden wird auf drei gängige Varianten näher eingegangen die für die Umsetzung in Frage kommen. Dabei werden die
Vor"~ und auch die Nachteile aufgezeigt und eine Variante ausgewählt, mit der die Umsetzung der Architektur am
Sinnvolsten erscheint.

Ziel des Kapitels ist es, eine optimale Umsetzung für die Architektur zu finden, die späteren Anforderungen und
Erweiterungen gerecht werden kann.

%% TODO noch schreiben
\subsection{Offline mit Python}
Einer Möglichkeit zum Aufbau eines neuronalen Netzes ist es, dieses selbst mit Quellcode zu schreiben. Zahlreiche
Bibliotheken für die verschiedensten Programmiersprachen existieren um dieses Problem anzugehen.

Eine der am Weitest verbreiteten Programmiersprachen zum Aufbau eines solchen Netzes ist die Programmiersprache
\textit{Python}\footnote{https://www.python.org}.

Viele Vorteile sprechen für die Umsetzung des neuronalen Netzes mit Python in Verbindung mit einem der zur Verfügung
stehenden Bibliotheken.

%Gerade die Geschwindigkeit, leichtigkeit und Weiterverwendbarkeit ist super, wenn man es nicht mit Quellcode lößt.

%% TODO noch schreiben
\subsection{Online in der Cloud}
Alternativ wäre es auch Möglich das neuronale Netz in der Cloud zu trainieren. Ex existierend unzählig viele Anbieter im
Bereich künstliche Intelligenz. Auch die IBM Cloud bietet mit dem \textit{Watson Studio} ein mechanismus ein neuronales
Netz einfach aufzubauen.

%Wir könnten es in der Cloud machen. Besprechen von Bluemix, Azure und AWS. Vor und auch Nachteile der Variante.

%Azure kann keine Multi-Label Classification. Darum nur über Umwege (Selber schreiben oder Neuronale Netz an Neuronale
%Netz) machbar. Fällt also weg.

%AWS hat hier ein paar Informationen\\
%\url{https://aws.amazon.com/de/blogs/big-data/building-a-multi-class-ml-model-with-amazon-machine-learning/}.

%% TODO noch schreiben
\subsection{Hybrid}
Der Beste Weg für die Umsetzung des neuronalen Netzes wäre eine Kombination aus beiden Vorteilen zu nutzen.

%Es wird eine Mischung aus den beiden Varianten. In der Cloud wird das Neuronale Netz aufgebaut und trainier -> Geschwindigkeit.
%Dann kann es aber auch Deployed werden. Dann aber runterladen und mit TensorFlow.js einen Wrapper bauen um die gleiche
%Datei/Modul.
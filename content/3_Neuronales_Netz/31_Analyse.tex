\section{Analyse}
\label{sec:analyse}
\colorbox{yellow}{Hier fehlt was}

Es existieren mehrere Möglichkeiten dies umzusetzen. Im folgenden wird auf drei Varianten näher eingegangen. Dabei werden
die Vor- und auch die Nachteile aufgezeigt.

\subsection{Cloud}
Wir könnten es in der Cloud machen. Besprechen von Bluemix, Azure und AWS. Vor und auch Nachteile der Variante.

Azure kann keine Multi-Label Classification. Darum nur über Umwege (Selber schreiben oder Neuronale Netz an Neuronale Netz)
machbar. Fällt also weg.

AWS hat hier ein paar Informationen\\
\url{https://aws.amazon.com/de/blogs/big-data/building-a-multi-class-ml-model-with-amazon-machine-learning/}.

\colorbox{yellow}{Hier fehlt was}

\subsection{TensorFlow}
Alternativ kann das neuronale Netz auch mittels TensorFlow umgesetzt werden.
Wie könnten es aber auch eigenständig mit Tensorflow machen. Vor und auch Nachteile der Variante. Vergleich Tensorflow
und TensorFlow.js. Gerade die Geschwindigkeit, leichtigkeit und Weiterverwendbarkeit ist super, wenn man es nicht mit
Quellcode lößt.

\colorbox{yellow}{Hier fehlt was}

\subsection{Hybrid}
Es wird eine Mischung aus den beiden Varianten. In der Cloud wird das Neuronale Netz aufgebaut und trainier -> Geschwindigkeit.
Dann kann es aber auch Deployed werden. Dann aber runterladen und mit TensorFlow.js einen Wrapper bauen um die gleiche
Datei/Modul.

\colorbox{yellow}{Hier fehlt was}
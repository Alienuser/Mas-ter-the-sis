\section{Analyse}
\label{sec:analyse}
In diesem Kapitel geht es um die Analyse verschiedener Ansätze zur Umsetzung eines neuronalen Netzes und den damit
verbundenen Trainings und Tests. Dafür existieren zum jetzigen Zeitpunkt mehrere Möglichkeiten.

Im Folgenden wird auf drei gängige Varianten näher eingegangen die für die Umsetzung in Frage kommen. Dabei werden die
Vor"~ und auch die Nachteile aufgezeigt und eine Variante ausgewählt, mit der die Umsetzung der Architektur am
Sinnvolsten erscheint.

Ziel des Kapitels ist es, eine optimale Umsetzung für die Architektur zu finden, die späteren Anforderungen und
Erweiterungen gerecht werden kann.

\subsection{Offline mit Python}
Einer der eingängigsten Möglichkeit zum Aufbauen eines neuronalen Netzes ist es, dieses selbst mit Quellcode zu
definieren. Zahlreiche Bibliotheken für die verschiedensten Programmiersprachen existieren um dieses Problem anzugehen.

Eine der am weitest verbreiteten Programmiersprachen zum Aufbau eines solchen Netzes ist die Programmiersprache
\textit{Python}\footnote{https://www.python.org} mit ihren vielen Bibliotheken.

Einer der größten Vorteil dieser Variante ist die Tatsache, dass die Daten alle lokal auf dem Rechner bleiben können und
man das Netz nach belieben aufbauen und verändern kann. Auch wiederkehrende Änderungen sind möglich.

Da Python-Quellcode übersichtlich ist, gehen Anpassungen schnell von der Hand. Dies ist insbesondere dann wichtig, wenn
man die Datensätze verändert oder andere Maschinen anschließen möchte. Der Quellcode ist gut wart"~ und anpassbar.

Allerdings ist die Geschwindigkeit mit der ein neuronales Netz lokal auf einem Rechner trainiert werden kann stark durch
die zur Verfügung gestellten Hardware abhängig. Nicht alle Laptops für Entwickler verfügen über eine sehr schnelle
Hardware.

So kann das Training eines Netzes bei sehr großen Datenmengen, die es im Bereich der künstlichen Intelligenz gebens
sollte, zu langen Wartezeiten führen, bis das Modell trainiert ist.

Es ist insbesondere dann ärgerlich, wenn das Training mehrere Stunden dauert und anschließend kein vernünftiges Modell
trainiert ist. Daraufhin müssen Parameter des neuronalen Netzes angepasst werden und das Training beginnt von vorne.
Diese Schleifen können sehr viel Zeit kosten.

\subsection{Online in der Cloud}
Alternativ ist es Möglich das neuronale Netz in der Cloud zu trainieren. Es existierend unzählig viele Anbieter im
Bereich künstliche Intelligenz. So bietet Azure mit \textit{Azure Machine Learning Studio}, Amazon mit
\textit{Amazon Machine Learning} und IBM mit \textit{Watson Studio} jeweils eine eigene Lösung für das Training von
neuronalen Netzen an.

Einer der größten Vorteile der Cloud ist der schier unendlich große Zugang zu Ressourcen. Das Trainig der Netze kann in
viel kürzerer Zeit geschehen und der Entwickler muss sich nicht mit der Umsetzung von Quellcode beschäftigen oder
verschiedene Bibliotheken miteinander zu vergleichen.

Bei den meisten Cloud-Anbietern stehen sehr erfahrene Entwickler hinter den Produkten und aktualisieren diese stetig. So
kann man sich selbst auf die Aufbereitung der Daten und das erstellen von Diagrammen konzentrieren anstatt auch die
Entwicklung von Python-Quellcode zu übernehmen.

In \textit{Azure Machine Learning} ist es aktuell nicht möglich ein neuronales Netz mit multidimensionaler linearer
Regression zu erstellen. Dies bedeutet, dass es lediglich möglich ist, ein neuronales Netz mit mehreren
Eingabeparametern zu versehen, es allerdings immer nur einen Ausgabeparameter für jedes Netz gibt.

Das ist in sofern schlecht für die Umsetzung der Architektur alsdas für der Wiegeeinheit der Robert Bosch GmbH
mindestens vier Parameter vorhergesagt werden müssen.

Es wäre möglich für jeden Ausgabeparameter der vorhergesagt werden muss ein eigenes Netz aufzubauen und trainieren zu
lassen. Die verschiedenen Netze hätten dann immer die selben Trainigsdaten. Eine Anfrage an das Backend würde dann immer
aus mindestens so vielen Anfragen wie Parametern bestehen.

Diese Anfragen müssten zeitversetzt passieren, da für das zweite neuronale Netz der vorhergesagte Parameter aus dem
ersten Netz ein Eingabeparameter ist. So würde eine Anfrage sehr lange dauern.

Auch \textit{Amazon Machine Learning} kann nicht mit multidimensionalen linearen Regressionen umgehen. Auch hier wäre es
Nötig mehrere neuronale Netze aufzubauen und zu trainieren.

Eine kleine Anleitung der
AWS-Dokumentation\footnote{https://aws.amazon.com/de/blogs/big-data/building-a-multi-class-ml-model-with-amazon-machine-learning}
zeigt sehr umständlich, wie es möglich wäre ein Netz auf der Basis aufzubauen, wie es in dieser Architektur gefordert
wird. Allerdings ist diese Funktion noch in einem Beta-Status.

Mit \textit{Watson Studio} aus der IBM Cloud ist es möglich mit einem Drag \& Drop Arbeitsumfeld ein neuronales Netz
aufzubauen, welches beliebig viele Eingabe"~ und Ausgabeparameter beseitzt. Auch kann man dieses neuronale Netz beliebig
oft und schnell trainieren und das Modell anschließend sehr einfach in einen Webdienst einbauen und mit einer
Schnittstelle versehen um Anfragen zu steuern.

Allerdings muss die Cloud mit Datenschutzproblemen und diversen Vorurteilen kämpfen. So ist es nicht immer möglich die
Testdaten in die Cloud zu laden um sie zu verarbeiten. Auch muss zur Erstellung des neuronalen Netzes eine stetige
Internetverbindung bestehen. Sollte der Service eingestellt werden, so verliert man unter umständen den Zugriff auf die
Daten.

\subsection{Hybrid}
Der für diese Architektur beste Weg zur Umsetzung des neuronalen Netzes ist eine Kombination aus den Vorteilen beider
Ansätze.

So wird das neuronale Netz in der Cloud trainiert um so den Vorteil der fast grenzenlosen Geschwindigkeit zu nutzen. Das
trainierte Modell kann dann in einem Webservice zur Verfügung gestellt werden um Vorhersagen zu erhalten.

Über IBM Cloud und Watson Studio entwickelte neuronale Netze kann man die resultierenden trainierten Modelle
herunterladen und erhält so ein JSON-Datei mit dem internen Aufbau des Netzes.

Dieses Modell kann man anschließend in einen eigenen Wrapper einbauen, welcher eine Schnittstelle nach Außen öffnet. So
ist es möglich das online trainierte Modell in einer eigenen Anwendung oder im eigenen Rechenzentrum laufen zu lassen.

So ist man im Weiteren unabhängig von der Cloud und kann Vorhersagen in seinen eigenen Rechenzentren für die Maschine
erstellen lassen. Allerdings müssen die eigenen Ressourcen nicht ausgenutzt oder aufgerüstet werden um ein neuronales
Netz mit Daten trainieren zu lassen.
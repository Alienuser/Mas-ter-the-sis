\section{Vorbereitung}
Dieses Kapitel beschreibt die benötigten Vorkehrungen für die Entwicklung eines neuronalen Netzes und einer
TensorFlow.js Applikation. Dazu muss man zunächst ein kostenloses IBM Cloud Konto erstellen und darin drei Services
instanziieren. Außerdem muss man zwei Programme auf dem Entwicklungsrechner installieren.

Diese Schritte benötigt man, um mit dem Systemen arbeiten zu können. Im Weiteren werden die dafür notwendigen Schritte
einzelnd erläutert und getestet.

\subsection{IBM Cloud Konto}
Für die Arbeit mit der IBM Cloud benötigt der Anwender ein kostenloses Benutzerkonto. Dieses wird auf der zugehörigen
Registrierungsseite\footnote{https://console.bluemix.net/registration} erstellt und eingerichtet.

Nachdem das Konto erfolgreich mit einem an die hinterlegte E-Mail-Adresse verschickten Bestäigungslink aktiviert ist,
kann der Anwender dieses 30 Tage lang ohne anfallende Gebühren für Services oder Runtimes nutzen.

Nach dem ersten Aufruf des IBM Cloud Dashboards, muss man einen Namen für die automatisch erstellte Organisation
eintragen. Dieser spielt für die Umsetzung keine Rolle und die Organisation kann zu jedem Zeitpunkt umbenannt oder auch
gelöscht werden. Ein Beispiel für den Organisationsnamen ist \textit{Machine-Learning}.

Direkt im Anschluss wird nach einem Namen für den ersten Space in der erstellten Organisation gefragt. Auch dieser
spielt für die Umsetzung keine Rollte und kann zu jedem Zeitpunkt umbenannt oder auch gelöscht werden. Ein Beispiel für
die Benamung des ersten Spaces ist \textit{dev}, was eine Abkürzung für \textit{developper} ist.

Ein \textit{Space} gruppiert mehrere Runtimes und Services in einem Rechenzentrum. Von diesen Rechenzentren gibt es in
der IBM Cloud aktuell sechs, verteilt auf drei Kontinente. Eine \textit{Organisation} kann mehrere Spaces beinhalten.

\subsection{IBM Cloud CLI}
Für die einfache Verwaltung der IBM Cloud empfiehlt es sich, dass zugehörige Command Line Interface (kurz \textit{CLI})
zu installieren. Die Installation unter gängigen Linux-Distributionen und macOS erfolgt durch die Eingabe des folgenden
Kommandos in eine Shell:

\begin{lstlisting}[caption=Installation des IBM Cloud CLI, label=ls:vorbereitung_ibmcli]
    $ curl -sL http://ibm.biz/idt-installer | bash
\end{lstlisting}

Das folgende Kommando überprüft die erfolgreiche Installation auf dem System:

\begin{lstlisting}[caption=Installation des CLI überprüfen, label=ls:vorbereitung_ibmclitest]
    $ ibmcloud dev help
\end{lstlisting}

Die Ausgabe zeigt eine Übersicht über alle möglichen Befehle des \textit{ibmcloud}-Tools. Eine angezeigte Fehlermeldung
gibt Auskunft über eine fehlgeschlagene Installation. Weitere Informationen sind auf der betreffenden
Installationsseite\footnote{https://console.bluemix.net/docs/cli/reference/bluemix\_cli/get\_started.html} zu finden.

Mit der Installation des IBM Cloud-CLI wird ein symbolischer Link über das Kommando \texttt{bx} angelegt. Somit kann man
den Befehl \textit{ibmcloud} auch immer durch \textit{bx} ersetzen.

\subsection{Watson Studio}
Für den Aufbau und die Konfiguration des neuronalen Netzes wird der Service \textit{Watson Studio} benötigt. Aktuell
gibt es zwei Möglichkeiten einen Service oder eine Runtime in der IBM-Cloud in einen schon erstellten Space einzubinden.

\subsubsection{Einrichtung über das IBM Cloud Dashboard}
Mit einem Klick auf \textit{Katalog} auf dem IBM Cloud Dashboard öffnet sich dieser und listet alle Services und
Runtimes auf, die der Anwender aktuell nutzen kann. Der Service \textit{Watson Studio} befindet sich in der Kategorie
\textit{künstliche Intelligenz}. Ein Klick auf diesen öffnet die entsprechende Konfigurationsseite mit Informationen zu
Preisen und einer Dokumentation mit Anwendungsbeispielen.

Auf der Konfigurationsseite muss man den frei wählbaren \textit{Servicename} für den Service definieren. Über diesen
erscheint er später im IBM Cloud Dashboard. Anschließend muss der Entwickler die Region auswählen, in welcher der
Service zur Verfügung steht. Über die Region wird das Rechenzentrum und somit auch der Kontinent definiert, in welchem
der Service läuft.

Die Beste Konfiguration für die Architektur sieht vor, dass sich die genutzten Services und Runtimes in der gleichen
Region befinden, damit die Kommunikation ohne große Latenz abläuft.

Ebenso muss man die Ressourcengruppe und die Organisation auswählen, in die der Service gespeichert wird. Da es aktuell
aber nur eine Ressourcengruppe und eine Organisation gibt, kann die Standardeinstellung beibehalten werden.

Mit einem Klick auf \texttt{Erstellen} instanziiert man den Service und es erfolgt eine Weiterleitung zurück auf das
Dashboard. Dort erscheint der Service mit dem eingetragenen Namen in der Liste der \textit{Services}.

Über das Dashboard kann man die Übersichtsseite des Watson Studio Service durch einen Klick auf den Namen öffnen. Auf
der folgenden Seite finden sich Hinweise über die Funktionenweiße des Services. Über die Schaltfläche
\texttt{Get Started} wechselt der Nutzer dann in das Watson Studio Dashboard.

\subsubsection{Einrichtung über die CLI}
Alternativ erfolgt die Installation des Services mittels dem installierten CLI. Dazu muss man im ersten Schritt das CLI
mit dem IBM Cloud-Konto verknüpfen. Der folgende Befehl setzt den dafür notwendigen API-Endpunkt in dem Terminal:

\begin{lstlisting}[caption=Setzen des API-Endpunktes, label=ls:vorbereitung_apiendpunkt]
    $ ibmcloud api "https://api.ng.bluemix.net"
\end{lstlisting}

Anschließend erfolgt der Login über den folgenden Befehl in der selben Shell:

\begin{lstlisting}[caption=Login über CLI und Single Sign-on, label=ls:vorbereitung_login]
    $ ibmcloud login --sso
\end{lstlisting}

Der Parameter \texttt{sso} bewirkt, dass der Login über ein sich öffnendes Browserfenster erfolgt. Dies ermöglicht einen
bequemen Login ohne eingabe des Passwortes in die Kommandozeile.

Nach einem erfolgreichen Login, muss der Nutzer die Organisation auswählen, welche er nutzen möchte. Dies geschieht über
die Eingabe der entsprechenden Nummer der Organisation, welche am linken Rand zu sehen ist. Der Login-Vogang ist damit
abgeschlossen.

Der Benutzer kann nun eine neue Instanz des \textit{Watson Studio}-Service mit dem folgenden Befehl erstellen:

\begin{lstlisting}[caption=Instanziierung des Watson Studio Services, label=ls:vorbereitung_watsonservice]
    $ ibmcloud cf create-service Watson-Studio lite sercive_name
\end{lstlisting}

Für den Parameter \texttt{service\_name} muss man einen Name eingetragen, unter welchem der Service aufrufbar ist.
Dieser Name muss für die komplette IBM Cloud Region einzigartig sein.

Über den Befehl \texttt{cf services} kann man alle instanziierten Services, welche sich in der vorher ausgewählten
Organisation befinden, auflisten.

\begin{lstlisting}[caption=Auflisten aller Services in einer Organisation, label=ls:vorbereitung_alleservices]
    $ ibmcloud cf services
\end{lstlisting}

Der vorher erstellte Service sollte nun mit dem selbst definierten Namen in der Liste der Services erscheinen.

\subsubsection{Projekt einrichten}
Abschließend muss man ein Projekt im Watson Studio Service anlegen. Dieses bündelt alle Daten und Informationen an einem
gemeinsamen Ort um sie besser Verwalten zu können. Um ein Projekt zu erstellen, muss man das Watson Studio Dashboard
über das IBM Cloud Dashboard aufrufen.

Die Schlatfläche \texttt{New project} legt im Watson Studio Dashboard ein neues Projekt an. Auch hier muss man einen
Namen für das Projekt definieren und anschließend einen \textit{Local Storage} einrichten.

Die Einrichtung des Local Storages geschieht über einen Mini-Wizzard auf der rechten Seite. Dabei muss man lediglich
einen Namen für diesen eintragen und dies dann bestätigen.

Das Projekt kann man über die Schaltfläche \texttt{Create} speichern und es wird für den Nutzer feftig eingerichtet.

Ein angelegtes Projekt kann man mit mehreren Personen, welche allerdings über ein IBM Cloud Konto verfügen müssen,
teilen. So haben alle Teammitglieder zugriff auf alle dort gespeicherten Informationen und Daten. Auch ein
kollaboratives Arbeiten ist so möglich.

\subsection{Node.js Runtime}
\label{ssc:nodejs_runtime}
Für die Erstellung der Node.js-Applikation benötigt man eine entsprechende Runtime in der IBM Cloud. Für die Runtime
nutzt man am Einfachsten einen vorkonfigurierten Cloud Foundry Container mit Node.js-Boilerplate. Für die Erstellung
des Containers existieren ebenfalls zwei Möglichkeiten.

Im vorangegangenen Kapitel wurden beiden Möglichkeiten (über den IBM Cloud Katalog und über die Kommandozeile)
ausführlich erläutert und aufgezeigt.

Über das folgende Kommando der CLI erstellt man die benötigte Runtime in einem Container über ein Terminal:

\begin{lstlisting}[caption=Instanziierung der Node.js Runtime, label=ls:vorbereitung_nodejstensorflow]
    $ ibmcloud cf create-service nodejs service_name
\end{lstlisting}

Auch hier wird über den Parameter \texttt{service\_name} der Name für die Applikation vergeben. Die URL, über die der
Container später aufrufbar ist, folgt dem Schema \texttt{https://service\_name.mybluemix.net}.

Dies bedeutet, dass der Name für die Applikation nur ein Mal im kompletten System vergeben sein darf. Eine Meldung gibt
Hinweise darüber, ob die Applikation erfolgreich erstellt werden konnte, oder ob zum Beispiel ein anderer Name für die
Applikation verwendet werden muss.

Sofern keine Fehlermeldung erscheint ist die Einrichtung der Node.js-Runtime erfolgreich verlaufen und ein Aufruf der
erstellen URL zeigt eine Standardseite der IBM Cloud mit einem kleinen Logo.

\subsection{API Connect}
\label{subsec:vorbereitung_apiconnect}
Um im Weiteren Verlauf den API-Gateway \textit{API Connect} zu nutzen, muss man diesen ebenfalls in der IBM Cloud
instanziieren. Wie auch in den beiden vorangegangenen Kapiteln kann man die Instanziierung sowohl über die CLI als auch
über den IBM Cloud Katalog einleiten.

Einfachheitshalber wird im Folgenden die Instanziierung des Services mittels IBM Cloud CLI aufgezeigt:

\begin{lstlisting}[caption=Instanziierung von API Connect, label=ls:vorbereitung_apiconnect]
$ ibmcloud cf create-service API-Connect lite service_name
\end{lstlisting}

Über den Parameter \texttt{service\_name} definiert man den Name für den Service. Über diesen ist er im IBM Cloud
Dashboard auffindbar.

\subsection{Git}
Für die Verwaltung und den späteren, automatisierten Installationsvorgang des geschriebenen Quellcodes wird ein
Repository auf Basis von Git verwendet. Unter Linux erfolgt die Installation des Programmes über das folgende Kommando:

\begin{lstlisting}[caption=Installation von Git, label=ls:vorbereitung_git]
    $ sudo apt-get install git
\end{lstlisting}

Unter macOS steht ein grafisches Installationsprogramm\footnote{https://sourceforge.net/projects/git-osx-installer} zur
Verfügung.

\subsection{Node.js und npm}
Für die Entwicklung und die damit verbundenen Tests und Probeläufe auf dem Entwicklungsrechner benötigt man ein
installierte Node.js Instanz. Die Installation der zur Zeit aktuellsten LTS-Version (10.13.0) erfolgt unter Linux über
die folgenden Kommandos:

\begin{lstlisting}[caption=Installation von Node.js und npm, label=ls:vorbereitung_nodejs]
    $ curl -L https://deb.nodesource.com/setup_10.x | sudo -E bash -
    $ sudo apt-get install -y nodejs
\end{lstlisting}

Für macOS existiert ein Installationspaket, welches über die Download-Seite\footnote{https://nodejs.org/dist/v10.13.0/node-v10.13.0.pkg}
zur Verfügung steht.

Bei der Installation von Node.js wird npm automatisch mitinstalliert. Dieser ist der Node.js eigene Paketmanager und
man benötigt diesen später für die Installation von Abhängigkeiten für die Applikation.

Die erfolgreiche Installation der beiden Anwendungen kann man nun über die Eingabe der jeweiligen Programmnamen in eine
Shell verifizieren.

Über die Eingabe von \texttt{node} startet man die entsprechende Runtime. Über \texttt{npm} öffnet man eine Übersicht
über die Funktionen des Paketmanagers.
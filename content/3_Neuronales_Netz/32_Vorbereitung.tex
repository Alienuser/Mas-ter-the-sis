\section{Vorbereitung}
In diesem Kapitel werden Vorkehrungen für die Entwicklung eines Neuronalen Netzes und der Tensorflow-Applikation
getroffen. Dafür muss ein kostenloses IBM Cloud Konto erstellt, ein Programm auf dem Entwicklungsrechner installiert und
zwei Services eingerichtet werden.

\subsection{Bluemix Konto}
Um mit der IBM Cloud arbeiten zu können wird ein kostenloses Konto benötigt. Dies kann auf der
Registrieruns-Seite\footnote{https://console.bluemix.net/registration} erstellt werden.

Nach erfolgreicher aktivierung des Kontos, mit einem an die hinterlegte E"=Mail"=Adresse verschickten Bestätigungslink,
kann das Konto 30 Tage lang ohne anfallende Gebühren für Services oder Runtimes genutzt werden.

Beim erstmaligen Aufruf des IBM Cloud Dashboards, wird nach einem Namen für die automatisch erstellte Organisation gefragt.
Dieser spielt für die Umsetzung keine Rolle und die Organisation kann zu jedem Zeitpunkt umbenannt oder auch gelöscht werden.
Ein Beispiel für den Organisationsnamen ist \path{Machine-Learning}.

In einer Organisation werden Runtimes uns Services gruppiert. Eine Organisation kann mehrere \path{Spaces} beinhalten.

\subsection{IBM Cloud CLI}
Für die einfache Verwaltung der IBM Cloud empfiehlt es sich das zugehörige Command Line Interface (kurz CLI) zu installieren.
Die Installation unter Linux und macOS erfolgt durch die Eingabe des folgenden Kommandos:

\begin{lstlisting}[language=bash, caption=Installation des IBM Cloud CLI, label=Installation des IBM Cloud CLI]
    $ curl -sL http://ibm.biz/idt-installer | bash
\end{lstlisting}

Anschließend kann die erfolgreiche Installation über das folgende Kommando überprüft werden:

\begin{lstlisting}[language=bash, caption=Installation des CLI überprüfen, label=Installation des CLI überprüfen]
    $ ibmcloud dev help
\end{lstlisting}

Die Ausgabe sollte eine Übersicht über alle möglichen Befehle des \textit{ibmcloud}-Tools geben.

\subsection{Watson Studio}
Für die Erarbeitung des Neuronalen Netzes wird der Service \path{Watson Studio} benötigt. Es gibt zwei Möglichkeiten diesen
in die erstellte Organisation einzubinden.

\subsubsection*{Über das IBM Cloud Dashboard}
Auf dem IBM Cloud Dashbaord können mit einem Klick auf \path{Katalog} alle Services und Runtimes, welche aktuell genutzt
werden können, aufgelistet werden. In der Kategorie \path{K\"Unstliche Intelligenz} kann der Service \path{Watson Studio}
ausgewählt werden.

Auf der sich öffnenden Seite kann der \path{Servicename}, welcher frei gewählt werden kann, eingetragen werden. Anschließend
kann noch die Region, in welcher der Service zur verfügung gestellt werden soll soll und die Ressourengruppe definiert werden.

Mit einem Klick auf \path{Erstellen}, wird der Service Instanziiert und es erfolgt eine Weiterleitung auf das Dashbaord
zurück. Dort sollte der Service mit dem definierten Name sichtbar sein.

Auf dem Dashboard kann der Watson Studio Service nun angeklickt werden. Auf der folgenden Seite kann mit \path{Get Startet}
in das Watson Studio gewechselt werden.

\subsubsection*{Über die CLI}
Alternatic kann der Service auch mittels dem installierten CLI eingerichtet werden. Dazu muss das CLI mit dem IBM
Cloud-Konto verknüpft werden. Mittels dem folgenden Befehl wird die api gesetzt:

\begin{lstlisting}[language=bash, caption=Setzen des API Targets, label=Setzen des API Targets]
$  ibmcloud api https://api.ng.bluemix.net
\end{lstlisting}

Anschließend kann der Benutzer sich mit dem folgenden Befehl einloggen:

\begin{lstlisting}[language=bash, caption=Login über CLI und SSO, label=Login über CLI und SSO]
$   ibmcloud login --sso
\end{lstlisting}

Der Parameter \path{sso} bewirkt, dass sich ein Browserfenster öffnet, das einen bequemen Login ohne Kommandozeile
ermöglicht.

Nach einem erfolgreichen login, kann die genutzte Organisation ausgewählt werden. Der Login-Vogang ist damit abgeschlossen.

Im weiteren kann ein \path{Watson Studio}-Service mit dem folgenden Befehl erstellt werden:

\begin{lstlisting}[language=bash, caption=Instanziierten des Watson Studio Services, label=Instanziierten des Watson Studio Services]
$   ibmcloud cf create-service Watson-Studio lite sercive_name
\end{lstlisting}

Für den Parameter \path{service_name} muss ein Name eingegeben werden, unter welchem der Service gefunden werden kann.
Über den Befehl \path{cf services} können alle instantiierten Services aufgelistet werden.

\begin{lstlisting}[language=bash, caption=Auflisten aller Services, label=Auflisten aller Services]
$   ibmcloud cf services
\end{lstlisting}

\subsection{Notebook für Tensorflow}
Für die Entwicklung des neuronalen Netzes mittels Tensorflow wird ein Notebook eingerichtet, welches un Anschluss gestartet
werden kann. Für die Einrichtung muss der \path{Watson Studio}-Service gestartet werden. Dazu wird im IBM Cloud Dashboard
der Service ausgewählt und im folgenden mit \path{Get Started} in den Service gewechselt.

Nun muss über \path{New Project} ein neues Projekt angelegt werden. Das neue Projekt wird als \path{Basic} angelegt und
die Benamung spielt keine Rolle. Gegebenenfalls muss noch ein Cloud Object Storage angelegt werden.

Mit \path{Create} werden die Einstellungen gespeichert und das neue Projekt erfolgreich angelegt.

Nun kann das neue Projekt geöffnet und im Reiter \path{Assets} in der Kategorie \path{Notebooks} auf \path{New Notebook}
geklickt werden. Es öffnet sich eine Eingabemaske für den Namen und die Beschreibung. Auch hier spielen beide keine Rolle.

Als Runtime wird \path{Default Python 3.5 Free} ausgewählt, damit keine Kosten entstehen. Die 1vCPU und die 4GB RAM
reichen für das kleine neuronale Netz vollkommen aus.

Ein klick auf \path{Create Notebook} bestätigt die getätigen Einstellungen und es erfolgt eine Weiterleitung auf das Watson
Studio Projekt.

Nun kann das erstellte Notebook über ein Klick geöffnet werden. Nach wenigen Augenblicken öffnet sich eine Eingabemaske.

Das Notebook kann über den das folgende Beispiel getestet werden. Nach dem Start sollte der entsprechende Satz unter dem
Codebeispiel und keine Fehlermeldungen erscheinen.

\begin{lstlisting}[language=Python, caption=Hello World für Notebook, label=Hello World für Notebook]
# Import Tensorflow
import tensorflow as tf

# Create TensorFlow object and call it hello
hello = tf.constant('Hello World!')

# Run the tf.constant operation in the session
with tf.Session() as sess:
    output = sess.run(hello)
    print(output)
\end{lstlisting}
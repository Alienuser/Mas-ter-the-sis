\section{Vorbereitung}
In diesem Kapitel werden Vorkehrungen für die Entwicklung des Neuronalen Netzes getroffen. Dabei muss ein kostenloses IBM
Cloud Konto erstellt, ein Programm auf dem Entwicklungsrechner installiert und zwei Services eingerichtet werden.

\subsection{Bluemix Konto}
Um mit der IBM Cloud arbeiten zu können wird ein kostenloses Konto benötigt. Dies kann auf der
Registrieruns-Seite\footnote{https://console.bluemix.net/registration} erstellt werden.

Nach erfolgreicher aktivierung des Kontos, mit einem an die hinterlegte E"=Mail"=Adresse verschickten Bestätigungslink,
kann das Konto 30 Tage lang ohne anfallende Gebühren für Services oder Runtimes genutzt werden.

Beim erstmaligen Aufruf des IBM Cloud Dashboards, wird nach einem Namen für die automatisch erstellte Organisation gefragt.
Dieser spielt für die Umsetzung keine Rolle und die Organisation kann zu jedem Zeitpunkt umbenannt oder auch gelöscht werden.
Ein Beispiel für den Organisationsnamen ist \path{Machine-Learning}.

\subsection{IBM Cloud CLI}
Für die einfache Verwaltung der IBM Cloud empfiehlt es sich das zugehörige Command Line Interface (kurz CLI) zu installieren.
Die Installation unter Linux und MacOS erfolgt durch die Eingabe des folgenden Kommandos:

\begin{lstlisting}[language=bash, caption=Installation des IBM Cloud CLI, label=Installation des IBM Cloud CLI]
    $ curl -sL http://ibm.biz/idt-installer | bash
\end{lstlisting}

Anschließend kann die erfolgreiche Installation über das folgende Kommando überprüft werden:

\begin{lstlisting}[language=bash, caption=Installation des CLI überprüfen, label=Installation des CLI überprüfen]
    $ ibmcloud dev help
\end{lstlisting}

Die Ausgabe sollte eine Übersicht über alle möglichen Befehle des \textit{ibmcloud}-Tools geben.

\subsection{Tensorflow Runtime}
Einrichten der Runtime für Tensorflow

\subsection{Watson Studio}
Einrichten von Watson Studio
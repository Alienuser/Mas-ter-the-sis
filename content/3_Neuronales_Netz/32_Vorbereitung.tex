\section{Vorbereitung}
In diesem Kapitel werden Vorkehrungen für die Entwicklung eines neuronalen Netzes und einer Tensorflow-Applikation
getroffen. Dafür muss zunächst ein kostenloses IBM Cloud Konto erstellt und drei Services eingerichtet werden.
Desweiteren werden zwei Programme auf dem Entwicklungsrechner installiert, die zur Entwicklung benötigt werden.

Im Weiteren werden die dafür notwendigen Schritte einzelnd erläutert.

\subsection{Bluemix Konto}
Um mit der IBM Cloud arbeiten zu können wird ein kostenloses Konto benötigt. Dies kann auf der
Registrierungsseite\footnote{https://console.bluemix.net/registration} erstellt werden.

Nach erfolgreicher Aktivierung des Kontos, mit einem an die hinterlegte E"=Mail"=Adresse verschickten Bestätigungslink,
kann das Konto 30 Tage lang ohne anfallende Gebühren für Services oder Runtimes genutzt werden.

Beim erstmaligen Aufruf des IBM Cloud Dashboards, wird nach einem Namen für die automatisch erstellte Organisation
gefragt. Dieser spielt für die Umsetzung keine Rolle und die Organisation kann zu jedem Zeitpunkt umbenannt oder auch
gelöscht werden. Ein Beispiel für den Organisationsnamen ist \texttt{Machine-Learning}.

Im Anschluss wird nach einem Namen für den ersten \texttt{Space} in der erstellten Organisation gefragt. Auch dieser
spielt für die Umsetzung keine Rollte und kann zu jedem Zeitpunkt umbenannt oder auch gelöscht werden. Ein Beispiel für
die Benamung des ersten Spaces ist \texttt{dev}, was eine Abkürzung für \textit{developper} ist.

In einem \texttt{Space} werden mehrere Runtimes und Services in einem Rechenzentrum gruppiert. Eine \texttt{Organisation}
kann mehrere Spaces beinhalten.

\subsection{IBM Cloud CLI}
Für die einfache Verwaltung der IBM Cloud empfiehlt sich, dass zugehörige Command Line Interface (kurz \textit{CLI}) zu
installieren. Die Installation unter Linux und macOS erfolgt durch die Eingabe des folgenden Kommandos in eine Shell:

\begin{lstlisting}[language=bash, caption=Installation des IBM Cloud CLI, label=Installation des IBM Cloud CLI]
    $ curl -sL http://ibm.biz/idt-installer | bash
\end{lstlisting}

Anschließend kann die erfolgreiche Installation über das folgende Kommando überprüft werden:

\begin{lstlisting}[language=bash, caption=Installation des CLI überprüfen, label=Installation des CLI überprüfen]
    $ ibmcloud dev help
\end{lstlisting}

Die Ausgabe sollte eine Übersicht über alle möglichen Befehle des \textit{ibmcloud}-Tools geben. Sollte eine Fehlermeldung
erscheinen, hat die Installation nicht geklappt. Weitere Informationen gibt es auf der betreffenden
Installationsseite\footnote{https://console.bluemix.net/docs/cli/reference/bluemix\_cli/get\_started.html}.

Mit der Installation des IBM Cloud-CLI wird ein symbolischer Link über das Kommando \texttt{bx} angelegt. Somit kann der
Befehl \textit{ibmcloud} auch immer durch \textit{bx} ersetzt werden.

\subsection{Watson Studio}
Für den Aufbau und die Konfiguration des neuronalen Netzes wird der Service \texttt{Watson Studio} benötigt. Aktuell gibt
es zwei Möglichkeiten einen Service oder eine Runtime in einen erstellten Space einzubinden.

\subsubsection*{Über das IBM Cloud Dashboard}
Auf dem IBM Cloud Dashbaord können mit einem Klick auf \texttt{Katalog} alle Services und Runtimes, welche aktuell genutzt
werden können, aufgelistet werden. In der Kategorie \texttt{Künstliche Intelligenz} kann der Service \texttt{Watson Studio}
ausgewählt werden.

Auf der sich öffnenden Seite kann der \textit{Servicename}, welcher frei gewählt werden kann, eingetragen werden.
Anschließend kann noch die Region, in welcher der Service zur Verfügung gestellt werden soll und die Ressourengruppe
definiert werden.

Mit einem Klick auf \texttt{Erstellen}, wird der Service Instanziiert und es erfolgt eine Weiterleitung zurück auf das
Dashbaord. Dort erscheint der Service mit dem eingetragenen Name.

Auf dem Dashboard kann der Watson Studio Service durch einen Klick ausgewählt werden. Auf der folgenden Seite über die
Schaltfläche \texttt{Get Started} in das Watson Studio gewechselt werden.

\subsubsection*{Über die CLI}
Alternativ erfolgt die Installation des Services mittels dem installierten CLI. Dazu muss das CLI mit dem IBM
Cloud-Konto verknüpft werden. Mittels dem folgenden Befehl wird der dafür nötige API-Endpunkt gesetzt:

\begin{lstlisting}[language=bash, caption=Setzen des API Targets, label=Setzen des API Targets]
    $ ibmcloud api https://api.ng.bluemix.net
\end{lstlisting}

Anschließend kann sich der Benutzer mit dem folgenden Befehl einloggen:

\begin{lstlisting}[language=bash, caption=Login über CLI und Single Sign-on, label=Login über CLI und SSO]
    $ ibmcloud login --sso
\end{lstlisting}

Der Parameter \texttt{sso} bewirkt, dass sich ein Browserfenster öffnet, das einen bequemen Login ohne Kommandozeile
ermöglicht.

Nach einem erfolgreichen Login, kann die genutzte Organisation ausgewählt werden. Dies geschieht über die Eingabe der
entsprechenden Nummer der Organisation. Der Login-Vogang ist damit abgeschlossen.

Im Weiteren kann ein \texttt{Watson Studio}-Service mit dem folgenden Befehl erstellt werden:

\begin{lstlisting}[language=bash, caption=Instanziierung des Watson Studio Services, label=Instanziierung des Watson Studio Services]
    $ ibmcloud cf create-service Watson-Studio lite sercive_name
\end{lstlisting}

Für den Parameter \texttt{service\_name} muss ein Name eingetragen werden, unter welchem der Service gefunden werden kann.
Über den Befehl \texttt{cf services} können alle instantiierten Services, welche sich in der vorher ausgewählten
Organisation befinden, aufgelistet werden.

\begin{lstlisting}[language=bash, caption=Auflisten aller Services, label=Auflisten aller Services]
    $ ibmcloud cf services
\end{lstlisting}

Der erstellte Service sollte mit dem eingegebenen Namen in der Liste aller Services erscheinen.

Abschließend wird im Watson Studio ein Projekt angelegt Dieses sammelt alle Daten und Informationen und kann im Watson
Studio Dashboard erstellt werden. Dazu wird das Watson Studio Dashboard über das IBM Bluemix Dashboard aufgerufen.

Im Watson Studio Dashboard kann über den Menüpunkt \texttt{New project} ein solches angelegt werden. Nach der Eingabe eines
Namens und der Einrichtung eines Local Storages (dies geschicht über den Mini-Wizzard) kann das Projekt über den Button
\texttt{Create} angelegt werden.

\subsection{Node.js Runtime}
\label{ssc:nodejs_runtime}
Für die Erstellung der Node.js-Applikation wird eine entsprechende Runtime in der IBM Cloud benötigt. Dafür wird ein
Cloud Foundry-Container mit Node.js-"-Konfiguration erstellt. Dieser kann ebenfalls auf zwei unterschiedliche Weisen erstellt
werden.

Im vorangegangenen Kapitel wurden die zwei Varianten (über den IBM Cloud Katalog und über die Kommandozeile) ausführlich
erläutert. Im Folgenden wird lediglich die Erstellung über das CLI aufgezeigt:

\begin{lstlisting}[language=bash, caption=Instanziierung der Node.js Runtime, label=Instanziierung der Node.JS Runtime]
    $ ibmcloud cf create-service nodejs service_name
\end{lstlisting}

Auch hier muss über den Parameter \texttt{service\_name} ein Name für die Applikation vergeben werden. Die URL, über die
der Container später aufrufbar ist folgt dem Schema \texttt{https://service\_name.mybluemix.net}.

\subsection{API Connect}
\label{subsec:apiconnect}
Um im weiteren Verlauf den API Gateway \texttt{API Connect} nutzen zu können, muss dieser ebenfalls in Bluemix instanziiert
werden. Wie auch in den beiden vorangegangenen Kapiteln kann dies über die CLI als auch über den Bluemix Katalog geschehen.

Einfachheitshalber wird im folgenden die Instanziierung des Services mittels Bluemix CLI aufgezeigt.

\begin{lstlisting}[language=bash, caption=Instanziierung von API Connect, label=Instanziierung von API Connect]
$ ibmcloud cf create-service API-Connect lite service_name
\end{lstlisting}

Als Parameter \texttt{service\_name} muss ein Name für den Service vergeben werden, mit dem er im Bluemix Dashboard angezeigt
werden soll.

\subsection{Git}
Für die Verwaltung und den späteren, automatisierten Installationsvorgang des geschriebenen Quellcodes wird Git verwendet.
Das Programm kann unter Linux über das Folgenden Kommando auf dem Entwicklnugsrechner installiert werden:

\begin{lstlisting}[language=bash, caption=Installation von Git, label=Installation von Git]
    $ sudo apt-get install git
\end{lstlisting}

Unter macOS steht ein Grafische Installationsprogramm\footnote{http://sourceforge.net/projects/git-osx-installer} zur
Verfügung.

\subsection{Node.js und npm}
Für die Entwicklung und die damit verbundenen Tests und Probeläufe auf dem Entwicklungsrechner wird ein installiertes
Node.js benötigt. Die zur Zeit aktuellste LTS-Version (8.11.4) kann unter Linux wie folgt installiert werden:

\begin{lstlisting}[language=bash, caption=Installation von Node.js, label=Installation von Node.js]
    $ curl -sL https://deb.nodesource.com/setup_8.x | sudo -E bash -
    $ sudo apt-get install -y nodejs
\end{lstlisting}

Für macOS gibt es ein Installationspaket, welches auf der Download-Seite\footnote{https://nodejs.org/dist/v8.11.4/node-v8.11.4.pkg}
heruntergeladen werden kann.

Bei der Installation von Node.js wird npm automatisch mitinstalliert. Dieser ist der Node.js eigene Paketmanager und wird
für die spätere Installation von Abhängigkeiten für die Applikation benötigt.
\chapter{Einleitung}
\label{ch:einleitung}

\section{Motivation}
\label{sec:motivation}
- Warum habe ich das gemacht?
- Was hat Bosch davon?
- Was ändert sich am Markt?
\\ \\
- Machine Learning ein immer Wichtigerer Aspekt im Leben
- Andere Firmen haben viel damit gemacht
- Maschinen schneller und besser einstellen können
- Maschinen Reibungsloser einstellen
- Wiederkehrende Probleme sehen
- Maschinen automatisch einstellen
\\ \\
- Hypothese, dass ich das Problem mit Machine Learning lösen will
\\ \\
2 Seiten

\section{Aufgabenstellung}
\label{sec:aufgabenstellung}
- Was soll in der Arbeit umgesetzt werden?
- Grob beschreiben was alles durchgeführt wird?
- Grob beschreiben, was rauskommen soll.
- Was ist das Ziel des ganzen
- Was kommt nachher raus
\\ \\
- Ich habe in der Cloud ein Machine Learning Netzwerk aufgebaut
- Machine Learning Netzwerk mit Tensorflow
- Webseite (Client) zum eingeben und auslesen der Daten
- REST-Schnittstelle zur Kommunikation
- Smartphone-Apps gebaut
- Bild mit kompletter Architektur
\\ \\
1 Seite

\newpage

\section{Aufbau der Arbeit}
\label{sec:aufbauDerArbeit}
Dieses Kapitel soll zur schnelleren Orientierung innerhalb der Arbeit dienen und zeigt, welche Themen in den jeweiligen
Kapiteln angesprochen werden.

\begin{description}

    \item[Kapitel 2 (Grundlagen)]\hfill \\
    Hier kommt der Inhalt

    \item[Kapitel 3 (Neuronales Netz)]\hfill \\
    Hier kommt der Inhalt

    \item[Kapitel 4 (Client)]\hfill \\
    Hier kommt der Inhalt

    \item[Kapitel 5 (Ausblick)]\hfill \\
    Hier kommt der Inhalt

    \item[Kapitel 6 (Zusammenfassung)]\hfill \\
    Hier kommt der Inhalt

\end{description}
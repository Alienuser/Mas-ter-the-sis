\chapter{Einleitung}
\label{ch:einleitung}

\section{Motivation}
\label{sec:motivation}
Wenn man der Antwort von Amy Webb, ihres Zeichens Gründerin des Future Today Institute, auf die Frage \enquote{Was ist
the next big tech-thing} Glauben schenken darf, \enquote{(\ldots) (ist) Künstliche Intelligenz die wichtigste technische
Entwicklung. Wir läuten damit die dritte Ära der Informatik ein. KI wird sich künftig überall
finden.}~\cite{article_einleitung_dub_aw}.

Die chinesische Regierung etwa investiert hunderte Milliarden Dollar bis zum Jahr 2030 in Themen wie Künstliche
Intelligenz und Deep Learning um ihren Fortschritt gegenüber anderen Ländern ausbauen~\cite{article_einleitung_css}.
Hingegen möchte die Europäische Union eine \enquote{(\ldots) dagegen fast schon kleinmütig erscheinende Summe von 1,5
Milliarden Euro (bis 2020 investieren).}~\cite{article_einleitung_ww_sg}.

Eine aktuelle Statistik von Statista\footnote{https://de.statista.com/infografik/14245/prognostizierter-umsatz-mit-ki-anwendungen-weltweit}
zeigt, dass der Umsatz durch Künstliche Intelligenz bis zum Jahr 2025 um das bis zu zwölffache des aktuellen Wertes
ansteigen wird und sich die größten Anteile auf Nordamerika und Asien verteilen werden.

Aber sind Themen wie Künstliche Intelligenz und die damit einhergehenden neuronalen Netze wirklich neu? Laut Sigmar
Gabriel und seinem veröffentlichten Artikel in der WirtschaftsWoche handelt es sich um ein \enquote{(\ldots) altes
Forschungsgebiet in dem seit den Fünfzigerjahren (geforscht wird)}~\cite{article_einleitung_ww_sg} und auch den
Aufzeichnungen von Google-Trends zufolge~\cite{online_einleitung_googletrends} ist gerade das Thema Künstliche
Intelligenz schon seit dem Jahr 2004 ein gesuchtes Thema und weckt bei vielen Menschen Interesse.

Nach anschließend stetig sinkenden Suchanfragen stieg das Interesse an der KI ab Mitte 2016 wieder stark an, sodass es
aktuell in aller Munde ist. Die Themen Machine Learning und Big Data allerdings wurden erst ab Anfang 2012 für die
breite Masse interessant und bilden meist die Ecksäulen der Künstlichen Intelligenz.

Laut Sigmar Gabriel sollen deutsche Unternehmen vorallem \enquote{(\ldots) neue Produkte und Verfahren entwickeln, die
das Vorhandene (verbessern).} und sich nicht \enquote{(\ldots) an den Geschäftsmodellen von Google, Amazon und
Co.}~\cite{article_einleitung_ww_sg} orientieren.

\enquote{Intelligenz ist die Fähigkeit, sich dem Wandel anzupassen.} sagte Stephen Hawking auf einer Pressekonferenz.
Aktuell zeigen zahlreiche Startups sowie amerikanische Großkonzerne, was mit Künstlicher Intelligenz alles möglich ist.
Diesen wachsenden Trend beziehungsweise diesen Wandel in der Entwicklung von Produkten dürfen hiesige Firmen nicht
verschlafen, indem sie Risiken vor die entstehenden Möglichkeiten oder die neu entstehenden Geschäftsbereiche stellen.

Denn laut Amy Webb können \enquote{Die Folgen (wenn ein Unternehmen nicht auf Künstliche Intelligenz reagiert)
verheerend sein (\ldots)} und da \enquote{(\ldots) jeder kommerzielle Sektor betroffen sein (wird) (\ldots)}, sollten
sich immer mehr Firmen näher mit dem Thema beschäftigen~\cite{article_einleitung_dub_aw}.

Einer McKinsey-Studie von 2017 zufolge könnte die deutsche Wirtschaftsleistung durch den Einsatz von intelligenten
Robotern und selbstlernenden Computern um 160 Milliarden Euro höher sein als heute und das nicht zwangsweise mit weniger
Beschäftigten~\cite{online_einleitung_mckinsey}. Allerdings gibt eine IDC-Studie vom Juni 2017 an, dass rund 40 Prozent
der befragten Unternehmen in Europa und den USA planen, bis Ende 2018 aktiv von KI-Technologien zu profitieren und diese
einzusetzen~\cite{article_grundlagen_salesforce}.

Um in diesem Wettstreit nicht abgehängt zu werden, sind Themen wie Künstliche Intelligenz und neuronale Netze langsam
auch für viele produzierende Firmen aus Deutschland interessant, die ihre Produkte mit Intelligenz auszustatten möchten,
um sie besser oder auch schneller zu machen.

Auch die Robert Bosch GmbH mit seiner Robert Bosch Packaging Technology Gesellschaft hat starkes Interesse daran, die
von ihnen produzierten Verpackungsmaschinen effizienter arbeiten zu lassen oder sie zum Beispiel schneller an den Kunden
auszuliefern, damit dieser sie eher nutzen kann.

Die Ursachenanalyse (englisch root cause analysis) von Don Norman aus seinem Buch \textit{The Design of Everyday
Things}~\cite{book_einleitung_donnorman} besagt, dass die Ursache des menschlichen Handelns zum Lösen von Problemen
durch beantwortung der fünf \textit{Warum}-Fragen geschehen muss. Durch eine Kettenreaktion sind dann auch alle anderen
Probleme gelöst.

Die fünfmalige Beantwortung der Frage, \textit{warum} dies oder jenes getan werden soll, lässt auf die Kernproblematik
schließen, welche eigentlich gelöst werden muss.

Daraus abgeleitet soll in dieser Arbeit mit der Hilfe von Künstlicher Intelligenz und einem neuronale Netz eine
Möglichkeit gefunden werden, um lange dauernde Konfigurationen von Maschinen oder unnötige Testfahren zu eleminieren.

So soll es Möglich werden, die Maschinen schneller an den Kunden auszuliefern um so frühzeitiger produktiv zu sein.

\newpage

%% TODO noch einbauen?
%%Deutsche Unternehmerbörse~\cite{article_einleitung_dub_ki} und~\cite{article_einleitung_dub_jb}
%%und~\cite{article_einleitung_dub_sb} und~\cite{article_einleitung_dub_il}

%% TODO noch schreiben
\section{Abstract}
\label{sec:abstract}
If one looks at the answer of Amy Webb, founder of the Future Today Institute, to the question \enquote{What is the next
big tech thing} may believe \enquote{(\ldots) (is) Artificial intelligence is the most important technical Development.
We are ringing in the third era of computer science. AI will be everywhere in the future find.}~\cite{article_einleitung_dub_aw}.

The Chinese government is investing hundreds of billions of dollars in artifacts by 2030 Intelligence and deep learning
to build on their progress towards other countries~\cite{article_einleitung_css}. On the other hand, the European Union
wants a \enquote{(\ldots), on the other hand, it seems almost faint-hearted sum of 1.5 Billion euros (invest until
2020).}~\cite{article_einleitung_ww_sg}.

An up-to-date statistics from Statista\footnote{https://de.statista.com/infografik/14245/prognostizierter-umsatz-mit-ki-anwendungen-weltweit}
shows that sales by artificial intelligence by the year 2025 by up to twelve times the current value will increase and
the largest shares will be distributed to North America and Asia.

But are topics like Artificial Intelligence and its neural networks really new? According to Sigmar Gabriel and his
published article in WirtschaftsWoche is a \enquote{(\ldots) old Research area in which since the fifties
(researched)}~\cite{article_einleitung_ww_sg} and also the According to records from Google
Trends~\cite{online_einleitung_googletrends} is just the topic of artificial Intelligence has been a sought-after topic
since 2004 and arouses interest among many people.

Nach anschließend stetig sinkenden Suchanfragen stieg das Interesse an der KI ab Mitte 2016 wieder stark an, sodass es
aktuell in aller Munde ist. Die Themen Machine Learning und Big Data allerdings wurden erst ab Anfang 2012 für die
breite Masse interessant und bilden meist die Ecksäulen der Künstlichen Intelligenz.

Following steadily declining search queries, the interest in the AI rose sharply again from mid-2016, so that it is
currently on everyone's lips. The topics Machine Learning and Big Data, however, were only from the beginning of 2012
for the broad mass interesting and usually form the cornerstones of Artificial Intelligence.

According to Sigmar Gabriel, German companies are mainly expected to develop \enquote{(\ldots) new products and
processes that will help them the existing (improve).} and not \enquote{(\ldots) on the business models of Google,
Amazon and Co.}~\cite{article_einleitung_ww_sg}.

\enquote{Intelligence is the ability to adapt to change.} said Stephen Hawking at a press conference. Currently,
numerous startups and American corporations show what is possible with Artificial Intelligence. This growing trend or
this change in the development of products should not be allowed by local companies overslept by placing risks ahead of
emerging opportunities or emerging businesses.

Because, according to Amy Webb, \enquote{The Consequences (if a company does not respond to Artificial Intelligence)
be devastating (\ldots)} and da \enquote{(\ldots) every commercial sector should be affected (\ldots)} more and more
companies are getting more involved with the topic~\cite{article_einleitung_dub_aw}.

According to a McKinsey study from 2017, German economic performance could be boosted by the use of smart Robots and
self-learning computers to be 160 billion euros higher than today and not necessarily with less
Employees~\cite{online_einleitung_mckinsey}. However, an IDC study from June 2017 indicates that around 40 percent
The surveyed companies in Europe and the US plan to actively benefit from AI technologies by the end of 2018
use~\cite{article_grundlagen_salesforce}.

In order not to be suspended in this competition, topics such as Artificial Intelligence and Neural Networks are slow
also interesting for many manufacturing companies from Germany, who want to equip their products with intelligence,
to make them better or faster.

Also the Robert Bosch GmbH with its Robert Bosch Packaging Technology society has strong interest in the Packaging
machines produced by them work more efficiently or, for example, faster to the customer deliver it so that it can use it
more.

The root cause analysis by Don Norman from his book \textit{The Design of Everyday
Things}~\cite{book_einleitung_donnorman} states that the cause of human action to solve problems by answering the five
\textit{why}-questions. By a chain reaction are then all the others Problem solved.

The five-time answer to the question, \textit{why} this or that should be done, leaves to the core problem close, which
actually has to be solved.

The result of this work is the use of artificial intelligence and a neural network Possibility to be found to eliminate
long-lasting configurations of machines or unnecessary test driving.

So it should be possible to deliver the machines faster to the customer so as to be productive earlier.

\newpage

\section{Aufgabenstellung}
\label{sec:aufgabenstellung}
Das folgende Kapitel soll eine kleine Übersicht über die Themen geben, welche in der Arbeit behandelt werden. Für die
leichtere Umsetzung ist die Arbeit in drei größere Themenbereiche unterteilt.

Der erste Teil der Arbeit befasst sich mit dem Aufbau eines neuronalen Netzes, welches in der Cloud trainiert wird. Der
Nutzer hat anschließend die Möglichkeit, das trainierte Netz aus der Cloud heraus zu nutzen oder es lokal auf einem
eigenen Rechner einzurichten.

Dabei sollen die Vorteile des Cloud Computings genutzt werden, um das Training des neuronalen Netzes erheblich zu
beschleunigen. DevOps-Tools sollen dabei helfen, die Bereitstellung des Netzes und erforderliche Schnittstellen zu
vereinfachen. Cloud-Services erleichtern die Entwicklung, indem sie das Schreiben von Quellcode verringern.

Im zweiten Teil der Arbeit wird das bereitgestellte neuronale Netz über ein Frontend angesteuert um Parameter zu
übergeben und die resultierenden Vorhersagen zu veranschaulichen. Ebenfalls sollen Smartphone-Apps für Android und iOS
entstehen, um die Nutzung der Anwendung im Bereich der Maschinenauslieferung zu vereinfachen.

Diverse Tests sollen die Funktionsweise des erstellten neuronalen Netzes, der Schnittstellen und des Frontends
sowie den Smartphone-Apps sicherstellen.

Der letzte Teil widmet sich der Adaptierbarkeit auf weitere Systeme, Maschinen und Netze der aufgebauten Architektur.
Dort werden weitere Szenarien aufgezeigt, die für das Automatisieren von Maschinenparametern entscheidend sein können.
Es soll eine Anleitung entstehen, die das Einbauen der neuen und auch zukünftigen Szenarien vereinfacht.

Zum Schluss folgt ein Ausblick mit Anregungen und Erweiterungen sowohl für die Weiterentwicklung des neuronalen Netzes
als auch für das Web-Frontend und die Smartphone-Apps. Auch sollen weitere Cloud-Services aufgezeigt werden, welche
zusätzliche Möglichkeiten zur Erweiterung bieten.

Ziel ist es, der Robert Bosch GmbH sowie der Abteilung Bosch Packaging Technology dabei zu helfen, Kundenaufträge, die
das Einstellen von Maschinen oder Maschinenkomponenten beinhalten, zu beschleunigen und so die Wartezeit für den Kunden
zu minimieren. Auch soll so Fehlern vorgebeut werden und eine Hilfestellung für neue Maschineneinsteller entstehen.

\newpage

\section{Aufbau der Arbeit}
\label{sec:aufbauDerArbeit}
Dieses Kapitel dient zur schnelleren Orientierung innerhalb der Arbeit und zeigt auf, welche Themen die jeweiligen
Kapitel behandeln.

\begin{description}

    \item[Kapitel 2 (Grundlagen)]\hfill \\
    Dieses Kapitel erklärt die Grundlagen. Es wird zum Beispiel der Unterschied zwischen den verschiedenen Cloudanbieter
    im Bereich Machine Learning erläutert, um was es sich bei TensorFlow.js handelt und was ein WebViewer ist. Außerdem
    werden die genutzten Bosch-Maschinen kurz vorgestellt.

    \item[Kapitel 3 (Neuronales Netz)]\hfill \\
    Das dritte Kapitel erläutert die Entwicklung des Machine Learning Algorithmuses in der Cloud. Dabei werden
    verschiedene Optionen für die einzelnen Teilaufgaben analysiert und im Anschluss die genutzten Programme installiert,
    eingerichtet und erläutert. Das trainierte Model wird abschließend in TensorFlow.js importiert und über einen Wrapper
    in einem Container zugänglich gemacht. Ein API Management soll für die Komunikation sorgen.

    \item[Kapitel 4 (Client)]\hfill \\
    Das vierte Kapitel setzt das in der Cloud trainierte Model voraus und implementiert dazu ein prototypisches Frontend.
    Dieses besteht aus einer Angular-Anwendung sowie zwei Smartphone-Apps. Dabei werden zu Anfang verschiedene Programme
    und Services analysiert, um diese im Anschluss zu nutzen.

    \item[Kapitel 5 (Adaptierbarkeit)]\hfill \\
    Hier werden Themen angesprochen, die für die weitere Verwendung oder Erweiterung des Machine Learning Algorithmus
    oder des Frontends, entscheidend sind. Unter anderem wird erläutert, welche Schritte notwendig sind, um neue
    Maschinen oder Komponenten in die Architektur einzubauen oder wie optimale Daten zusammengestellt werden müssen. Die
    Adaptierbarkeit wird über mehreren Szenarien veranschaulicht.

    \item[Kapitel 6 (Ausblick)]\hfill \\
    Wie sowohl das neuronale Netz, das Model oder auch das Frontend erweitert werden kann, zeigt dieses Kapitel auf.
    Dabei können neue Cloud-Services hinzugefügt oder auch weitere Funktionen der schon eingesetzten Services verwendet
    werden.

    \item[Kapitel 7 (Zusammenfassung)]\hfill \\
    Das letzte Kapitel enthält eine Zusammenfassung aller vorangegangenen Kapitel und ein Schlusswort.

\end{description}
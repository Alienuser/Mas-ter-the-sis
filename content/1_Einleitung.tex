\chapter{Einleitung}
\label{ch:einleitung}

\section{Motivation}
\label{sec:motivation}
Wenn man der Antwort von Amy Webb, ihres Zeichens Gründerin des Future Today Institute, auf die Frage \enquote{Was ist
the next big tech-thing} Glauben schenken darf, \enquote{(\ldots) (ist) Künstliche Intelligenz die wichtigste technische
Entwicklung. Wir läuten damit die dritte Ära der Informatik ein. KI wird sich künftig überall
finden.}~\cite{article_einleitung_dub_aw}.

Die chinesische Regierung etwa investiert hunderte Milliarden Dollar bis zum Jahr 2030 in Themen wie künstliche
Intelligenz und Deep Learning um ihren Fortschritt gegenüber anderen Ländern ausbauen~\cite{article_einleitung_css}.
Hingegen möchte die Europäische Union eine \enquote{(\ldots) dagegen fast schon kleinmütig erscheinende Summe von 1,5
Milliarden Euro (bis 2020 investieren).}~\cite{article_einleitung_ww_sg}.

Eine aktuelle Statistik von
Statista\footnote{https://de.statista.com/infografik/14245/prognostizierter-umsatz-mit-ki-anwendungen-weltweit} zeigt,
dass sich der durch künstliche Intelligenz generierte Umsatz bis zum Jahr 2025 um das bis zu zwölffache des aktuellen
Wertes steigert und sich die größten Anteile auf Nordamerika und Asien verteilen werden.

Sind aber Themen wie künstliche Intelligenz und die damit einhergehenden neuronalen Netze wirklich neu? Laut Sigmar
Gabriel und seinem veröffentlichten Artikel in der WirtschaftsWoche handelt es sich um ein \enquote{(\ldots) altes
Forschungsgebiet in dem seit den Fünfzigerjahren (geforscht wird)}~\cite{article_einleitung_ww_sg} und auch den
Aufzeichnungen von Google-Trends zufolge~\cite{online_einleitung_googletrends} ist gerade das Thema künstliche
Intelligenz schon seit dem Jahr 2004 ein gesuchtes Thema und weckt bei vielen Menschen Interesse.

Nach anfänglich stetig sinkenden Suchanfragen stieg das Interesse an der KI, bedingt durch schnell wachsende
Rechenleistung, ab Mitte 2016 wieder stark an, sodass es aktuell in aller Munde ist. Die Themen
\textit{Machine Learning} und \textit{Big Data} allerdings wurden erst ab Anfang 2012 für die breite Masse interessant
und bilden meist die Ecksäulen der künstlichen Intelligenz.

\enquote{Intelligenz ist die Fähigkeit, sich dem Wandel anzupassen.}, sagte Stephen Hawking auf einer Pressekonferenz.
Aktuell zeigen zahlreiche Startups sowie amerikanische Großkonzerne, was mit künstlicher Intelligenz alles möglich ist.
Diesen wachsenden Trend beziehungsweise diesen Wandel in der Entwicklung von Produkten dürfen hiesige Firmen nicht
verschlafen, indem sie Risiken vor neue Möglichkeiten oder die neu entstehenden Geschäftsbereiche stellen.

Allerdings sollen laut Sigmar Gabriel deutsche Unternehmen \enquote{(\ldots) neue Produkte und Verfahren
entwickeln, die das Vorhandene (verbessern).} und sich nicht \enquote{(\ldots) an den Geschäftsmodellen von Google,
Amazon und Co.}~\cite{article_einleitung_ww_sg} orientieren.

Einer McKinsey-Studie von 2017 zufolge könnte die deutsche Wirtschaftsleistung durch den Einsatz von intelligenten
Robotern und selbstlernenden Computern um 160 Milliarden Euro höher sein als heute und das nicht zwangsweise mit weniger
Beschäftigten~\cite{online_einleitung_mckinsey}. Allerdings gibt eine IDC-Studie vom Juni 2017 an, dass rund 40 Prozent
der befragten Unternehmen in Europa und den USA planen, bis Ende 2018 aktiv von KI-Technologien zu profitieren und diese
einzusetzen~\cite{article_grundlagen_salesforce}.

Denn laut Amy Webb können \enquote{Die Folgen (wenn ein Unternehmen nicht auf künstliche Intelligenz reagiert)
verheerend sein (\ldots)} und da \enquote{(\ldots) jeder kommerzielle Sektor betroffen sein (wird) (\ldots)}, sollten
sich immer mehr Firmen näher mit dem Thema beschäftigen~\cite{article_einleitung_dub_aw}.

Um in diesem Wettstreit nicht abgehängt zu werden, beschäftigen sich immer mehr produzierende Firmen aus Deutschland mit
Themen wie künstliche Intelligenz und neuronale Netzen, um ihre Produkte mit Intelligenz auszustatten, sodass diese
besser oder auch schneller werden.

So hat auch die Bosch-Gruppe mit der Tochtergesellschaft Packaging Technology GmbH ein starkes Interesse daran, die von
ihnen produzierten Verpackungsmaschinen effizienter arbeiten zu lassen oder sie einfacher auf neue Produkte umzurüsten.

Die Ursachenanalyse (englisch root cause analysis) von Don Norman aus seinem Buch \textit{The Design of Everyday
Things}~\cite{book_einleitung_donnorman} besagt, dass die Ursache des menschlichen Handelns zum Lösen von Problemen
durch Beantwortung der fünf \textit{Warum}-Fragen geschieht. Durch eine Kettenreaktion sind dann auch alle anderen
Probleme gelöst.

Die fünfmalige Beantwortung der Frage, \textit{warum} dies oder jenes getan werden soll, lässt auf die Kernproblematik
schließen, die es eigentlich zu Lösen gilt.

Daraus abgeleitet soll in dieser Arbeit mit der Hilfe von künstlicher Intelligenz und einem neuronalen Netz eine
Möglichkeit gefunden werden, um lang andauernde Konfigurationen von Maschinen oder unnötige Testläufe zu eliminieren.

So soll es möglich werden, die produzierten Verpackungsmaschinen effizienter arbeiten zu lassen und sie einfacher auf
neue Produkte umzurüsten.

\newpage

\section{Abstract}
\label{sec:abstract}
If one can believe the answer of Amy Webb, founder of the Future Today Institute, to the question \enquote{What is the
next big tech-thing}, \enquote{(\ldots) artificial intelligence (is) the most important technical development. It
heralds the third era of computer science. AI will be everywhere in the future.}~\cite{article_einleitung_dub_aw}.

The Chinese government, for example, is investing hundreds of billions of dollars by 2030 in topics such as artificial
intelligence and deep learning to improve its progress over other countries~\cite{article_einleitung_css}.

On the other hand, the European Union would like to invest a \enquote{(\ldots) sum of 1.5 billion euros (by 2020),
which seems almost petty in comparison.}~\cite{article_einleitung_ww_sg}.

A current statistic from
Statista\footnote{https://de.statista.com/infografik/14245/prognostizierter-umsatz-mit-ki-anwendungen-weltweit} shows
that the turnover generated by artificial intelligence will increase by up to twelve times the current value by 2025
and that the largest shares will be distributed over North America and Asia.

But are topics such as artificial intelligence and the associated neuronal networks really new? According to Sigmar
Gabriel and his published article in WirtschaftsWoche it is \enquote{(\ldots) an old field of research in which
research has been carried out since the 1950s}~\cite{article_einleitung_ww_sg} and also according to Google
trends~\cite{online_einleitung_googletrends} the topic of artificial intelligence has been a sought-after topic since
2004 and arouses interest among many people.

After an initial steady decline in search queries, interest in the AI rose sharply again from mid-2016 onwards due to
rapidly increasing computing power, so that it is currently on everyone's lips.

However, the topics of \textit{machine learning} and \textit{big data} only became interesting for the general public
at the beginning of 2012 and usually form the cornerstones of artificial intelligence.

\enquote{Intelligence is the ability to adapt to change.}, Stephen Hawking said at a press conference. Currently,
numerous startups and large American corporations are showing what artificial intelligence can do.

Local companies must not oversleep this growing trend or this change in the development of products by putting risks
before new opportunities or the newly emerging business areas.

According to Sigmar Gabriel, however, German companies should \enquote{(\ldots) develop new products and processes that
(improve) what already exists} and not \enquote{(\ldots) follow the business models of Google, Amazon and
Co.}~\cite{article_einleitung_ww_sg}.

And according to a McKinsey study conducted in 2017, the use of intelligent robots and self-learning computers could
increase Germany's economic output by 160 billion euros compared to today, and not necessarily with fewer
employees~\cite{online_einleitung_mckinsey}.

However, an IDC study from June 2017 states that around 40 percent of the companies surveyed in Europe and the USA plan
to actively benefit from AI technologies by the end of 2018 and use them~\cite{article_grundlagen_salesforce}.

Amy Webb says that \enquote{the consequences (if a company does not react to artificial intelligence) can be devastating
(\ldots)} and that \enquote{(\ldots) any commercial sector will be affected (\ldots)} if more and more companies take a
closer look at the topic~\cite{article_einleitung_dub_aw}.

In order not to be left behind in this competition, more and more manufacturing companies from Germany are dealing with
topics such as artificial intelligence and neural networks in order to equip their products with intelligence so that
they become better or faster.

Robert Bosch GmbH, with its subsidiary Packaging Technology GmbH, also has a strong interest in making the packaging
machines it produces more efficient or in making it easier to convert them to new products.

The root cause analysis by Don Norman from his own book \textit{The Design of Everyday
Things}~\cite{book_einleitung_donnorman} states that the cause of human action to solve problems occurs by answering
the five \textit{why} questions. All other problems are then solved by a chain reaction.

Five answers to the question as to \textit{why} this or that should be done lead to the conclusion that the core
problem actually needs to be solved.

Derived from this, this work, with the help of artificial intelligence and a neural network, aims to find a way to
eliminate long-lasting configurations of machines or unnecessary test runs.

This should make it possible to make the packaging machines produced work more efficiently and to convert them more
easily to new products.

\newpage

\section{Aufgabenstellung}
\label{sec:aufgabenstellung}
Das folgende Kapitel soll eine kleine Übersicht über die Themen geben, welche in der Arbeit behandelt werden. Für das
bessere Verständnis ist die Arbeit in drei größere Themenbereiche unterteilt.

Der erste Teil der Arbeit befasst sich mit dem Aufbau eines neuronalen Netzes, welches in der Cloud trainiert wird. Der
Nutzer hat anschließend die Möglichkeit, das trainierte Netz aus der Cloud heraus zu nutzen oder es lokal auf einem
eigenen Rechner einzurichten.

Dabei sollen die Vorteile des Cloud Computings genutzt werden, um das Training des neuronalen Netzes erheblich zu
beschleunigen. Ein Prozessverbesserungs-Ansatz aus den Bereichen der Softwareentwicklung (kurz DevOps) hilft dabei, die
Bereitstellung des Netzes und erforderliche Schnittstellen zu vereinfachen. Cloud-Services erleichtern die Entwicklung,
indem sie das Schreiben von Quellcode verringern.

Im zweiten Teil der Arbeit wird das bereitgestellte neuronale Netz über ein Frontend angesteuert um Parameter  zu
übergeben und die resultierenden Vorhersagen zu veranschaulichen. Ebenfalls sollen Smartphone-Apps für Android und iOS
entstehen, um die Nutzung der Anwendung im Bereich der Maschinenauslieferung zu vereinfachen.

Diverse Tests sollen die Funktionsweise des erstellten neuronalen Netzes, der Schnittstellen und des Frontends
sowie den Smartphone-Apps sicherstellen, damit bei Änderungen keine Ausfälle entstehen.

Der letzte Teil der Arbeit widmet sich der Adaptierbarkeit auf weitere Systeme, Maschinen und Netze der aufgebauten
Architektur. Dort werden weitere Szenarien aufgezeigt, die für das Automatisieren von Maschinenparametern entscheidend
sein können. Es soll eine Anleitung entstehen, die das Einbauen der neuen und auch zukünftigen Szenarien vereinfacht.

Zum Schluss folgt ein Ausblick mit Anregungen und Erweiterungen sowohl für die Weiterentwicklung des neuronalen Netzes
als auch für das Web-Frontend und die Smartphone-Apps. Auch sollen weitere Cloud-Services aufgezeigt werden, welche
zusätzliche Möglichkeiten zur Erweiterung des Systems bieten.

Ziel ist es, der Bosch-Gruppe sowie der Tochtergesellschaft Packaging Technology GmbH dabei zu helfen, Kundenaufträge,
die das Einstellen von Maschinen oder Maschinenkomponenten beinhalten, zu beschleunigen und so die Wartezeit für den
Endkunden zu minimieren. Auch soll so Fehlern vorgebeut werden und eine Hilfestellung für neue Maschineneinsteller
entstehen.

\newpage

\section{Aufbau der Arbeit}
\label{sec:aufbauDerArbeit}
Dieses Kapitel dient zur schnelleren Orientierung innerhalb der Arbeit und zeigt auf, welche Themen die jeweiligen
Kapitel behandeln.

\begin{description}

    \item[Kapitel 2 (Grundlagen)]\hfill \\
    Dieses Kapitel erläutert die Grundlagen. Es wird zum Beispiel der Unterschied zwischen den verschiedenen
    Cloudanbietern im Bereich Machine Learning aufgezeigt, um was es sich bei TensorFlow.js handelt und was ein WebViewer
    ist. Außerdem werden die genutzten Bosch-Maschinen kurz eingeführt.

    \item[Kapitel 3 (Neuronales Netz)]\hfill \\
    Das dritte Kapitel beschreibt die Entwicklung des Machine Learning Algorithmuses in der Cloud. Dabei werden
    verschiedene Optionen für die einzelnen Teilaufgaben analysiert und im Anschluss die genutzten Programme
    installiert, eingerichtet und erklärt. Das trainierte Modell wird abschließend in TensorFlow.js importiert und über
    einen Wrapper in einem Container zugänglich gemacht. Ein API Management soll für die Kommunikation sorgen.

    \item[Kapitel 4 (Client)]\hfill \\
    Das vierte Kapitel setzt das in der Cloud trainierte Modell voraus und implementiert dazu ein prototypisches
    Frontend. Dieses besteht aus einer Angular-Anwendung sowie zwei Smartphone-Apps. Dabei werden zu Anfang verschiedene
    Programme und Services analysiert, um diese im Anschluss zu nutzen.

    \item[Kapitel 5 (Adaptierbarkeit)]\hfill \\
    Hier werden Themen angesprochen, die für die weitere Verwendung oder Erweiterung des Machine Learning Algorithmus
    oder des Frontends, entscheidend sind. Unter anderem wird erläutert, welche Schritte notwendig sind, um neue
    Maschinen oder Komponenten in die Architektur einzubauen oder wie optimale Daten zusammengestellt werden müssen. Die
    Adaptierbarkeit wird in einem Szenario veranschaulicht.

    \item[Kapitel 6 (Ausblick)]\hfill \\
    Wie sowohl das neuronale Netz, das Modell oder auch das Frontend erweitert werden kann, zeigt dieses Kapitel auf.
    Dabei können neue Cloud-Services hinzugefügt oder auch weitere Funktionen der schon eingesetzten Services verwendet
    werden.

    \item[Kapitel 7 (Zusammenfassung)]\hfill \\
    Das letzte Kapitel enthält eine Zusammenfassung aller vorangegangener Kapitel und ein Schlusswort.

\end{description}
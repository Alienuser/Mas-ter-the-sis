\chapter{Einleitung}
\label{ch:einleitung}

\section{Motivation}
\label{sec:motivation}
Wenn man der Antwort von Amy Webb, ihres Zeichens Gründerin des Future Today Institute, auf die Frage \enquote{Was ist
the next big tech-thing} Glauben schenken darf, \enquote{(\ldots) (ist) Künstliche Intelligenz die wichtigste technische
Entwicklung. Wir läuten damit die dritte Ära der Informatik ein. KI wird sich künftig überall
finden.}~\cite{article_einleitung_dub_aw}.

Die chinesische Regierung etwa investiert hunderte Milliarden Dollar bis zum Jahr 2030 in Themen wie künstliche
Intelligenz und Deep Learning um ihren Fortschritt gegenüber anderen Ländern ausbauen~\cite{article_einleitung_css}.
Hingegen möchte die Europäische Union eine \enquote{(\ldots) dagegen fast schon kleinmütig erscheinende Summe von 1,5
Milliarden Euro (bis 2020 investieren).}~\cite{article_einleitung_ww_sg}.

Eine aktuelle Statistik von
Statista\footnote{https://de.statista.com/infografik/14245/prognostizierter-umsatz-mit-ki-anwendungen-weltweit} zeigt,
dass sich der durch künstliche Intelligenz generierte Umsatz bis zum Jahr 2025 um das bis zu zwölffache des aktuellen
Wertes steigert und sich die größten Anteile auf Nordamerika und Asien verteilen werden.

Aber sind Themen wie künstliche Intelligenz und die damit einhergehenden neuronalen Netze wirklich neu? Laut Sigmar
Gabriel und seinem veröffentlichten Artikel in der WirtschaftsWoche handelt es sich um ein \enquote{(\ldots) altes
Forschungsgebiet in dem seit den Fünfzigerjahren (geforscht wird)}~\cite{article_einleitung_ww_sg} und auch den
Aufzeichnungen von Google-Trends zufolge~\cite{online_einleitung_googletrends} ist gerade das Thema künstliche
Intelligenz schon seit dem Jahr 2004 ein gesuchtes Thema und weckt bei vielen Menschen Interesse.

Nach anschließend stetig sinkenden Suchanfragen stieg das Interesse an der KI ab Mitte 2016 wieder stark an, sodass es
aktuell in aller Munde ist. Die Themen \textit{Machine Learning} und \textit{Big Data} allerdings wurden erst ab Anfang
2012 für die breite Masse interessant und bilden meist die Ecksäulen der künstlichen Intelligenz.

\enquote{Intelligenz ist die Fähigkeit, sich dem Wandel anzupassen.} sagte Stephen Hawking auf einer Pressekonferenz.
Aktuell zeigen zahlreiche Startups sowie amerikanische Großkonzerne, was mit künstlicher Intelligenz alles möglich ist.
Diesen wachsenden Trend beziehungsweise diesen Wandel in der Entwicklung von Produkten dürfen hiesige Firmen nicht
verschlafen, indem sie Risiken vor die neuen Möglichkeiten oder die neu entstehenden Geschäftsbereiche stellen.

Allerdings sollen laut Sigmar Gabriel deutsche Unternehmen \enquote{(\ldots) neue Produkte und Verfahren
entwickeln, die das Vorhandene (verbessern).} und sich nicht \enquote{(\ldots) an den Geschäftsmodellen von Google,
Amazon und Co.}~\cite{article_einleitung_ww_sg} orientieren.

Einer McKinsey-Studie von 2017 zufolge könnte die deutsche Wirtschaftsleistung durch den Einsatz von intelligenten
Robotern und selbstlernenden Computern um 160 Milliarden Euro höher sein als heute und das nicht zwangsweise mit weniger
Beschäftigten~\cite{online_einleitung_mckinsey}. Allerdings gibt eine IDC-Studie vom Juni 2017 an, dass rund 40 Prozent
der befragten Unternehmen in Europa und den USA planen, bis Ende 2018 aktiv von KI-Technologien zu profitieren und diese
einzusetzen~\cite{article_grundlagen_salesforce}.

Denn laut Amy Webb können \enquote{Die Folgen (wenn ein Unternehmen nicht auf künstliche Intelligenz reagiert)
verheerend sein (\ldots)} und da \enquote{(\ldots) jeder kommerzielle Sektor betroffen sein (wird) (\ldots)}, sollten
sich immer mehr Firmen näher mit dem Thema beschäftigen~\cite{article_einleitung_dub_aw}.

Um in diesem Wettstreit nicht abgehängt zu werden, beschäftigen sich immer mehr produzierende Firmen aus Deutschland mit
Themen wie künstliche Intelligenz und neuronale Netzen, um ihre Produkte mit Intelligenz auszustatten sodass diese
besser oder auch schneller werden.

Auch die Robert Bosch GmbH mit seiner Robert Bosch Packaging Technology Gesellschaft hat starkes Interesse daran, die
von ihnen produzierten Verpackungsmaschinen effizienter arbeiten zu lassen oder sie schneller an den Kunden
auszuliefern, damit dieser sie eher nutzen kann.

Die Ursachenanalyse (englisch root cause analysis) von Don Norman aus seinem Buch \textit{The Design of Everyday
Things}~\cite{book_einleitung_donnorman} besagt, dass die Ursache des menschlichen Handelns zum Lösen von Problemen
durch Beantwortung der fünf \textit{Warum}-Fragen geschieht. Durch eine Kettenreaktion sind dann auch alle anderen
Probleme gelöst.

Die fünfmalige Beantwortung der Frage, \textit{warum} dies oder jenes getan werden soll, lässt auf die Kernproblematik
schließen, die es eigentlich zu Lösen gilt.

Daraus abgeleitet soll in dieser Arbeit mit der Hilfe von künstlicher Intelligenz und einem neuronale Netz eine
Möglichkeit gefunden werden, um lang dauernde Konfigurationen von Maschinen oder unnötige Testfahren zu eliminieren.

So soll es Möglich werden, die produzierten Verpackungsmaschinen schneller an den Kunden auszuliefern um so frühzeitiger
produktiv mit ihnen zu sein.

\newpage

\section{Abstract}
\label{sec:abstract}
If one can believe the answer of Amy Webb, founder of the Future Today Institute, to the question \enquote{What is the
next big tech-thing}, \enquote{(\ldots) artificial intelligence (is) the most important technical development. This
marks the beginning of the third era of computer science. AI will be everywhere in the
future.}~\cite{article_einleitung_dub_aw}.

The Chinese government, for example, is investing hundreds of billions of dollars by the year 2030 in topics such as
artificial intelligence and deep learning in order to expand its progress compared to other
countries~\cite{article_einleitung_css}. On the other hand, the European Union would like to invest a \enquote{(\ldots)
sum of 1.5 billion euros (by 2020), which seems almost petty in comparison.}~\cite{article_einleitung_ww_sg}.

A current statistic from
Statista\footnote{https://de.statista.com/infografik/14245/prognostizierter-umsatz-mit-ki-anwendungen-weltweit} shows
that by the year 2025 the turnover from artificial intelligence will increase by up to twelve times the current value
and that the largest shares will be distributed over North America and Asia.

But are topics such as artificial intelligence and the associated newronal networks really new? According to Sigmar
Gabriel and his published article in WirtschaftsWoche, it is an \enquote{(\ldots) old field of research in which
research has been carried out since the 1950s}~\cite{article_einleitung_ww_sg} and according to Google
Trends~\cite{online_einleitung_googletrends}, the topic of artificial intelligence has been a sought after topic since
2004 and has aroused the interest of many people.

After a steady decline in search queries, interest in the AI rose sharply again from mid-2016 onwards, so that there was
no need to wait for a new AI. is currently on everyone's lips. The topics Machine Learning and Big Data, however, were
not introduced until the beginning of 2012 for the and are usually the cornerstones of artificial intelligence.

\enquote{Intelligence is the ability to adapt to change.} said Stephen Hawking at a press conference. Currently,
numerous startups and large American corporations are showing what artificial intelligence can do.

Local companies must not oversleep this growing trend or this change in the development of products by placing risks
before the emerging opportunities or the newly emerging business areas.

According to Sigmar Gabriel, German companies should above all \enquote{(\ldots) develop new products and processes that
(improve) what already exists} and not \enquote{(\ldots) orient themselves on the business models of Google, Amazon and
Co.}~\cite{article_einleitung_ww_sg}.

According to a McKinsey study conducted in 2017, the use of intelligent robots and self-learning computers could
increase Germany's economic output by 160 billion euros compared to today, and not necessarily with fewer
employees~\cite{online_einleitung_mckinsey}.

However, an IDC study from June 2017 states that around 40 percent of the companies surveyed in Europe and the USA plan
to actively benefit from AI technologies by the end of 2018 and use them~\cite{article_grundlagen_salesforce}.

According to Amy Webb, \enquote{The consequences (if a company does not react to artificial intelligence) can be
devastating (\ldots)} and because \enquote{(\ldots) every commercial sector will be affected (\ldots)} more and more
companies should take a closer look at the topic~\cite{article_einleitung_dub_aw}.

In order not to be left behind in this competition, topics such as artificial intelligence and neural networks are
slowly becoming interesting for many manufacturing companies from Germany who want to equip their products with
intelligence in order to make them better or faster.

Robert Bosch GmbH and its Robert Bosch Packaging Technology company also have a strong interest in making the packaging
machines they produce more efficient or, for example, in delivering them to customers faster so that they can use them
sooner.

The root cause analysis by Don Norman from his book \textit{The Design of Everyday
Things}~\cite{book_einleitung_donnorman} states that the cause of human action to solve problems must be done by
answering the five \textit{why} questions. By a chain reaction all other problems are solved.

Five answers to the question as to why this or that should be done lead to the conclusion that the core problem actually
needs to be solved.

Derived from this, this work, with the help of artificial intelligence and a neural network, aims to find a way to
eliminate long-lasting configurations of machines or unnecessary test drives.

This should make it possible to deliver the packaging machines produced to the customer more quickly so that they can be
productive earlier.

\newpage

\section{Aufgabenstellung}
\label{sec:aufgabenstellung}
Das folgende Kapitel soll eine kleine Übersicht über die Themen geben, welche in der Arbeit behandelt werden. Für das
bessere Verständnis ist die Arbeit in drei größere Themenbereiche unterteilt.

Der erste Teil der Arbeit befasst sich mit dem Aufbau eines neuronalen Netzes, welches in der Cloud trainiert wird. Der
Nutzer hat anschließend die Möglichkeit, das trainierte Netz aus der Cloud heraus zu nutzen oder es lokal auf einem
eigenen Rechner einzurichten.

Dabei sollen die Vorteile des Cloud Computings genutzt werden, um das Training des neuronalen Netzes erheblich zu
beschleunigen. Ein Prozessverbesserungs-Ansatz aus den Bereichen der Softwareentwicklung (kurz DevOps) hilft dabei, die
Bereitstellung des Netzes und erforderliche Schnittstellen zu vereinfachen. Cloud-Services erleichtern die Entwicklung,
indem sie das Schreiben von Quellcode verringern.

Im zweiten Teil der Arbeit wird das bereitgestellte neuronale Netz über ein Frontend angesteuert um Parameter  zu
übergeben und die resultierenden Vorhersagen zu veranschaulichen. Ebenfalls sollen Smartphone-Apps für Android und iOS
entstehen, um die Nutzung der Anwendung im Bereich der Maschinenauslieferung zu vereinfachen.

Diverse Tests sollen die Funktionsweise des erstellten neuronalen Netzes, der Schnittstellen und des Frontends
sowie den Smartphone-Apps sicherstellen, damit bei Änderungen keine Ausfälle entstehen.

Der letzte Teil der Arbeit widmet sich der Adaptierbarkeit auf weitere Systeme, Maschinen und Netze der aufgebauten
Architektur. Dort werden weitere Szenarien aufgezeigt, die für das Automatisieren von Maschinenparametern entscheidend
sein können. Es soll eine Anleitung entstehen, die das Einbauen der neuen und auch zukünftigen Szenarien vereinfacht.

Zum Schluss folgt ein Ausblick mit Anregungen und Erweiterungen sowohl für die Weiterentwicklung des neuronalen Netzes
als auch für das Web-Frontend und die Smartphone-Apps. Auch sollen weitere Cloud-Services aufgezeigt werden, welche
zusätzliche Möglichkeiten zur Erweiterung des Systems bieten.

Ziel ist es, der Robert Bosch GmbH sowie der Abteilung Bosch Packaging Technology dabei zu helfen, Kundenaufträge, die
das Einstellen von Maschinen oder Maschinenkomponenten beinhalten, zu beschleunigen und so die Wartezeit für den
Endkunden zu minimieren. Auch soll so Fehlern vorgebeut werden und eine Hilfestellung für neue Maschineneinsteller
entstehen.

\newpage

\section{Aufbau der Arbeit}
\label{sec:aufbauDerArbeit}
Dieses Kapitel dient zur schnelleren Orientierung innerhalb der Arbeit und zeigt auf, welche Themen die jeweiligen
Kapitel behandeln.

\begin{description}

    \item[Kapitel 2 (Grundlagen)]\hfill \\
    Dieses Kapitel erklärt die Grundlagen. Es wird zum Beispiel der Unterschied zwischen den verschiedenen Cloudanbieter
    im Bereich Machine Learning erläutert, um was es sich bei TensorFlow.js handelt und was ein WebViewer ist. Außerdem
    werden die genutzten Bosch-Maschinen kurz vorgestellt.

    \item[Kapitel 3 (Neuronales Netz)]\hfill \\
    Das dritte Kapitel erläutert die Entwicklung des Machine Learning Algorithmuses in der Cloud. Dabei werden
    verschiedene Optionen für die einzelnen Teilaufgaben analysiert und im Anschluss die genutzten Programme installiert,
    eingerichtet und erläutert. Das trainierte Modell wird abschließend in TensorFlow.js importiert und über einen Wrapper
    in einem Container zugänglich gemacht. Ein API Management soll für die Komunikation sorgen.

    \item[Kapitel 4 (Client)]\hfill \\
    Das vierte Kapitel setzt das in der Cloud trainierte Modell voraus und implementiert dazu ein prototypisches Frontend.
    Dieses besteht aus einer Angular-Anwendung sowie zwei Smartphone-Apps. Dabei werden zu Anfang verschiedene Programme
    und Services analysiert, um diese im Anschluss zu nutzen.

    \item[Kapitel 5 (Adaptierbarkeit)]\hfill \\
    Hier werden Themen angesprochen, die für die weitere Verwendung oder Erweiterung des Machine Learning Algorithmus
    oder des Frontends, entscheidend sind. Unter anderem wird erläutert, welche Schritte notwendig sind, um neue
    Maschinen oder Komponenten in die Architektur einzubauen oder wie optimale Daten zusammengestellt werden müssen. Die
    Adaptierbarkeit wird über mehreren Szenarien veranschaulicht.

    \item[Kapitel 6 (Ausblick)]\hfill \\
    Wie sowohl das neuronale Netz, das Modell oder auch das Frontend erweitert werden kann, zeigt dieses Kapitel auf.
    Dabei können neue Cloud-Services hinzugefügt oder auch weitere Funktionen der schon eingesetzten Services verwendet
    werden.

    \item[Kapitel 7 (Zusammenfassung)]\hfill \\
    Das letzte Kapitel enthält eine Zusammenfassung aller vorangegangenen Kapitel und ein Schlusswort.

\end{description}
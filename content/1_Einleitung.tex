\chapter{Einleitung}
\label{ch:einleitung}

\section{Motivation}
\label{sec:motivation}
- Warum habe ich das gemacht?
- Was hat Bosch davon?
- Was ändert sich am Markt?
\\ \\
- Seit wann gibt es Machine Learning und seit wann wird es erst richtig genutzt
- Machine Learning ein immer Wichtigerer Aspekt im Leben
- Andere Firmen haben viel damit gemacht
- Maschinen schneller und besser einstellen können
- Maschinen Reibungsloser einstellen
- Wiederkehrende Probleme sehen
- Maschinen automatisch einstellen
- Performance der Maschine verbessern
\\ \\
- Hypothese, dass ich das Problem mit Machine Learning lösen will
\\ \\
2 Seiten

\colorbox{yellow}{Hier fehlt was}

\section{Abstract}
English Version of the motivation.
\colorbox{yellow}{Hier fehlt was}

\section{Aufgabenstellung}
\label{sec:aufgabenstellung}
- Was soll in der Arbeit umgesetzt werden?
- Grob beschreiben was alles durchgeführt wird?
- Grob beschreiben, was rauskommen soll.
- Was ist das Ziel des ganzen
- Was kommt nachher raus
\\ \\
- Ich habe in der Cloud ein Machine Learning Netzwerk aufgebaut
- Machine Learning Netzwerk mit TensorFlow.js
- Webseite (Client) zum eingeben und auslesen der Daten
- REST-Schnittstelle zur Kommunikation
- Smartphone-Apps gebaut
- Bild mit kompletter Architektur
\\ \\
Bosch hat viele Fabriken, da gibt es Maschinen, die müssen eingestellt werden für den Kunden. (Von oben immer weiter
rein gehen)
\\ \\
1 Seite

\colorbox{yellow}{Hier fehlt was}

\newpage

\section{Aufbau der Arbeit}
\label{sec:aufbauDerArbeit}
Dieses Kapitel dient zur schnelleren Orientierung innerhalb der Arbeit und zeigt auf, welche Themen die jeweiligen
Kapiteln behandeln.

\begin{description}

    \item[Kapitel 2 (Grundlagen)]\hfill \\
    Dieses Kapitel erklärt die Grundlagen. Es wird zum Beispiel der Unterschied zwischen den verschiedenen Cloudanbieter
    im Bereich Maschine-Learning erläutert, um was es sich bei TensorFlow handelt und was ein WebViewer ist. Außerdem
    sind die genutzten Bosch-Maschinen kurz vorgestellt.

    \item[Kapitel 3 (Neuronales Netz)]\hfill \\
    Das dritte Kapittel erläutert die Entwicklung des Machine-Learning Algorithmus in der Cloud. Dabei werden verschiedene
    Optionen für die einzelnen Teilaufgaben analysiert und im Anschluss die genutzten Programme installiert, eingerichtet
    und erläutert. Das trainierte Model wird abschließend in TensorFlow importiert und über einen Container zugänglich
    gemacht.

    \item[Kapitel 4 (Client)]\hfill \\
    Das vierte Kapitel setzt das in der Cloud trainierte Model voraus und implementiert dazu ein prototypisches Frontend.
    Dieses besteht aus einer Angular-Anwendung sowie zwei Smartphone-Apps. Dabei werden zu Anfang verschiedene Programme
    und Services analysiert, um diese im Anschluss zu nutzen.

    \item[Kapitel 5 (Adaptierbarkeit)]\hfill \\
    Hier werden Themen angesprochen, die für die weitere Verwendung oder Erweiterung des Machine-Learning Algorithmus
    oder des Frontends, entscheiden sind. Unter anderem wird erläutert, welche Schritte notwendig sind, um neue Maschinen
    oder Komponenten in die Architektur einzubauen oder wie optimale Daten zusammengestellt werden müssen.

    \item[Kapitel 6 (Ausblick)]\hfill \\
    Wie sowohl der Machine-Learning Algorithmus als auch das Frontend erweitert werden kann, zeigt dieses Kapitel auf.
    Dabei können neue Cloud-Services hinzugefügt oder auch weitere Funktionen der schon eingesetzten Services verwendet
    werden.

    \item[Kapitel 7 (Zusammenfassung)]\hfill \\
    Das letzte Kapitel enthält eine Zusammenfassung aller vorangegangenen Kapitel und ein Schlusswort.

\end{description}
\chapter{Adaptierbarkeit}
\label{ch:adaptierbarkeit}
Die in Kapitel~\ref{ch:neuronalesNetz} ab Seite~\pageref{ch:neuronalesNetz} aufgebaute und erstellte Architektur soll
zusammen mit dem in Kapitel~\ref{ch:client} ab Seite~\pageref{ch:client} implementierten Frontend durch weitere Ideen
erweitert und verbessert werden.

Eine Untersuchung soll ergeben, ob sowohl die Architektur als auch das Frontend so modular entwickelt wurden, dass sie 
problemlos durch weitere Funktionen und Ideen erweitert werden können. Eventuell Probleme im System oder in der
Architektur sollen gelößt werden.

Dies ist zwingend erforderlich um zum Beispiel eine weitere Maschine oder neue Module ebenfalls in das System zu
integrieren. Auch soll so auf neue Versionen oder neu herausgebrachte Maschinen reagieren werden.

Für dieses Ziel werden in den folgenden Kapitel mehrere Ideen für passende Erweiterungen behandelt und diese Schritt für 
Schritt umgesetzt und eingebaut.

Die Abbildung~\ref{fig:schematische_architektur_5} auf Seite~\pageref{fig:schematische_architektur_5} zeigt eine
schematische Erweiterung des Systems. Dort ist sehr gut ersichtlich, dass der API Connect Service eine zentrale Rolle
in der Erweiterbarkeit einnimmt.

\begin{figure}[h]
    \centering
    \includegraphics[width=\textwidth]{images/kapitel_5/architektur_schematisch.pdf}
    \caption{Schematische Darstellung der Adaptierbarkeit}
    \label{fig:schematische_architektur_5}
\end{figure}
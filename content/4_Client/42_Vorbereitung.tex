\section{Vorbereitung}
Damit im Weiteren das Frontend und die Smartphone-Apps entwickelt und gebaut werden können, werden drei Programme auf
dem Entwicklungsrechner installiert und ein Services in der IBM Cloud instanziiert sowie eingerichtet und verwaltet.

Die Vorgehensweise wird in den folgenden Kapiteln erläutert und Schritt für Schritt erklärt.

\subsection{Angular CLI}
Für die Erstellung, die Verwaltung und die Aktualisierung einer Angular-Applikation empfiehlt es sich das hauseigene
Command Line Interface (CLI) zu nutzen. Über den folgenden Befehl wird dieses auf einem Linux oder macOS Rechner
installiert.

\begin{lstlisting}[caption=Installation des Angular CLI, label=ls:vorbereitung_angularcli]
    $ npm install -g @angular/cli
\end{lstlisting}

Anschließend steht die Applikation über das Kommando \texttt{ng} zur Verfügung. Dies ist eine Abkürzung für
\textit{A\textbf{ng}ular}.

\subsection{Android Studio}
Für die Entwicklung einer Android-App stehen zahlreiche Tools zur Verfügung. Eine der bekanntesten ist das Android
Studio. Bei diesem Programm handelt es sich um eine IDE auf Basis von IntelliJ Community
Edition\footnote{https://www.jetbrains.com/idea} und ist kostenlos.

Die Installation von Android Studio erfolgt durch das Herunterladen des jeweiligen Installationspaketes auf der
Installationsseite\footnote{https://developer.android.com/studio/install}.

Nach erfolgreicher Installation kann man die IDE zum ersten Mal starten und ein neues Projekt anlegen.

\subsection{Xcode}
iOS-Apps kann man lediglich durch das Apple eigene Xcode entwickln. Die Insallation kann nur auf einem macOS
erfolgen und dort durch den internen App Store\footnote{https://itunes.apple.com/de/app/xcode/id497799835}.

Nach erfolgreicher Installation erscheint die IDE im Dock und ist bereit zur Nutzung.

\subsection{Node.js Runtime}
Damit man das erstellte Frontend online aufgerufen kann wird in der IBM Cloud eine zusätzliche Node.js Runtime
benötigt.

Genau wie in Kapitel~\ref{ssc:nodejs_runtime} auf Seite~\pageref{ssc:nodejs_runtime} beschrieben ist die Einrichtung
und Konfiguration einer solchen Runtime über den folgenden Befehl möglich:

\begin{lstlisting}[caption=Instanziierung der Node.js Runtime, label=ls:vorbereitung_nodejsdashboard]
    $ ibmcloud cf create-service nodejs service_name
\end{lstlisting}

Anschließend steht die Runtime in der Kategorie \textit{Cloud Foundry-Anwendungen} im IBM Cloud Dashboard zur Verfügung
und man kann sie nutzen.
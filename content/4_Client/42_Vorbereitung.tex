\section{Vorbereitung}
Damit im Weiteren das Frontend und die Smartphone-Apps entwickelt und gebaut werden können, werden drei Programme auf
dem Entwicklungsrechner installiert und ein Services in der IBM Cloud instanziiert. Die Vorgehensweise wird im
folgenden Kapitel erläutert.

\subsection{Angular CLI}
Für die Erstellung, Verwaltung und Aktualisierung einer Angular-Applikation, empfiehlt es sich das hauseigene Command
Line Interface (CLI) zu nutzen. Über den folgenden Befehl wird dieses auf einem Linux oder macOS Rechner installiert.

\begin{lstlisting}[language=bash, caption=Installation des Angular CLI, label=ls:vorbereitung_angularcli]
    $ npm install -g @angular/cli
\end{lstlisting}

Anschließend steht die Applikation über das Kommando \texttt{ng} zur Verfügung. Dies ist eine Abkürzung für
\textit{A\textbf{ng}ular}.

\subsection{Android Studio}
Für die Entwicklung einer Android-App stehen zahlreiche Tools zur Verfügung. Eine der bekanntesten ist das Android
Studio. Bei diesem Programm handelt es sich um eine IDE auf Basis von IntelliJ Community
Edition\footnote{https://www.jetbrains.com/idea} und ist kostenlos.

Die Installation von Android Studio erfolgt durch das herunterladen des jeweiligen Installationspaketes auf der
Installationsseite\footnote{https://developer.android.com/studio/install}.

Nach erfolgreicher Installation kann die IDE gestartet und ein neues Projekt angelegt werden.

\subsection{Xcode}
iOS-Apps können lediglich durch das Apple eigene Xcode entwicklet werden. Die Insallation kann nur auf einem macOS
erfolgen und dort durch den internen App Store\footnote{https://itunes.apple.com/de/app/xcode/id497799835?mt=12}.

Nach erfolgreicher Installation erscheint die IDE im Dock und kann genutzt werden.

\subsection{Node.js Runtime}
Damit das erstellte Frontend online aufgerufen werden kann, wird in der IBM Cloud eine zusätzliche Node.js Runtime
benötigt. Genau wie in Kapitel \ref{ssc:nodejs_runtime} auf Seite \pageref{ssc:nodejs_runtime} beschrieben, ist die
Einrichtung und Konfiguration der Runtime über den folgenden Befehl möglich.

\begin{lstlisting}[language=bash, caption=Instanziierung der Node.js Runtime, label=ls:vorbereitung_nodejsdashboard]
    $ ibmcloud cf create-service nodejs service_name
\end{lstlisting}

Anschließend steht die Runtime in der Kategorie \textit{Cloud Foundry-Anwendungen} im IBM Cloud Dashboard zur Verfügung
und kann genutzt werden.
\section{Vorbereitung}
Damit im Weiteren das Frontend entwickelt und gebaut werden kann, müssen ein Programm auf dem Entwicklungsrechner
installiert sowie ein IBM Cloud Service instanziiert werden. Die Vorgehensweise wird im folgenden Kapitel erläutert.

\subsection{Angular CLI}
Für die Erstellung, Verwaltung und Aktualisierung einer Angular-Applikation, wird die Hauseigene Command Line Interface
(kurz Angular CLI) genutzt. Diese kann über den folgenden Befehl auf Linux und macOS installiert und eingerichtet werden:

\begin{lstlisting}[language=bash, caption=Installation Angular CLI, label=Installation Angular CLI]
$ npm install -g @angular/cli
\end{lstlisting}

Anschließend steht die Applikation über das Kommando \path{ng} zur Verfügung.

\subsection{Node.JS Runtime}
Damit auch das erstellte Frontend in der Cloud laufen kann, wird in der IBM Cloud eine zusätzliche Node.JS-Runtime
benötigt. Genau wie in Kapitel \ref{ssc:nodejs_runtime} auf Seite \pageref{ssc:nodejs_runtime} beschrieben, kann diese
mit dem folgenden Befehl eingerichtet werden:

\begin{lstlisting}[language=bash, caption=Instanziierung der Node.JS Runtime, label=Instanziierung der Node.JS Runtime]
$ ibmcloud cf create-service nodejs service_name
\end{lstlisting}
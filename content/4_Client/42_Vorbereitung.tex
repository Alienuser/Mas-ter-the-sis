\section{Vorbereitung}
Damit im Weiteren das Frontend entwickelt und gebaut werden kann, wird ein Programm auf dem Entwicklungsrechner
installiert und zwei Services in der IBM Cloud instanziiert. Die Vorgehensweise wird im folgenden Kapitel erläutert.

\subsection{Angular CLI}
Für die Erstellung, Verwaltung und Aktualisierung einer Angular-Applikation, empfiehlt es sich das Hauseigene Command
Line Interface zu nutzen. Über den folgenden Befehl wird dieses auf einem Linux oder macOS Rechner installiert.

\begin{lstlisting}[language=bash, caption=Installation des Angular CLI, label=Installation des Angular CLI]
$ npm install -g @angular/cli
\end{lstlisting}

Anschließend steht die Applikation über das Kommando \texttt{ng} zur Verfügung. Dies ist eine Abkürzung für
\textit{A\textbf{ng}ular}.

\subsection{Node.js Runtime}
Damit das erstellte Frontend online aufgerufen werden kann, wird in der IBM Cloud eine zusätzliche Node.js-Runtime
benötigt. Genau wie in Kapitel \ref{ssc:nodejs_runtime} auf Seite \pageref{ssc:nodejs_runtime} beschrieben, ist die
Einrichtung und Konfiguration der Runtime über den folgenden Befehl möglich.

\begin{lstlisting}[language=bash, caption=Instanziierung der Node.js Runtime, label=Instanziierung der Node.js Runtime]
$ ibmcloud cf create-service nodejs service_name
\end{lstlisting}

Anschließend steht die Runtime in der Kategorie \textit{Cloud Foundry-Anwendungen} im IBM Cloud Dashboard zur Verfügung.
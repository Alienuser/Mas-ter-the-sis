\section{Analyse}
Dieses Kapitel soll verschiedene Umsetzungsmöglichkeiten für die Webseite als auch für die Smartphone-Apps analysieren
und eine Einleitung für die präferierte Möglichkeit geben.

Ziel ist es für sowohl die Webseite als auch für die Smartphone-Apps eine optimale Möglichkeit zu finden, diese
umzusetzen, zu verwalten und in Betrieb zu nehmen.

Mit den Ergebnissen wird in den darauffolgenden Kapiteln fortgesetzt und sie werden Schritt für Schritt umgesetzt.

\subsection{Architektur Webseite}
Für die Entwicklung von Webseiten existieren zahlreiche Frameworks, welche wiederum zahlreiche Architekturen umsetzen.
Das Model-View-Controler-Pattern (kurz MVC) hat sich in der Vergangenheit großflächig durchgesetzt (siehe
hierzu~\cite{book_grundlagen_mvc}) und soll in diesem Beispiel Verwendung finden.

Einer der größten Vertreter des MVC-Patterns ist Angular oder AngularJS. Ersteres ist die Weiterentwicklung und
mittlerweile in Version 6 Verfügbar. Neben der sehr schnellen Einarbeitung ist Angular auch sehr
mächtig was die mitgelieferten Funktionen und Helfer angeht.

Es ist darauf spezialisiert Single-Page-Applications (kurz SPA) zu bauen, was einen schnellen Wechsel zwischen den
einzelnen Seiten nach sich zieht. Bei dem zu entwickelnden Frontend soll es sich um eine SPA handeln.

\subsection{Runtime}
Um das Frontend in der IBM Cloud laufen zu lassen, benötigt man eine Runtime in der die Anwendung bestehen kann. Zur
Auswahl stehen neben \textit{Cloud Foundry} auch ein \textit{Docker-Container} oder der Service \textit{Object Storage}.

Der Object Storage Service speichert belibige Daten online. Diese kann man dann mittels statischer URLs aufrufen und
nutzen. Um eine Angular Seite damit zu hosten muss man alle benötigten Daten hochladen und anschließend referenzieren.
Dieses Vorgehen ist relativ mühsam, da eine Referenzierung in der lokalen IDE erst dann möglich ist, wenn man die Daten
hochgeladen hat.

Alternativ kann man die Anwendung auch in ein Docker Container laden und mittels eines Apache oder Nginx Servers
ausliefern. Dies bedeutet allerdings, dass man ein Dockerimage bauen muss, welches die Anwendung dann aus einem
Repository lädt.

Eine sehr einfache Möglichkeit ist das Nutze eines Cloud Foundry Containers. Für diese gibt es, genau wie für
Dockerimages, zahlreiche Buildpacks. Ein Buildpack enthält eine NodeJS-Runtime, mit der man die Angular-Anwendung
problemlos ausliefern lassen kann.

Zudem ermöglicht der Cloud Foundry Container vereinfachten Zugriff auf die Cloud Foundry
Environment-Variablen\footnote{https://docs.run.pivotal.io/devguide/deploy-apps/environment-variable.html} (kurz env)
auf die man über die VCAP-Variablen zugreifen kann.

Über diese Variablen kann man zum Beispiel Zugangsdaten für die Verbundenen Cloud Services auslesen, um diese nicht in
die Anwendung eintragen zu müssen. Auch wenn sich beispielsweise die Zugangsdaten für die Services ändern, kann die
Anwendung problemlos weiterlaufen.

\subsection{Smartphone App}
Eine Möglichkeit ist die Entwicklung nativer Smartphone Apps mit der jeweiligen Programmiersprache und einer
anschließenden installation auf einem Endgerät. In diesem Fall müsste man zwei seperate Apps entwickeln.

Der Nachteil der Variante ist die Tatsache, dass man bei jeglicher Änderungen am Design oder Anpassungen der
REST-Schnittstelle eine neue Version der beiden Apps entwickeln muss und sie neu im entsprechenden Store veröffentlichen
muss.

Alternativ kann man eine native App für die jeweiligen Systeme bauen, welche allerdings als Layout eine Vollbild WebView
enthält. Dies ermöglicht das laden des aktuellen Web-Frontends aus dem Internet und eine anschließende Darstellung in
der App. Dem Nutzer wird so suggeriert, als ob er eine komplett nativ geschriebene App nutzt.

Sofern die App ähnlich einer Smartphone App gebaut und designt ist und sich ähnlich verhält, wird dem Endnutzer dieser
Unterschied allenfalls in der Geschwindigkeit der App auffallen. Diese ist gerade beim Start etwas langsamer, da der
Nutzer die komplette Webseite herunterladen muss.

Sofern die Webseite allerdings das erste Mal komplett heruntergeladen ist, wird diese auf dem Smartphone gecached um sie
schneller zu öffnen. Dabei hilft auch der entwickelte Service Worker, der sich nach dem Start dann nach Aktualisierungen
der Webseite umschaut und diese gegebenenfalls im Hintergrund herunterlädt.

Auch der eingerichtete offline Modus ist für die App interessant, da auch dieser hier problemlos funktioniert.
Allerdings kann man im offline Modus natürlich keine Vorhersagen starten und es wird mit einer Fehlermeldung abgebrochen.

Da es sich teilweise um eine native App handelt, kann man diese Problemlos in den Store laden. Lediglich eine
Aktualisierung am Design muss man nicht mit einer neuen Version der App einleiten.

Dies hat den großen Vorteil, dass man kommende Änderungen lediglich an der Webseite umsetzen muss und die beiden Apps
bekommen diese Aktualisierungen direkt mit und funktionieren weiterhin.

Da es sowohl für Android als auch iOS ein WebView-Layout gibt, kann man beide Apps damit implementieren und so den
Entwicklungsaufwand auf ein Minimum reduzieren.

In diesem Beispiel wird eine WebView-App entwickelt, da man so dass schon geschriebene Web-Frontend wiederverwenden
kann und sich dadurch die Entwicklungszeit und Wartung deutlich verkürzt.

\subsection{Frontend-Design}
Genau wie bei der Architektur einer Anwendung gibt es auch im Bereich des Designs einer Webseite zahlreiche Frameworks.
Nach t3n~\cite{online_analyse_css} ist \textit{Bootstrap} das wohl bekannteste CSS Framework, welches aktuell existiert.

Da man in diesem Beispiel allerdings das Web-Frontend auch als App nutzen möchte, muss es eine sehr große Ähnlichkeit zu
diesen besitzen.

Aus diesem Grund ist es sinnvoller das Google Material Design\footnote{https://material.angularjs.org/latest} zu nutzen.
Für Angular existiert eine Bibliothek, welche die Material-Design Spezifika\footnote{https://material.io/guidelines} in
Komponenten umsetzt und zum einfachen kopieren zur Verfügung stellt.
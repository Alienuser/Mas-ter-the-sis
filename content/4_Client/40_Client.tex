\chapter{Client}
\label{ch:client}
Das nachfolgende Kapitel beschreibt die Entwicklung einer Applikation, dem Client, welches die in
Kapitel~\ref{ch:neuronalesNetz} ab Seite~\pageref{ch:neuronalesNetz} implementierte Architektur nutzt, um Anfragen an
die neuronalen Netze für den Endnutzer einfacher zu gestalten.

Dabei ist die Anwendung für die Eingabe der benötigten Parameter zuständig und das Abfangen von falsch eingetragenen
Werten in das dargestellte Formular. Anschließend überträgt der Client die eingetragenen Werte an die neuronalen Netze
und stellt die Vorhersagen aufbereitet dar.

Die Architektur beherbergt zwar zwei Anwendungen mit jeweils einem Endpunkt, allerdings sind diese gemeinsam über
den API Connect Service aufrufbar. Dies erleichtert die Entwicklung des Clients dahingehend, dass lediglich eine
Kommunikation zwischen der neuen Applikation und dem Bluemix-Service notwendig ist.

Da, wie in~\cite{online_client_apps} zu lesen, Smartphone-Apps eine immer wichtigere Rolle im Leben eines Endnutzers
spielen, soll neben einer Webanwendung (englisch Frontend) auch für die beiden größten Hersteller von
Smartphone-Betriebssystemen eine App entstehen. Somit soll ein noch viel größerer Kundenbereich abgedekt werden.

Das Frontend soll in einem Container laufen um ihn möglichst schnell und komfortabel sowohl in der Cloud als auch 
im eigenen Rechenzentrum betreiben zu können. Die Abbilung~\ref{fig:schematische_architektur_4} auf
Seite~\pageref{fig:schematische_architektur_4} zeigt die schematische Architektur des Frontends mit den ausgehenden
Verbindungen mit den Anfragen.

Die notwenigen Schritte für den Aufbau eines Containers sollen in einem Continuous Deployment verankert werden, um sie
möglichst Modular zu halten.

\begin{figure}[h]
    \centering
    \includegraphics[scale=0.5]{images/kapitel_4/architektur_schematisch.pdf}
    \caption{Schematische Darstellung der Architektur}
    \label{fig:schematische_architektur_4}
\end{figure}
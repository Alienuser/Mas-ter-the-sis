\chapter{Ausblick}
\label{ch:ausblick}
Dieses Kapitel soll einen Überblick über weitere größere Funktionen und Erweiterungen geben, durch die die Architektur,
die Webanwendung und die Smartphone-Apps erweitert und verbessert werden können.

Außerdem sollen Möglichkeiten aufgezeigt werden, welche die Arbeit mit Maschinen und künstlicher Intelligenz
vereinfacht. Dazu zählt unter anderem eine Möglichkeit, eine direkte Verbindung zwischen Maschine und künstlicher
Intelligenz herzustellen.

Auch sind Themen, welche die klassische Container-Architektur verbessern für Erweiterungen interessant oder
Möglichkeiten zur schnelleren Vorhersage von Daten und eingabe von Daten unter erschwerten Bedingungen.

%% TODO noch schreiben
\section{Eine direkte Verbindung}
Daten in Maschine generrieren, IoT hochladen, in Maschine Learning über Connections einbauen und immer wieder
trainieren.

Allerdings sind automatisierte Tests nicht immer und für jede Anwendung geeignet. Michael Lüttel von der Deutschen 
Flugsicherung sagte auf der iqnite-Konferenz in Düsseldorf \enquote{Automatisierung macht nur dann Sinn, wenn sie mehr 
Aufwände einspart als sie selbst erzeugt.}\footnote{https://www.qz-online.de/news/uebersicht/nachrichten/vor-und-nachteile-von-automatisierten-software-tests-890130.html}

\colorbox{yellow}{Hier fehlt was}

\begin{figure}[h]
    \centering
    \includegraphics[width=\textwidth]{images/kapitel_6/architektur_uebersicht.pdf}
    \caption{Übersicht der Zielarchitektur}
    \label{fig:ausblick_uebersicht}
\end{figure}

%% TODO noch schreiben
\section{Function as a Service}
Wie könnte man Functions/Lambda einbauen. FaaS

\colorbox{yellow}{Hier fehlt was}

%% TODO noch schreiben
\section{Machine Learning für Smartphones}
Wie könnte man das Feature nutzen und was würde das bringen

\colorbox{yellow}{Hier fehlt was}

%% TODO noch schreiben
\section{Offline Modus}
Ein Offline-Mode für die Website mit TensorFlow.js.

\colorbox{yellow}{Hier fehlt was}

%% TODO noch schreiben
\section{AI OpenScale}
\label{ai_openscale}
Hiermit kann man veranschaulichen, warum und wie die AI auf das Ergebnis gekommen ist. Das kann man mit dem Deployment
und mit dem TensorFlow.js Ding machen. Dafür braucht man eine Datenbank.

\colorbox{yellow}{Hier fehlt was}

%% TODO nosch schreiben
\section{Skalierbarkeit}
Wie könnte man die Anwendung skalieren?

\colorbox{yellow}{Hier fehlt was}

%% TODO noch schreiben
\section{Audit mit Blockchain}
Für die Waage muss Audit gemacht werden. Könnte man Blockchian da nicht noch einbauen?

\colorbox{yellow}{Hier fehlt was}

%% TODO noch schreiben
\section{Nutzen von IoT}
Nutzen von IoT

\colorbox{yellow}{Hier fehlt was}

\begin{figure}[h]
    \centering
    \includegraphics[width=\textwidth]{images/kapitel_6/iot_waage.pdf}
    \caption{Übersicht der Zielarchitektur}
    \label{fig:ausblick_iot}
\end{figure}

%% TODO noch schreiben
\subsection{Daten einlesen}
Wie könnten die Daten, welche durch das Netzwerk herausgefunden werden, auch automatisch in die Machine eingegeben werden?

\colorbox{yellow}{Hier fehlt was}

%% TODO noch schreiben
\subsection{Daten auslesen}
Wie könnte man Daten der Maschine auslesen, damit man sie nutzen kann um das Neuronale Netz weiter zu verbessern?

\colorbox{yellow}{Hier fehlt was}